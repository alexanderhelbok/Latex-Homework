% !TeX document-id = {7b7d88a4-0972-4202-9c89-33327d1050f5}
% !TeX spellcheck = en_EN
\documentclass[10pt, dvipsnames]{beamer}
\usepackage[utf8]{inputenc}
\usepackage{hyperref}
\usepackage{fontawesome5, lipsum}
\usepackage{graphicx}
\usepackage[USenglish]{babel}
\usepackage{setspace, multicol, makecell, booktabs, multirow}
\usepackage{mathtools, amsthm, amsfonts, siunitx, physics, chemformula, empheq}
\usepackage{mtpro2, pifont}
\pdfmapfile{=mtpro2.map}
\usepackage{enumitem}
\usepackage[backend=biber, abbreviate=true, doi=false, style=numeric-comp, giveninits=true, sorting=none]{biblatex}

\usepackage[font=footnotesize, bf, format=plain, font=normalsize]{caption}
\graphicspath{ {./bilder/} }
\addbibresource{literatur.bib}

%\catcode`_=\active
%\newcommand_[1]{\ensuremath{\sb{\fontfamily{Times New Roman}\selectfont \mathrm{#1}}}}

\AtBeginDocument{\sisetup{per-mode = symbol, sticky-per, mode=text, text-font-command=\fontfamily{Times New Roman}\selectfont}}
\let\oldunit\unit
\renewcommand{\unit}[1]{\hspace{4pt}\oldunit{#1}}
\DeclareSIUnit[]\ko{\kilo\ohm}

\newcommand{\nair}{n\sb{\fontfamily{Times New Roman}\selectfont \mathrm{air}}}

\usepackage{scalerel,stackengine}
\newcommand\equalhat{\mathrel{\stackon[1.5pt]{=}{\stretchto{%
				\scalerel*[\widthof{=}]{\wedge}{\rule{1ex}{3ex}}}{0.5ex}}}}

% ------------------------------------------------------------------------------
% Use the beautiful metropolis beamer template
% ------------------------------------------------------------------------------
\usepackage[T1]{fontenc}
%\usepackage{FiraSans} 
\mode<presentation>
{
	\usetheme[progressbar=frametitle,numbering=fraction,background=light]{metropolis} 
	\usecolortheme{default} % or try albatross, beaver, crane, ...
	\usefonttheme[onlymath]{serif}  % or try serif, structurebold, ...
	\setbeamertemplate{navigation symbols}{}
	\setbeamertemplate{caption}[numbered]
	\setbeamertemplate{section in toc}[sections numbered]
	\setbeamerfont{frametitle}{size=\LARGE}
	\metroset{block=fill}
} 

\makeatletter
\setlength{\metropolis@progressinheadfoot@linewidth}{3pt}
\setlength{\metropolis@titleseparator@linewidth}{3pt}
\setlength{\metropolis@progressonsectionpage@linewidth}{3pt}

\makeatother

% ------------------------------------------------------------------------------
% tcolorbox / tcblisting
% ------------------------------------------------------------------------------	
\definecolor{MyBlue}{HTML}{0072B2}
\definecolor{MyOrange}{HTML}{D55E00}
\definecolor{MyRed}{HTML}{F00F0F}

\setstretch{1.5}

\usepackage{multimedia}


\title{Das Hubble Gesetz}
\author{Alexander Helbok\\Experiment date: May 4, 2023}
\date{}

\begin{document}
	\maketitle
	\begin{frame}[noframenumbering, plain]{Overview}
		\tableofcontents
	\end{frame}

	\begin{frame}{Setting 1925}
		Zitat beide seiten (statisches universum + dynamisches)
		Gr war draußen, sagt dynamisch voraus, statisch instabil
	\end{frame}
	
%	########### Theorie #############
	\begin{frame}{Enternungsbestimmung über Standardkerzen}
		Cepheiden in nebulae (H. Swan Leavitt relation)
	\end{frame}

%	########## Methodik ############
	\begin{frame}{Messdaten}
		von slipher am … aufgenommen
		optisches teleskop
	\end{frame}
	
%	########## ergebnisse ##########
	\begin{frame}{Ergebnisse}
		originaler plot, obwohl viel scattering klarer zusammenhang
	\end{frame}
	
%	######### 1st Interpretation ########
	\begin{frame}{Intrepretation}
		de Sitter 
	\end{frame}
	
%	######### folgeexperimente #########
	\begin{frame}{Messung über SN1a}
		Vermessung üer Typ 1a supernovae (standardkerzen)
	\end{frame}
	
%	

%	\begin{frame}{Introduction}
%	\begin{itemize}[label={--}, itemindent=0.5cm]
%		%\item Daten sind Farbcodiert \\ \hspace{1cm}\ding{241} {\color{MyBlue}Eingangsspannung}, {\color{black}Ausgangsspannung}, {\color{MyOrange}Fits}
%		\item color coding: {\color{Black}Data} and {\color{MyBlue}Fit}
%		\item used Python/Julia for the Analysis (SciPy, NumPy)
%		\item error propagation fully automatic with Uncertainties/Measurements package \cite{Hubble1929}\cite{Hubble1931}\cite{betoule2014improved}
%	\end{itemize}
%\end{frame}

\begin{frame}[standout]
	\Huge Fragen?
\end{frame}

\begin{frame}{Sources}
	\printbibliography
\end{frame}

\end{document}