\documentclass{alex_hü}

\name{Alexander Helbok}
\course{PS Physik}
\hwnumber{9}


\begin{document}
\renewcommand{\labelenumi}{\alph{enumi})}


\begin{mybox}{1. Magnetfeld eines asymmetrischen Leiters}
	\centering \(  \)
	\tcblower
	\begin{enumerate}
		\item \( r = \sqrt{x^2+y^2} \)
		\begin{flalign*}
			I_+ &= I(1 + \tfrac{a^2}{R^2}) &&\\
			I_- &= -I\tfrac{a^2}{R^2} &&\\
			B_+ &= \tfrac{\mu_0I_+}{2r\pi} = \tfrac{\mu_0I_+}{2\pi\sqrt{x^2+y^2}} &&\\
			B_- &= \tfrac{\mu_0I_-}{2r\pi} = \tfrac{\mu_0I_-}{2\pi\sqrt{(x - b)^2+y^2}} &&\\
			B &= B_+ + B_- = \tfrac{\mu_0I_+}{2\pi\sqrt{x^2+y^2}} + \tfrac{\mu_0I_-}{2\pi\sqrt{(x - b)^2+y^2}} = &&\\ 
			&= \tfrac{\mu_0I}{2\pi}\left[ \tfrac{1}{\sqrt{x^2+y^2}} + \tfrac{a^2}{R^2} \left( \tfrac{1}{\sqrt{x^2+y^2}} - \tfrac{1}{\sqrt{(x - b)^2+y^2}} \right) \right] &&\\
			B(2R, 0) &= \dl{\tfrac{\mu_0I}{4R\pi}\left[ 1 + \tfrac{a^2}{R} \left( \tfrac{1}{R} - \tfrac{1}{R - b} \right) \right]} &&
		\end{flalign*}
	\tcbline
		\item \(  \)
		\begin{flalign*}
			B(0, 2R) &= \dl{\tfrac{\mu_0I}{2R\pi}\left[ \tfrac{1}{2} + \tfrac{a^2}{R} \left( \tfrac{1}{2R} - \tfrac{1}{\sqrt{b^2+4R^2}} \right) \right]} &&
		\end{flalign*}
	\end{enumerate}
\end{mybox}

%\begin{mybox}{193. Gekoppelte physikalische Pendel}
%	\centering \( m = 5 \unit{kg};\quad k = 2 \unit{N/m};\quad a = 0.1 \unit{m};\quad h = 0.4 \unit{m};\quad l = 0.1 \unit{m} \)
%	\tcblower
%	\begin{enumerate}
%		\item \( \omega = \sqrt{\tfrac{mgl}{I}} \)
%		\begin{flalign*}
%			I &= \tfrac{1}{12}m(a^2 + h^2) + ml^2 = 0.12 \unit{kg.m^2} &&\\
%			\omega_a &= \sqrt{\tfrac{mgl}{I}} = \dl{6.37 \unit{rad/s}} &&
%		\end{flalign*}
%		\tcbline
%		\item \( F = -m\omega^2 x;\quad \Delta x = x_1 - x_2 \)
%		%		\begin{flalign*}
%			%
%			%		\end{flalign*}
%	\end{enumerate}
%\end{mybox}
%
%\begin{mybox}{193. Gekoppelte physikalische Pendel}
%	\centering \( m = 5 \unit{kg};\quad k = 2 \unit{N/m};\quad a = 0.1 \unit{m};\quad h = 0.4 \unit{m};\quad l = 0.1 \unit{m} \)
%	\tcblower
%	\begin{enumerate}
%		\item \( \omega = \sqrt{\tfrac{mgl}{I}} \)
%		\begin{flalign*}
%			I &= \tfrac{1}{12}m(a^2 + h^2) + ml^2 = 0.12 \unit{kg.m^2} &&\\
%			\omega_a &= \sqrt{\tfrac{mgl}{I}} = \dl{6.37 \unit{rad/s}} &&
%		\end{flalign*}
%		\tcbline
%		\item \( F = -m\omega^2 x;\quad \Delta x = x_1 - x_2 \)
%		%		\begin{flalign*}
%			%
%			%		\end{flalign*}
%	\end{enumerate}
%\end{mybox}


\end{document}