\documentclass{alex_hü}

\name{Alexander Helbok}
\course{PS Physik}
\hwnumber{9}


\begin{document}
\renewcommand{\labelenumi}{\alph{enumi})}


\begin{mybox}{1. Magnetfeld eines asymmetrischen Leiters}
	\centering \(  \)
	\tcblower
	\begin{enumerate}
		\item \( r = \sqrt{x^2+y^2} \)
		\begin{flalign*}
			I_+ &= I(1 + \tfrac{a^2}{R^2}) &&\\
			I_- &= -I\tfrac{a^2}{R^2} &&\\
			B_+ &= \tfrac{\mu_0I_+}{2r\pi} = \tfrac{\mu_0I_+}{2\pi\sqrt{x^2+y^2}} &&\\
			B_- &= \tfrac{\mu_0I_-}{2r\pi} = \tfrac{\mu_0I_-}{2\pi\sqrt{(x - b)^2+y^2}} &&\\
			B &= B_+ + B_- = \tfrac{\mu_0I_+}{2\pi\sqrt{x^2+y^2}} + \tfrac{\mu_0I_-}{2\pi\sqrt{(x - b)^2+y^2}} = &&\\ 
			&= \tfrac{\mu_0I}{2\pi}\left[ \tfrac{1}{\sqrt{x^2+y^2}} + \tfrac{a^2}{R^2} \left( \tfrac{1}{\sqrt{x^2+y^2}} - \tfrac{1}{\sqrt{(x - b)^2+y^2}} \right) \right] &&\\
			B(2R, 0) &= \dl{\tfrac{\mu_0I}{4R\pi}\left[ 1 + \tfrac{a^2}{R} \left( \tfrac{1}{R} - \tfrac{1}{R - b} \right) \right]} &&
		\end{flalign*}
	\tcbline
		\item \(  \)
		\begin{flalign*}
			B(0, 2R) &= \dl{\tfrac{\mu_0I}{2R\pi}\left[ \tfrac{1}{2} + \tfrac{a^2}{R} \left( \tfrac{1}{2R} - \tfrac{1}{\sqrt{b^2+4R^2}} \right) \right]} &&\\
		\end{flalign*}
	\end{enumerate}
\end{mybox}

\begin{mybox}{2. Induktion}
	\centering \( \vec{B} = B_x\hat{x} \)
	\tcblower
	\begin{enumerate}
		\item \(  \)
		\begin{flalign*}
			\varPhi(t) &= \uint{\vec{B}}{\vec{A}} = B\cos(\omega t)\uint{1}{\vec{A}} = r^2\pi B\cos(\omega t) &&\\
			U(t) &= -\dv{\varPhi}{t} = \dl{r^2\pi\omega B\sin(\omega t)} &&
		\end{flalign*}
	\tcbline
		\item \(  \)
		For \( \alpha = \tfrac{\pi}{4} \) current I will flow counterclockwise to increase the B-Field \\
		For \( \alpha = \tfrac{3\pi}{4} \) current I will flow clockwise to counter the B-Field
	\tcbline
		\item 
		\begin{flalign*}
			I &= \tfrac{U}{R} &&\\
			\dv{Q}{t} &= -\tfrac{1}{R}\dv{\varPhi}{t} &&\\
			\Delta Q &= -\tfrac{r^2\pi B}{R} \uint[0,\pi/2\omega]{\cos(\omega t)}{t} = \dl{\tfrac{r^2\pi B}{R}} &&
		\end{flalign*}
	\end{enumerate}
\end{mybox}

\begin{mybox}{3. Indunktionsspannung - Stab}
	\centering \( I := I_{\text{Bat}} \)
	\tcblower
	\begin{enumerate}
		\item \(  \)
		\begin{flalign*}
			\varPhi &= Blx &&\\
			U &= -Blv &&\\
			I_{\text{Ind}} &= -\tfrac{Blv}{R} &&\\
			\vec{F}_{\text{Ind}} &= I_{\text{Ind}}\vec{l} \times \vec{B} = -\tfrac{B^2l^2v}{R} \hat{x} &&\\
			\vec{F}_L &= I\vec{l} \times \vec{B} = IlB \hat{x} &&\\
			\vec{F}_{\text{ges}}(v) &= \vec{F}_{\text{Ind}} + \vec{F}_L = \dl{lB(I - \tfrac{Blv}{R}) \hat{x}} &&
		\end{flalign*}
	\tcbline
		\item \(  \)
		\begin{flalign*}
			m\dv{v}{t} &= lB\left(I - \tfrac{Bl}{R}v(t)\right) &&\\
			v(t) &= \tfrac{IR}{Bl}\left( 1 - \mathrm{e}^{\supfrac[-]{B^2l^2}{mR}t} \right) &&\\
			\lim_{t \to \infty} v(t) &= \dl{\tfrac{IR}{Bl}} &&
		\end{flalign*}
	\tcbline
		\item 
		\begin{flalign*}
			I_{\text{ges}} &= I + I_{\text{Ind}} = I - \tfrac{Blv}{R} = I - \tfrac{BlRI}{BlR} = \dl{0} &&
		\end{flalign*}
	\end{enumerate}
\end{mybox}


\end{document}