\documentclass{alex_hü}

\usepackage{array}   % for \newcolumntype macro
\newcolumntype{L}{>{$}c<{$}} % math-mode version of "l" column type

\name{Alexander Helbok}
\course{PS Physik}
\hwnumber{12}


\begin{document}
\renewcommand{\labelenumi}{(\alph{enumi})}


\begin{mybox}{1. Elektromagnetische Welle im Vakuum}
	\centering \( E_x = 0;\quad E_y = 30\cos(2\pi * 10^8 t - \tfrac{2\pi}{3}x);\quad E_z = 0 \)
	\tcblower
	\begin{enumerate}
		\item \( \omega = 2\pi * 10^8 \unit{\per\s} \)
		\begin{flalign*}
			f &=  \tfrac{\omega}{2\pi} = \dl{10^8 \unit{\per\s}} &&
		\end{flalign*}
	\tcbline
		\item \( k = \tfrac{2\pi}{3} \)
		\begin{flalign*}
			\lambda &= \tfrac{2\pi}{k} = \dl{3 \unit{m}} &&
		\end{flalign*}
	\tcbline
		\item \(  \)
		\begin{flalign*}
			\text{Direction: }\dl{\hat{x}} &&
		\end{flalign*}
	\tcbline
		\item \( B_0 = \tfrac{E_0}{c} \)
		\begin{flalign*}
			\vec{B} &= \dl{10^7 \cos(2\pi * 10^8 t - \tfrac{2\pi}{3}x)} &&
		\end{flalign*}
	\end{enumerate}
\end{mybox}

\begin{mybox}{2. Photonen-Ping-Pong}
	\centering \(  \)
	\tcblower
	\begin{enumerate}
		\item \(  \)
		\begin{tabular}{ L | L | L }
			\vec{E}_1(t) = E_0\vector{\cos(\omega t)\\ 0\\ 0} & \vec{E}_2(t) = E_0\vector{\cos(\omega t)\\ \sin(\omega t)\\ 0} & \vec{E}_3(t) = E_0\vector{a\cos(\omega t)\\ b\sin(\omega t)\\ 0} \\[2em]
			w_{rms} = \tfrac{\epsilon_0E_0^2}{2} & w_{rms} = \epsilon_0E_0^2 & w_{rms} = \tfrac{\epsilon_0E_0^2(a^2 + b^2)}{2} \\[1em]
			I_{rms} = \tfrac{c\epsilon_0E_0^2}{2} & I_{rms} = c\epsilon_0E_0^2 & I_{rms} = \tfrac{c\epsilon_0E_0^2(a^2 + b^2)}{2} \\[1em]
			\text{Linear Polarization} & \text{Circular Polarization} & \text{Elliptical Polarization} \\
		\end{tabular}
	\tcbline
		\item \( \vec{F} = \dv{\vec{p}}{t} = \tfrac{P}{c} = \epsilon_0 E_{\text{rms}}^2A \)
		\begin{flalign*}
			F_1 &= \epsilon_0 E_{\text{rms}}^2A &&\\
			F_2 &= 2\epsilon_0 E_{\text{rms}}^2A &&\\
			F_3 &= 
		\end{flalign*}
	\tcbline
		\item \(  \)
%		\begin{flalign*}
		%			
%		\end{flalign*}
	\end{enumerate}
\end{mybox}

\begin{mybox}{3. Stehende Wellen}
	\centering \( \vec{E}(\vec{r}, t) = \Re\left( \vec{E}_0\, \expo[\iu\vec{k}\vec{r} - \iu\omega t] \right) \)
	\tcblower
	\begin{enumerate}
		\item \(  \)
		\begin{flalign*}
			\vec{E}_1(x\hat{x}, t) &=  \Re\left( \vec{E}_0\, \expo[\iu k_x x - \iu\omega t] \right) &&\\
			\vec{E}_2(-x\hat{x}, t) &=  \Re\left( \vec{E}_0\, \expo[-\iu k_x x - \iu\omega t] \right) &&
		\end{flalign*}
		\begin{flalign*}
			\vec{E} = \vec{E}_1 + \vec{E}_2 &= \Re\left( \vec{E}_0\, \left[ \expo[\iu k_x x - \iu\omega t] + \expo[-\iu k_x x - \iu\omega t] \right]\right) = &&\\
			&= \Re\left( \vec{E}_0\, \expo[-][\iu\omega t] \right) * 2\cos(k_x x) &&
		\end{flalign*}
		For \( x = \tfrac{n\pi}{k_x},\ n \in \mathbb{N}:\quad \cos(k_x x) = 0\quad \Rightarrow\quad \vec{E} = 0\) \\[1em]
		\( \Rightarrow \vec{E} \) describes a standing Wave
	\tcbline
		\item \(  \)
		\begin{flalign*}
			\vec{E}_1(x\hat{x}, t) &=  \Re\left( \vec{E}_0\, \expo[\iu k_x x - \iu\omega t] \right) &&\\
			\vec{E}_2(y\hat{y}, t) &=  \Re\left( \vec{E}_0\, \expo[\iu k_y y - \iu\omega t] \right) &&
		\end{flalign*}
		\begin{flalign*}
			\vec{E} = \vec{E}_1 + \vec{E}_2 &= \Re\left( \vec{E}_0\, \left[ \expo[\iu k_x x - \iu\omega t] + \expo[\iu k_y y - \iu\omega t] \right]\right) = &&\\
			&= \Re\left( \vec{E}_0\, \expo[-][\iu\omega t] \right) * 2\cos(k_x x) &&
		\end{flalign*}
		For \( x = \tfrac{n\pi}{k_x},\ n \in \mathbb{N}:\quad \cos(k_x x) = 0\quad \Rightarrow\quad \vec{E} = 0\) \\[1em]
		\( \Rightarrow \vec{E} \) describes a standing Wave
	\tcbline
		\item \(  \)
%		\begin{flalign*}
%			
%		\end{flalign*}
	\tcbline
		\item \( \omega_1 = 2\omega_2 \)
		\begin{flalign*}
			\vec{E}_1(\hat{x}, t) &=  \Re\left( \vec{E}_0\, \expo[\iu\vec{k}\hat{x} - \iu\omega_1 t] \right) &&\\
			\vec{E}_2(-\hat{x}, t) &=  \Re\left( \vec{E}_0\, \expo[-\iu\vec{k}\hat{x} - \iu\omega_2 t] \right) &&\\
			\vec{E} &= \vec{E}_1 + \vec{E}_2 = \Re\left( \vec{E}_0\, \expo[\iu\vec{k}\hat{x} - \iu2\omega_2	 t] + \expo[-\iu\vec{k}\hat{x} - \iu\omega_2 t] \right) &&\\
		\end{flalign*}
	\end{enumerate}
\end{mybox}


\end{document}