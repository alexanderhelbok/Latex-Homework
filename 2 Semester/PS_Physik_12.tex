\documentclass{alex_hü}

\name{Alexander Helbok}
\course{PS Physik}
\hwnumber{12}


\begin{document}
\renewcommand{\labelenumi}{(\alph{enumi})}


\begin{mybox}{1. Elektromagnetische Welle im Vakuum}
	\centering \(  \)
	\tcblower
	\begin{enumerate}
		\item \(  \)
%		\begin{flalign*}
%			
%		\end{flalign*}
	\tcbline
		\item \(  \)
%		\begin{flalign*}
%		
%		\end{flalign*}
	\tcbline
		\item \(  \)
%		\begin{flalign*}
%			
%		\end{flalign*}
	\tcbline
		\item \(  \)
%		\begin{flalign*}
	%			
%		\end{flalign*}
	\tcbline
		\item \(  \)
%		\begin{flalign*}
	%			
%		\end{flalign*}
	\end{enumerate}
\end{mybox}

\begin{mybox}{2. Photonen-Ping-Pong}
	\centering \(  \)
	\tcblower
	\begin{enumerate}
		\item \(  \)
%		\begin{flalign*}
	%			
%		\end{flalign*}
	\tcbline
		\item \(  \)
%		\begin{flalign*}
	%		
%		\end{flalign*}
	\tcbline
		\item \(  \)
%		\begin{flalign*}
		%			
%		\end{flalign*}
	\end{enumerate}
\end{mybox}

\begin{mybox}{3. Stehende Wellen}
	\centering \( \vec{E}(\vec{r}, t) = \Re\left( \vec{E}_0\, \mathrm{e}^{\iu\vec{k}\vec{r} - \iu\omega t} \right) \)
	\tcblower
	\begin{enumerate}
		\item \(  \)
		\begin{flalign*}
			\vec{E}_1(\hat{x}, t) &=  \Re\left( \vec{E}_0\, \mathrm{e}^{\iu\vec{k}\hat{x} - \iu\omega t} \right) &&\\
			\vec{E}_2(-\hat{x}, t) &=  \Re\left( \vec{E}_0\, \mathrm{e}^{-\iu\vec{k}\hat{x} - \iu\omega t} \right) &&\\
			\vec{E} &= \vec{E}_1 + \vec{E}_2 = \Re\left( \vec{E}_0\, \mathrm{e}^{-\iu\vec{k}\hat{x} - \iu\omega t} + \mathrm{e}^{-\iu\vec{k}\hat{x} - \iu\omega t} \right) &&\\
		\end{flalign*}
	\tcbline
		\item \(  \)
%		\begin{flalign*}
	%		
%		\end{flalign*}
	\tcbline
		\item \(  \)
%		\begin{flalign*}
		%			
%		\end{flalign*}
	\tcbline
		\item \(  \)
%		\begin{flalign*}
	%			
%		\end{flalign*}´
	\end{enumerate}
\end{mybox}


\end{document}