\documentclass{alex_hü}

\name{Alexander Helbok}
\course{PS Physik}
\hwnumber{4}


\begin{document}
\renewcommand{\labelenumi}{(\alph{enumi})}


\begin{mybox}{1. Der Dipol}
	\centering \(k = \tfrac{1}{4\pi \epsilon_0};\quad x_1 = -\tfrac{d}{2};\quad x_2 = \tfrac{d}{2} \)
	\tcblower
	\begin{enumerate}
		\item \(  \)
		\begin{flalign*}
			V_{ges} &= V_+ + V_- = kq\left(\tfrac{1}{r_+} - \tfrac{1}{r_-} \right) &&
		\end{flalign*}
	\tcbline
		\item \( d \ll r;\quad \vec{p} := q\vec{d} \)
		\begin{flalign*}
			kq\left(\tfrac{1}{r_+} - \tfrac{1}{r_-} \right) &= kq\left(\tfrac{r_- - r_+}{r_+r_-} \right) &&\\
			&\approx kq\tfrac{d\cos(\theta)}{r^2} &&
		\end{flalign*}
		\hfill
		\begin{minipage}[t]{0.3\textwidth}
			\vspace{-2.3cm}
			\boxed{
				\begin{aligned}
					r_- - r_+ &\approx d\cos(\theta) \\
					r_+r_- \pm l &\approx r^2
				\end{aligned}
			}
		\end{minipage}\\
	\tcbline
		\item \( \abs*{\vec{M}} = \abs*{\vec{p} \times \vec{E}} = \vec{p}\vec{E}\sin(\theta);\quad E_{pot} = \int\limits_{\theta}^{\pi/2}\ \abs*{\vec{M}}\ \mathrm{d}\tilde{\theta} \)
		\begin{flalign*}
			0 &= \cos(\tfrac{1}{2}\delta\omega t) &&\\
			t &= \tfrac{2\arccos(0)}{\delta\omega} = \dl{50.29 \unit{s}} &&
		\end{flalign*}
	\end{enumerate}
\end{mybox}


\end{document}