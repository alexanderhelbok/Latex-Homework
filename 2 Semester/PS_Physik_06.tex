\documentclass{alex_hü}

\name{Alexander Helbok}
\course{PS Physik}
\hwnumber{6}


\begin{document}
\renewcommand{\labelenumi}{\alph{enumi})}


\begin{mybox}{1. Falsche Batterie}
	\centering \( R = 100 \unit{\ohm};\quad R_i = 0.1 \unit{\ohm};\quad U_1 = U_2 = 1.5 \unit{V} \)
	\tcblower
	\begin{enumerate}
		\item 
		\begin{flalign*}
			M_1 :&&\\
			M_2 :&&\\
			M_3 :&&\\
			K_1 :&&\\
		\end{flalign*}
		\hfill
		\begin{minipage}[t]{0.85\textwidth}
			\vspace{-3.85cm}
			\boxed{
				\begin{aligned}
					&U_1 - R_i(I_1 + I_2) - U_2 = 0 \\
					&U_1 - R_iI_1 - RI = 0 \\
					&U_2 - R_iI_2 - RI = 0 \\
					&I_1 + I_2 = I 
				\end{aligned}
			} 
		\end{minipage}
		\vspace{-1cm}
	\tcbline
		\item 
		\begin{flalign*}
			I &= \tfrac{2U_1}{R_i + 2R} = \dl{14.99 \unit{mA}} &&\\
			U &= IR = \dl{1.499 \unit{V}} &&\\
			P &= UI = \dl{22.48 \unit{mW}} &&
		\end{flalign*}
	\tcbline
		\item \( U_1 = 1.5 \unit{V};\quad U_2 = 1.2 \unit{V} \)
		\begin{flalign*}
			I &= \tfrac{U_1 + U_2}{R_i + 2R} = \dl{13.49 \unit{mA}} &&\\
			U &= IR = \dl{1.349 \unit{V}} &&\\
			P &= UI = \dl{18.21 \unit{mW}} &&
		\end{flalign*}
	\tcbline
		\item 
		\begin{flalign*}
			I_1 &= \tfrac{U_1 + U_2}{4R + 2R_i} - \tfrac{U_2 - U_1}{2R_i} = \dl{1.51 \unit{A}} &&\\
			I_2 &= I - I_1 = \dl{-1.49 \unit{A}} &&
		\end{flalign*}
	\end{enumerate}
\end{mybox}
\newpage
\begin{mybox}{2. Umladen von Kondensatoren}
	\centering \(  \)
	\tcblower
	\begin{enumerate}
		\item 
		\begin{flalign*}
			q_1 &= \dl{\tfrac{qC_1}{C_1 + C_2}} &&\\
			q_2 &= \dl{\tfrac{qC_2}{C_1 + C_2}} &&\\
			U_1 &= U_1 = \dl{\tfrac{q}{C_1 + C_2}} &&\\
			t &= \tfrac{q_2}{I} = \dl{\tfrac{qC_2}{(C_1 + C_2) I}} &&
		\end{flalign*}
	\tcbline
		\item 
		\begin{flalign*}
			Hallo
		\end{flalign*}
	\tcbline
		\item 
%		\begin{flalign*}
%			
%		\end{flalign*}
	\tcbline
		\item \( C_1 = 1 \unit{nF};\quad C_2 = 200 \unit{pF};\quad R = 150 \unit{\ohm} \)
%		\begin{flalign*}
%			
%		\end{flalign*}
	\tcbline
		\item \( U_1(0) = 100 \unit{V};\quad U_2(0) = 0 \unit{V} \)
%		\begin{flalign*}
%			
%		\end{flalign*}
	\end{enumerate}
\end{mybox}


\begin{mybox}{3. Wheatstone’sche Brucke}
	\centering \( R_1 = 17 \unit{\kilo\ohm};\quad R_2 = 9 \unit{\kilo\ohm};\quad R_3 = 3.6 \unit{\kilo\ohm} \)
	\tcblower
	\begin{enumerate}
		\item 
		\begin{flalign*}
			U_1 &= \tfrac{R_1}{R_1 + R_2} U &&\\
			U_2 &= \tfrac{R_x}{R_x + R_3} U &&\\
			\Delta U &= U_1 - U_2 = \dl{\left( \tfrac{R_1}{R_1 + R_2} - \tfrac{R_x}{R_x + R_3}\right) U} &&
		\end{flalign*}
	\tcbline
		\item 
		\begin{flalign*}
			0 &= U_1 - U_2 &&\\
			0 &= \tfrac{R_1}{R_1 + R_2} - \tfrac{R_x}{R_x + R_3} &&\\
			R_2 &= \dl{\tfrac{R_1R_3}{R_x}} &&
		\end{flalign*}
	\tcbline
		\item 
		\begin{flalign*}
			R_x &= \tfrac{R_1R_3}{R_2} = \dl{6.8 \unit{\kilo\ohm}} &&
		\end{flalign*}
	\end{enumerate}
\end{mybox}

\end{document}