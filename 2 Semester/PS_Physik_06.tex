\documentclass{alex_hü}

\name{Alexander Helbok}
\course{PS Physik}
\hwnumber{6}


\begin{document}
\renewcommand{\labelenumi}{\alph{enumi})}


\begin{mybox}{1. Falsche Batterie}
	\centering \( R = 100 \unit{\ohm};\quad R_i = 0.1 \unit{\ohm};\quad U_1 = U_2 = 1.5 \unit{V} \)
	\tcblower
	\begin{enumerate}
		\item 
		\item 
		\begin{flalign*}
			U &= \dl{1.499 \unit{V}} &&\\
			I &= \tfrac{U}{R} = \dl{14.99 \unit{mA}} &&\\
			P &= UI = \dl{22.485 \unit{mW}} &&
		\end{flalign*}
		\item \( U_1 = 1.5 \unit{V};\quad U_2 = 1.2 \unit{V} \)
%		\begin{flalign*}
%			
%		\end{flalign*}
	\end{enumerate}
\end{mybox}

%\begin{mybox}{2. Umladen von Kondensatoren}
%	\centering \( m = 5 \unit{kg};\quad k = 2 \unit{N/m};\quad a = 0.1 \unit{m};\quad h = 0.4 \unit{m};\quad l = 0.1 \unit{m} \)
%	\tcblower
%	\begin{enumerate}
%		\item 
%	\end{enumerate}
%\end{mybox}
%
%
%\begin{mybox}{3. Wheatstone’sche Brucke}
%	\centering \( m = 5 \unit{kg};\quad k = 2 \unit{N/m};\quad a = 0.1 \unit{m};\quad h = 0.4 \unit{m};\quad l = 0.1 \unit{m} \)
%	\tcblower
%	\begin{enumerate}
%		\item 
%	\end{enumerate}
%\end{mybox}

\end{document}