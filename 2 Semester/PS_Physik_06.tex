\documentclass{alex_hü}

\name{Alexander Helbok}
\course{PS Physik}
\hwnumber{6}


\begin{document}
\renewcommand{\labelenumi}{\alph{enumi})}


\begin{mybox}{1. Falsche Batterie}
	\centering \( R = 100 \unit{\ohm};\quad R_i = 0.1 \unit{\ohm};\quad U_1 = U_2 = 1.5 \unit{V} \)
	\tcblower
	\begin{enumerate}
		\item 
		\begin{flalign*}
			M_1 :&&\\
			M_2 :&&\\
			M_3 :&&\\
			K_1 :&&\\
		\end{flalign*}
		\hfill
		\begin{minipage}[t]{0.85\textwidth}
			\vspace{-3.85cm}
			\boxed{
				\begin{aligned}
					&U_1 - R_i(I_1 + I_2) - U_2 = 0 \\
					&U_1 - R_iI_1 - RI = 0 \\
					&U_2 - R_iI_2 - RI = 0 \\
					&I_1 + I_2 = I 
				\end{aligned}
			} 
		\end{minipage}
		\vspace{-1cm}
	\tcbline
		\item 
		\begin{flalign*}
			I &= \tfrac{2U_1}{R_i + 2R} = \dl{14.99 \unit{mA}} &&\\
			U &= IR = \dl{1.499 \unit{V}} &&\\
			P &= UI = \dl{22.48 \unit{mW}} &&
		\end{flalign*}
	\tcbline
		\item \( U_1 = 1.5 \unit{V};\quad U_2 = 1.2 \unit{V} \)
		\begin{flalign*}
			I &= \tfrac{U_1 + U_2}{R_i + 2R} = \dl{13.49 \unit{mA}} &&\\
			U &= IR = \dl{1.349 \unit{V}} &&\\
			P &= UI = \dl{18.21 \unit{mW}} &&
		\end{flalign*}
	\tcbline
		\item 
		\begin{flalign*}
			I_1 &= \tfrac{U_1 + U_2}{4R + 2R_i} - \tfrac{U_2 - U_1}{2R_i} = \dl{1.51 \unit{A}} &&\\
			I_2 &= I - I_1 = \dl{-1.49 \unit{A}} &&
		\end{flalign*}
	\end{enumerate}
\end{mybox}
\newpage
\begin{mybox}{2. Umladen von Kondensatoren}
	\centering \(  \)
	\tcblower
	\begin{enumerate}
		\item 
		\begin{flalign*}
			q_1 &= \dl{\tfrac{qC_1}{C_1 + C_2}} &&\\
			q_2 &= \dl{\tfrac{qC_2}{C_1 + C_2}} &&\\
			U_1 &= U_2 = \dl{\tfrac{q}{C_1 + C_2}} &&\\
			t &= \tfrac{q_2}{I} = \dl{\tfrac{qC_2}{(C_1 + C_2) I}} &&
		\end{flalign*}
	\tcbline
		\item 
		\begin{flalign*}
			U_1(t) &= \dl{\dv{U_1(t)}{t}C_1R + U_2(t)} &&\\
			U_2(t) &= \dl{\dv{U_2(t)}{t}C_2R + U_1(t)} &&\\
		\end{flalign*}
	\tcbline
		\item 
		\begin{flalign*}
			\Delta U(t) &= U_2(t) - U_1(t) = \dv{\Delta U(t)}{t}R(C_2 - C_1) - \Delta U(t) &&\\
			\dv{\Delta U(t)}{t} &= \tfrac{2}{R(C_2 - C_1)}\Delta U(t) &&\\
			\Delta U(t) &= \dl{c\,\mathrm{e}^{\supfrac[-]{2}{R(C_2 - C_1)}t}} &&
		\end{flalign*}
	\tcbline
		\item \( C_1 = 1 \unit{nF};\quad C_2 = 200 \unit{pF};\quad R = 150 \unit{\ohm} \)
		\begin{flalign*}
			\tfrac{\Delta U(t)}{\Delta U(0)} &= \mathrm{e}^{\supfrac[-]{2}{R(C_2 - C_1)}t} = 0.01 &&\\
			t &= -\tfrac{\ln(0.01) R(C_2 - C_1)}{2} = \dl{2.76 * 10^{-7} \unit{s}} &&
		\end{flalign*}
	\tcbline
		\item \( U_1(0) = 100 \unit{V};\quad U_2(0) = 0 \unit{V};\Delta U(0) = U_0 = 100 \unit{V} \)
		\begin{flalign*}
			\Delta U(t) &= U_0\,\mathrm{e}^{-\tfrac{2}{R(C_2 - C_1)}t} &&\\
			P(t) &= \tfrac{\Delta U(t)^2}{R} = \tfrac{U_0^2}{R}\,\mathrm{e}^{\supfrac[-]{2}{R(C_2 - C_1)}t} &&\\
			P_{max} &= P(0) = \dl{66.6 \unit{W}} &&\\
			E_{tot} &= \uint[0,\infty]{P(t)}{t} = \tfrac{U_0^2}{R}\uint[0,\infty]{\mathrm{e}^{\supfrac[-]{2}{R(C_2 - C_1)}t}}{t} = \dl{2 \unit{\micro\J}} &&
		\end{flalign*}
	\end{enumerate}
\end{mybox}


\begin{mybox}{3. Wheatstone’sche Brucke}
	\centering \( R_1 = 17 \unit{\kilo\ohm};\quad R_2 = 9 \unit{\kilo\ohm};\quad R_3 = 3.6 \unit{\kilo\ohm} \)
	\tcblower
	\begin{enumerate}
		\item 
		\begin{flalign*}
			U_1 &= \tfrac{R_1}{R_1 + R_2} U &&\\
			U_2 &= \tfrac{R_x}{R_x + R_3} U &&\\
			\Delta U &= U_1 - U_2 = \dl{\left( \tfrac{R_1}{R_1 + R_2} - \tfrac{R_x}{R_x + R_3}\right) U} &&
		\end{flalign*}
	\tcbline
		\item 
		\begin{flalign*}
			0 &= U_1 - U_2 &&\\
			0 &= \tfrac{R_1}{R_1 + R_2} - \tfrac{R_x}{R_x + R_3} &&\\
			R_2 &= \dl{\tfrac{R_1R_3}{R_x}} &&
		\end{flalign*}
	\tcbline
		\item 
		\begin{flalign*}
			R_x &= \tfrac{R_1R_3}{R_2} = \dl{6.8 \unit{\kilo\ohm}} &&
		\end{flalign*}
	\end{enumerate}
\end{mybox}

\end{document}