\documentclass{alex_hü}

\name{Alexander Helbok}
\course{PS Physik}
\hwnumber{8}


\begin{document}
\renewcommand{\labelenumi}{\alph{enumi})}

\begin{mybox}{1. Magnetfeld eines Koaxialkabels}
	\centering \( \)
	\tcblower
	\underline{For  \(r \leq R_1\):}
	\begin{flalign*}
		I_{in} &= I\tfrac{r^2}{R_1^2} &&\\
		I_{in}\mu_0 &= \oint \vec{B}(\vec{r})\ \mathrm{d}\vec{s} = B(r) \oint \mathrm{d}s = B(r)2r\pi &&\\
		B(r) &= \tfrac{I\mu_0}{2\pi R_1^2}r &&
	\end{flalign*}
	\underline{For  \(R_1 \leq r \leq R_2\):}
	\begin{flalign*}
		I_{in} &= I &&\\
		I_{in}\mu_0 &= \oint \vec{B}(\vec{r})\ \mathrm{d}\vec{s} = B(r) \oint \mathrm{d}s = B(r)2r\pi &&\\
		B(r) &= \tfrac{I\mu_0}{2\pi}\tfrac{1}{r} &&
	\end{flalign*}
	\underline{For  \(R_2 \leq r \leq R_3\):}
	\begin{flalign*}
		I_{in} &= I\left( 1 - \tfrac{r^2 - R_2^2}{R_3^2 - R_2^2} \right)  &&\\
		I_{in}\mu_0 &= \oint \vec{B}(\vec{r})\ \mathrm{d}\vec{s} = B(r) \oint \mathrm{d}s = B(r)2r\pi &&\\
		B(r) &= \tfrac{I\mu_0}{2\pi}\left( \tfrac{1}{r} - \tfrac{r^2-R_2^2}{r(R_3^2 - R_2^2)} \right) &&
	\end{flalign*}
	\underline{For  \(R_3 \leq r\):}
	\begin{flalign*}
		I_{in} &= 0 &&\\
		B(r) &= 0 &&
	\end{flalign*}
	\boxed{
		\begin{aligned}
			B(r) = \begin{cases}
				\tfrac{I\mu_0}{2\pi R_1^2}r& \quad $for $\ 0 < r \le R_1, \\
				\tfrac{I\mu_0}{2\pi}\tfrac{1}{r}& \quad $for $\ R_1 \leq r \leq R_2, \\
				\tfrac{I\mu_0}{2\pi}\left( \tfrac{1}{r} - \tfrac{r^2-R_2^2}{r(R_3^2 - R_2^2)} \right)& \quad $for $\ R_2 \leq r \leq R_3, \\
				0& \quad $for $\ R_3 \leq r. 
			\end{cases} 
		\end{aligned}
	}\\[3cm]
	\begin{tikzpicture}
		\begin{axis}[
			width=250pt,
			height=150pt,
			axis lines=center,
			%y axis line style={thick},
			tick align=outside,
			xmin=0,xmax=3.5,ymin=0,ymax=1.2,
			xlabel style={below},
			xtick = {1,2,3}, ytick = {1},
			xticklabels={\( R_1 \), \( R_2 \), \( R_3 \)},
			yticklabels={\( B_{max} \)},
			xlabel=$r$,
			ylabel=$B$,
			grid=major,
			grid style={thin,densely dotted,black!20},
			%legend columns=2,
			legend style={at={(axis description cs:1,0.35)},anchor=east}]
			\addplot [-, thick,  blue, domain = 0:1, smooth] {x};
			\addplot [-, thick,  blue, domain = 1:2, smooth] {1/x};
			\addplot [-, thick,  blue, domain = 2:3, smooth] {1/x-(x^2-4)/(5*x)};
			\draw[thick, blue] (3, 0) -- (3.4, 0);
		\end{axis}
	\end{tikzpicture}\\
\end{mybox}

\begin{mybox}{2. Anwendung des Gesetzes von Biot-Savart – „Haarnadel“}
	\centering \( \)
	\tcblower
	\begin{flalign*}
		\dd{\vec{B}_1} &= \tfrac{\mu_0I}{4\pi}\tfrac{\dd{\vec{l}}\times \vec{r}}{r^3} = \tfrac{\mu_0I}{4\pi}\tfrac{\dd{l}\sin(\theta)}{r^2} = \tfrac{\mu_0I}{4\pi}\tfrac{\dd{\varphi}R^2}{r^3}  &&\\
		B_1 &= \tfrac{\mu_0I}{4\pi}\tfrac{R^2}{r^3}\uint[0,\pi]{1}{\varphi} = \tfrac{\mu_0I}{4}\tfrac{R^2}{r^3} &&\\
		\dd{\vec{B}_2} &= \tfrac{\mu_0I}{4\pi}\tfrac{\dd{\vec{l}}\times \vec{r}}{r^3} = \tfrac{\mu_0I}{4\pi}\tfrac{\dd{l}\sin(\theta)}{r^2} = \tfrac{\mu_0I}{4\pi}\tfrac{\dd{l}R}{r^3}  &&\\
	\end{flalign*}
\end{mybox}

\begin{mybox}{3. Drehmoment auf rechteckige Leiterschleife}
	\centering \( \)
	\tcblower
	\begin{enumerate}
		\item \(  \)
%		\begin{flalign*}
%				
%		\end{flalign}
	\tcbline
		\item 
	\end{enumerate}
\end{mybox}


\end{document}