\documentclass{alex_hü}

\name{Alexander Helbok}
\course{PS Physik}
\hwnumber{8}


\begin{document}
\renewcommand{\labelenumi}{\alph{enumi})}

\begin{mybox}{1. Magnetfeld eines Koaxialkabels}
	\centering \( \)
	\tcblower
	\underline{For  \(r \leq R_1\):}
	\begin{flalign*}
		I_{in} &= I\tfrac{r^2}{R_1^2} &&\\
		I_{in}\mu_0 &= \oint \vec{B}(\vec{r})\ \mathrm{d}\vec{s} = B(r) \oint \mathrm{d}s = B(r)2r\pi &&\\
		B(r) &= \tfrac{I\mu_0}{2\pi R_1^2}r &&
	\end{flalign*}
	\underline{For  \(R_1 \leq r \leq R_2\):}
	\begin{flalign*}
		I_{in} &= I &&\\
		I_{in}\mu_0 &= \oint \vec{B}(\vec{r})\ \mathrm{d}\vec{s} = B(r) \oint \mathrm{d}s = B(r)2r\pi &&\\
		B(r) &= \tfrac{I\mu_0}{2\pi}\tfrac{1}{r} &&
	\end{flalign*}
	\underline{For  \(R_2 \leq r \leq R_3\):}
	\begin{flalign*}
		I_{in} &= I\left( 1 - \tfrac{r^2 - R_2^2}{R_3^2 - R_2^2} \right)  &&\\
		I_{in}\mu_0 &= \oint \vec{B}(\vec{r})\ \mathrm{d}\vec{s} = B(r) \oint \mathrm{d}s = B(r)2r\pi &&\\
		B(r) &= \tfrac{I\mu_0}{2\pi}\left( \tfrac{1}{r} - \tfrac{r^2-R_2^2}{r(R_3^2 - R_2^2)} \right) &&
	\end{flalign*}
	\underline{For  \(R_3 \leq r\):}
	\begin{flalign*}
		I_{in} &= 0 &&\\
		B(r) &= 0 &&
	\end{flalign*}
	\boxed{
		\begin{aligned}
			B(r) = \begin{cases}
				\tfrac{I\mu_0}{2\pi R_1^2}r& \quad $for $\ 0 < r \le R_1, \\
				\tfrac{I\mu_0}{2\pi}\tfrac{1}{r}& \quad $for $\ R_1 \leq r \leq R_2, \\
				\tfrac{I\mu_0}{2\pi}\left( \tfrac{1}{r} - \tfrac{r^2-R_2^2}{r(R_3^2 - R_2^2)} \right)& \quad $for $\ R_2 \leq r \leq R_3, \\
				0& \quad $for $\ R_3 \leq r. 
			\end{cases} 
		\end{aligned}
	}\\[3cm]
	\begin{tikzpicture}
		\begin{axis}[
			width=250pt,
			height=150pt,
			axis lines=center,
			%y axis line style={thick},
			tick align=outside,
			xmin=0,xmax=4.5,ymin=0,ymax=1.2,
			xlabel style={below},
			xtick = {1,3,4}, ytick = {1},
			xticklabels={\( R_1 \), \( R_2 \), \( R_3 \)},
			yticklabels={\( B_{max} \)},
			xlabel=$r$,
			ylabel=$B$,
			grid=major,
			grid style={thin,densely dotted,black!20},
			%legend columns=2,
			legend style={at={(axis description cs:1,0.35)},anchor=east}]
			\addplot [-, thick,  blue, domain = 0:1, smooth] {x};
			\addplot [-, thick,  blue, domain = 1:3, smooth] {1/x};
			\addplot [-, thick,  blue, domain = 3:4, smooth] {1/x-(x^2-9)/(7*x)};
			\draw[thick, blue] (4, 0) -- (4.4, 0);
		\end{axis}
	\end{tikzpicture}\\
\end{mybox}

\begin{mybox}{2. Anwendung des Gesetzes von Biot-Savart – „Haarnadel“}
	\centering \( R = \sqrt{x^2+y^2};\quad r = \sqrt{x^2 + y^2 + z^2} = \sqrt{R^2 + z^2} \)
	\tcblower
	\begin{flalign*}
		\dd{B_1} &= \tfrac{\mu_0I}{4\pi}\tfrac{\abs*{\dd{\vec{l}}\times \vec{r}}}{r^3} = \tfrac{\mu_0I}{4\pi}\tfrac{\dd{l}}{r^2} = \tfrac{\mu_0I}{4\pi}\tfrac{\dd{\varphi}R}{r^2}  &&\\
		B_1 &= \tfrac{\mu_0I}{4\pi}\tfrac{R}{r^2}\uint[0,\pi]{1}{\varphi} = \tfrac{\mu_0I}{4}\tfrac{R}{r^2} &&\\
		B_{x, 1} &= B_1\sin(\theta) = \tfrac{\mu_0I}{4}\tfrac{R}{r^2} \sin(\theta) = \tfrac{\mu_0I}{4}\tfrac{R^2}{r^3} &&\\
		B_{z, 1} &= B_1\cos(\theta) = \tfrac{\mu_0I}{4}\tfrac{R}{r^2} \cos(\theta) = \tfrac{\mu_0I}{4}\tfrac{Rz}{r^3} &&\\[2em]
		\dd{B_2} &= \tfrac{\mu_0I}{4\pi}\tfrac{\abs*{\dd{\vec{l}}\times \vec{r}}}{r^3} = \tfrac{\mu_0I}{4\pi}\tfrac{\dd{l} \sin(\theta)}{r^2} = \tfrac{\mu_0I}{4\pi}\tfrac{\dd{l} \sqrt{y^2 + z^2}}{r^3} &&\\
		B_2 &= \tfrac{\mu_0I}{2\pi} \uint[-\infty,0]{\tfrac{\sqrt{y^2 + z^2}}{r^3}}{x} = \tfrac{\mu_0I}{2\pi}\tfrac{1}{\sqrt{y^2 + z^2}} &&\\
		B_{z, 2} &= B_2\cos(\theta) = \tfrac{\mu_0I}{2\pi}\tfrac{1}{\sqrt{y^2 + z^2}} \cos(\theta) = \tfrac{\mu_0I}{2\pi}\tfrac{z}{\sqrt{y^2 + z^2}\sqrt{x^2+y^2+z^2}} &&\\
		\vec{B}(z) &= \dl{\tfrac{\mu_0I}{4}\vector{R^2 / \left( R^2+z^2 \right)^{3/2}\\ 0\\ \tfrac{Rz}{\left( R^2 + z^2 \right)^{3/2}} + \tfrac{2z}{\pi\sqrt{y^2 + z^2}\sqrt{R^2+z^2}}}} &&
	\end{flalign*}
\end{mybox}

\begin{mybox}{3. Drehmoment auf rechteckige Leiterschleife}
	\centering \(  \)
	\tcblower
	\begin{enumerate}
		\item \( \varphi = \arctan(\tfrac{d}{2x});\quad r = \sqrt{\tfrac{4x^2 + d^2}{4}}  \)
		\begin{flalign*}
			\vec{B}(r) &= \tfrac{\mu_0I_1}{2r\pi}\vector{-\sin(\varphi)\\ \cos(\varphi)\\ 0} &&\\
			\vec{F}_1 &= I_2\vec{h}\times\vec{B} = \tfrac{\mu_0I_1I_2h}{2\pi} \vector{-\cos(\varphi)\\ -\sin(\varphi)\\ 0} &&\\
			\vec{F}_2 &= I_2\vec{h}\times\vec{B} = \tfrac{\mu_0I_1I_2h}{2r\pi} \vector{\cos(\varphi)\\ \sin(\varphi)\\ 0} &&\\
			\vec{M}_1 &= \vec{r}_1 \times \vec{F}_1 = \vector{0\\ d/2\\ 0} \times \tfrac{\mu_0I_1I_2h}{2r\pi} \vector{-\cos(\varphi)\\ -\sin(\varphi)\\ 0} = \tfrac{\mu_0I_1I_2hd}{4r\pi}\vector{0\\ 0\\ \cos(\varphi)} &&\\
			\vec{M}_2 &= \vec{r}_2 \times \vec{F}_2 = \vector{0\\ -d/2\\ 0} \times \tfrac{\mu_0I_1I_2h}{2r\pi} \vector{\cos(\varphi)\\ \sin(\varphi)\\ 0} = \tfrac{\mu_0I_1I_2hd}{4r\pi}\vector{0\\ 0\\ \cos(\varphi)} &&\\
			\vec{M} &= \vec{M}_1 + \vec{M}_2 = \dl{\tfrac{\mu_0I_1I_2hd}{2r\pi}\vector{0\\ 0\\ \cos(\varphi)}} &&\\
		\end{flalign*}
	\tcbline
		\item \( \mu_0 = 4\pi * 10^{-7} \unit{H/m};\quad I_1 = 16 \unit{A};\quad I_2 = 1 \unit{A};\quad d = h = x = 0.1 \unit{m}\) 
	\end{enumerate}
	\begin{flalign*}
		\vec{M} &= \tfrac{\mu_0I_1I_2hd}{2r\pi}\cos(\varphi)\hat{z} = \dl{2.56 * 10^{-7} \unit{Nm}} &&
	\end{flalign*}
	\begin{flalign*}
		\text{Bolts on Bike}&: 5-30 \unit{Nm} &&\\
		\text{Basic Electromotor}&: 0.3-0.5 \unit{Nm} &&\\
		\text{Car wheels}&: 110-120 \unit{Nm} &&
	\end{flalign*}
\end{mybox}


\end{document}