\documentclass{alex_hü}

\name{Alexander Helbok}
\course{PS Physik}
\hwnumber{8}


\begin{document}
\renewcommand{\labelenumi}{\alph{enumi})}

\begin{mybox}{1. Magnetfeld eines Koaxialkabels}
	\centering \( \)
	\tcblower
	\begin{flalign*}
		\dd{\vec{B}_1} &= \tfrac{\mu_0I}{4\pi}\tfrac{\dd{\vec{l}}\times \vec{r}}{r^3} = \tfrac{\mu_0I}{4\pi}\tfrac{\dd{l}\sin(\theta)}{r^2} = \tfrac{\mu_0I}{4\pi}\tfrac{\dd{\varphi}R^2}{r^3}  &&\\
		B_1 &= \tfrac{\mu_0I}{4\pi}\tfrac{R^2}{r^3}\uint[0,\pi]{1}{\varphi} = \tfrac{\mu_0I}{4}\tfrac{R^2}{r^3} &&\\
		\dd{\vec{B}_2} &= \tfrac{\mu_0I}{4\pi}\tfrac{\dd{\vec{l}}\times \vec{r}}{r^3} = \tfrac{\mu_0I}{4\pi}\tfrac{\dd{l}\sin(\theta)}{r^2} = \tfrac{\mu_0I}{4\pi}\tfrac{\dd{l}R}{r^3}  &&\\
	\end{flalign*}
\end{mybox}

\begin{mybox}{2. Anwendung des Gesetzes von Biot-Savart – „Haarnadel“}
	\centering \( \)
	\tcblower
	\begin{flalign*}
		\dd{\vec{B}_1} &= \tfrac{\mu_0I}{4\pi}\tfrac{\dd{\vec{l}}\times \vec{r}}{r^3} = \tfrac{\mu_0I}{4\pi}\tfrac{\dd{l}\sin(\theta)}{r^2} = \tfrac{\mu_0I}{4\pi}\tfrac{\dd{\varphi}R^2}{r^3}  &&\\
		B_1 &= \tfrac{\mu_0I}{4\pi}\tfrac{R^2}{r^3}\uint[0,\pi]{1}{\varphi} = \tfrac{\mu_0I}{4}\tfrac{R^2}{r^3} &&\\
		\dd{\vec{B}_2} &= \tfrac{\mu_0I}{4\pi}\tfrac{\dd{\vec{l}}\times \vec{r}}{r^3} = \tfrac{\mu_0I}{4\pi}\tfrac{\dd{l}\sin(\theta)}{r^2} = \tfrac{\mu_0I}{4\pi}\tfrac{\dd{l}R}{r^3}  &&\\
	\end{flalign*}
\end{mybox}

\begin{mybox}{3. Drehmoment auf rechteckige Leiterschleife}
	\centering \( \)
	\tcblower
	\begin{enumerate}
		\item \(  \)
%		\begin{flalign*}
%				
%		\end{flalign}
	\tcbline
		\item 
	\end{enumerate}
\end{mybox}


\end{document}