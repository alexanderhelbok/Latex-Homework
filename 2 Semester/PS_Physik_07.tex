\documentclass{alex_hü}

\name{Alexander Helbok}
\course{PS Physik}
\hwnumber{7}


\begin{document}
\renewcommand{\labelenumi}{(\alph{enumi})}

\begin{mybox}{1. Zyklotron}
	\centering \( \)
	\tcblower
	\begin{enumerate}
		\item \( \)
		\begin{flalign*}
			F &= qvB = q\omega rB = ma = m\tfrac{v^2}{r} = m\omega^2r &&\\
			\omega &= \dl{\tfrac{qB}{m}} &&\\
			f &= \dl{\tfrac{qB}{2\pi m}} &&
		\end{flalign*}
	\tcbline
		\item 
		\begin{flalign*}
			v &= \tfrac{qBr}{m} &&\\
			K &= \tfrac{mv^2}{2} = \dl{\tfrac{(qBr)^2}{2m}} &&
 		\end{flalign*}
	\tcbline
		\item 
		\begin{flalign*}
			r &= \tfrac{mv}{qB} &&\\
			v_n &= v_{n-1} + \sqrt{\tfrac{8W_0}{m}} = n\sqrt{\tfrac{8W_0}{m}}&&\\
			r_n &= r_{n-1} + \tfrac{\sqrt{8W_0m}}{qB}  = \dl{n\tfrac{\sqrt{8m}}{qB} \sqrt{W_0}} &&
		\end{flalign*}
	\tcbline
		\item The Cyclotron is used to accelerate charged Particles to up to \( 50 \unit{MeV} \). At these energies relativistic effects come into play and disrupt the cyclotron motion. To reach higher energies a similar machine with an adjustable B-Field (the Synchrotron) can be used.
	\end{enumerate}
\end{mybox}

\begin{mybox}{2. Geschwindigkeitsfilter}
	\centering \( \vec{E} = \vector{0\\ 0\\ E};\quad \vec{B} = \vector{0\\ B\\ 0} \)
	\tcblower
	\begin{enumerate}
		\item \(  \)
		\begin{flalign*}
		 	\vec{a} &= \frac{q}{m}(\vec{E} + \vec{v} \times \vec{B}) = \frac{q}{m}\vector{-v_zB\\ 0\\ E + v_xB} &&\\
		 	\dv{v_x}{t} &= -\tfrac{qB}{m}v_z(t) &&\\
		 	\dv{v_z}{t} &= \tfrac{qE}{m} + \tfrac{qB}{m}v_x(t) &&\\
		\end{flalign*}
	\tcbline
		\item \( v_x(0) = v_0;\quad v_z(0) = 0 \)
		\begin{flalign*}
			v_x(t) &= -\tfrac{E}{B} + \left(\tfrac{E}{B} + v_0\right)\cos(\tfrac{qB}{m}t) &&\\
			v_z(t) &= \left(\tfrac{E}{B} + v_0\right)\sin(\tfrac{qB}{m}t) &&\\[2em]
			\vec{v}(t) &= \dl{\vector{-\tfrac{E}{B} + \left(\tfrac{E}{B} + v_0\right)\cos(\tfrac{qB}{m}t)\\ 0\\ \left(\tfrac{E}{B} + v_0\right)\sin(\tfrac{qB}{m}t)}} &&
		\end{flalign*}
	\tcbline
		\item \( \vec{r}(0) = \vec{0} \)
		\begin{flalign*}
			\vec{r}(t) &= \uint{\vec{v}(t)}{t} = \dl{\vector{-\tfrac{E}{B}t + \left(\tfrac{E}{B} + v_0\right)\tfrac{m}{qB}\sin(\tfrac{qB}{m}t)\\ 0\\ -\left(\tfrac{E}{B} + v_0\right)\tfrac{m}{qB}\cos(\tfrac{qB}{m}t)}} &&
		\end{flalign*}
	\tcbline
		\item 
		\begin{flalign*}
			v_z(t) &= \left(\tfrac{E}{B} + v_0\right)\sin(\tfrac{qB}{m}t) = 0 \quad \Rightarrow \quad \tfrac{E}{B} = -v_0 &&
		\end{flalign*}
		\begin{flalign*}
			v_x(t) &= -\tfrac{E}{B} + \left(\tfrac{E}{B} + v_0\right)\cos(\tfrac{qB}{m}t) = -(-v_0) + \left(-v_0 + v_0\right)\cos(\tfrac{qB}{m}t) = v_0 &&\\
		\end{flalign*}
		\begin{flalign*}
			&\Rightarrow \dl{\tfrac{E}{B} = -v_0;\quad m,q \text{ choose freely}} &&
		\end{flalign*}
	\end{enumerate}
\end{mybox}

\begin{mybox}{3. Der magnetische Spiegel}
	\centering \( \)
	\tcblower
	\begin{enumerate}
		\item \( \)
		\begin{flalign*}
			\vec{v} &= \vec{v}_\perp + \vec{v}_\parallel = \vector{v_x\\ v_y\\ 0}  + \vector{0\\ 0\\ v_z} &&\\
			F_L &= qv_\perp B = m\tfrac{v_\perp^2}{r} = F_Z &&\\
			r_L &= \dl{\tfrac{mv_\perp}{qB}} &&
		\end{flalign*}
	\tcbline
		\item \( T = \tfrac{2\pi m}{qB};\quad r = \tfrac{mv_\perp}{qB} \)
		\begin{flalign*}
			E &= E_z + E_\perp = \tfrac{{mv_\parallel}^2}{2} + \tfrac{mv_\perp}{2} &&\\
			\mu &= I\vec{A} = \tfrac{q}{T}r^2\pi = \tfrac{q}{\tfrac{2\pi m}{qB}}\left( \tfrac{mv_\perp}{qB} \right)^2\pi\vec{n_a} = \tfrac{q^2B}{2\pi m}\tfrac{m^2{v_\perp}^2}{q^2B^2}\pi = \tfrac{m{v_\perp}^2}{2}\tfrac{1}{B} = &&\\
			&= \dl{\tfrac{E_\perp}{B}} &&
		\end{flalign*}
	\tcbline
		\item \( E = E_\perp + E_\parallel = \text{const.};\quad \mu = \tfrac{E_\perp}{B} = \text{const.} \) 
		\begin{flalign*}
			\tfrac{m{v_\parallel}^2}{2} &= E - \tfrac{mv{_\perp}^2}{2} = E - \mu B &&\\
			v_z(B) &= \dl{\sqrt{\tfrac{2E}{m} - \tfrac{2\mu B}{m}}} &&
		\end{flalign*}
	\tcbline
		\item When \( v_z = 0 \) the particle has come to a halt and will reverse its z-direction (due to the non-homogeneity of the B-Field). Earths Magnetic Field acts like a magnetic bottle, trapping particles emitted by the sun in the van Allen belts.
	\end{enumerate}
\end{mybox}


\end{document}