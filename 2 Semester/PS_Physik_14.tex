\documentclass{alex_hü}

\name{Alexander Helbok}
\course{PS Physik}
\hwnumber{14}


\begin{document}
\renewcommand{\labelenumi}{(\alph{enumi})}


\begin{mybox}{1. Abbildungseigenschaften eines Linsensystems}
	\centering \(  \)
	\tcblower
	\begin{enumerate}
		\item \(  \)
%		\begin{flalign*}
%			
%		\end{flalign*}
	\tcbline
		\item \(  \)
%		\begin{flalign*}
%		
%		\end{flalign*}
	\tcbline
		\item \(  \)
%		\begin{flalign*}
%			
%		\end{flalign*}
	\end{enumerate}
\end{mybox}

\begin{mybox}{2. Laserlicht und chromatische Abberation}
	\centering \(  \)
	\tcblower
	\begin{enumerate}
		\item \(  \)
%		\begin{flalign*}
	%			
%		\end{flalign*}
	\tcbline
		\item \(  \)
%		\begin{flalign*}
	%		
%		\end{flalign*}
	\tcbline
		\item \(  \)
%		\begin{flalign*}
		%			
%		\end{flalign*}
	\end{enumerate}
\end{mybox}

\begin{mybox}{3. Sagnac-Interferometer}
	\centering \( \lambda_1 = 397 \unit{nm};\quad \lambda_2 = 866 \unit{nm};\quad R_1 = 0.1 \unit{m};\quad R_2 = 0.2 \unit{m} \)
	\tcblower
	\begin{enumerate}
		\item \(  \)
%		\begin{flalign*}
		%			
%		\end{flalign*}
	\tcbline
		\item \(  \)
%		\begin{flalign*}
	%		
%		\end{flalign*}
	\tcbline
		\item \(  \)
%		\begin{flalign*}
		%			
%		\end{flalign*}
	\end{enumerate}
\end{mybox}


\end{document}