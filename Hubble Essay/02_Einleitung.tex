% !TeX root = Bericht.tex
% !TeX spellcheck = en_US
\section{introduction}
In this essay I would like to go through the original paper published by Edwin Hubble in 1929 \autocite{Hubble1929}, in which he measures the expansion rate of the universe. This paper revolutionized the field of astrophysics, shifting the then predominant paradigm of a static universe, supported by many renowned physicists like Albert Einstein (admitting his support to be his greatest blunder). In his paper he uses Doppler shift measurements and the period-luminosity relation of Cepheids as well as statistical distribution of star brightness to determine the velocity and distance of various stars. Combining the two measurements Hubble came the astonishing conclusion that a \blockcquote{Hubble1929}{roughly linear relation between velocities and distances} can be established. 

Even though Edwin Hubble was criticized for poor scientific behavior, his findings could not be ignored and were verified by further experiments. Up until today experiments are carried through trying to determine the Hubble parameter at different points in time as this leads valuable insights into the driving mechanisms behind cosmic expansion and the composition of the universe. 