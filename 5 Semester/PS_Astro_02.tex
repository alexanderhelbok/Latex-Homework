% !TeX spellcheck = de_DE
\documentclass{alex_hü}

\name{Alexander Helbok}
\course{PS Astrophysics}
\hwnumber{2}


\begin{document}
\renewcommand{\labelenumi}{\alph{enumi})}


\begin{mybox}{Die Friedmann–Gleichung}
	\centering \( H^2(t) = \left( \tfrac{\dot{a}}{a} \right)^2 = H_0^2\left( \tfrac{\Omega_r}{a^4} + \tfrac{\Omega_m}{a^3} + \Omega_\Lambda - \tfrac{\Omega_0 - 1}{a^2} \right) \)
	\tcblower
	\begin{enumerate}
		\item 
		\begin{itemize}
			\item \( \Omega_0 \dots \) Density 
			\item \( \Omega_r \dots \) radiation density
			\item \( \Omega_m \dots \) matter density (Dark + Baryonic)
			\item \( \Omega_\Lambda \dots \) cosmological constant (vacuum density)
		\end{itemize}
	\tcbline
		\item \(  \)
		\begin{flalign*}
			H^2(t) &= \left( \tfrac{\dot{a}}{a} \right)^2
				= \left( \tfrac{\dv{a}{t}}{a} \right)^2 
				\quad\Rightarrow\quad a^2H^2(t) = \left(\dv{a}{t}\right)^2
				\quad\Rightarrow\quad \dd{a} = \tfrac{\dd{t}}{a H(t)} &&\\
			\uint[]{\ }{t} &= \uint[]{\tfrac{1}{a H(a)}}{a}
		\end{flalign*}
	\tcbline
		\item \(  \)
%		\begin{flalign*}
%			
%		\end{flalign*}
	\end{enumerate}
\end{mybox}

\end{document}