\documentclass{alex_hü}

\name{Alexander Helbok}
\course{PS Physik}
\hwnumber{12}


\begin{document}
\renewcommand{\labelenumi}{\alph{enumi})}

\begin{mybox}{207. Kupferblock in Styropor}
	\centering \( \rho = 8930 \unit{kg/m^3};\quad c = 385 \unit{J/kg.K} \)
	\tcblower
	\begin{enumerate}
		\item \(  \)
%		\begin{flalign*}
%			
%		\end{flalign*}
	\Sepline
	\end{enumerate}
\end{mybox}

\begin{mybox}{209. Limonade mit Eis}
	\centering \( m_1 = 0.24 \unit{kg};\quad T_1 = 306.15 \unit{K};\quad m_2 = 0.025 \unit{kg};\quad T_2 = 273.15 \unit{K} \)
	\tcblower
	\begin{enumerate}
		\item \( Q = cm\Delta T;\quad \Delta T = T_{Ende} - T_{Start} \)
		\begin{flalign*}
			Q_1 &= cm_1(T - T_1) &&\\
			Q_2 &= cm_2(T - T_2) &&\\
			Q_1 + 2Q_2 &= 0 &&\\
			\Rightarrow T &= \tfrac{m_1T_1 + 2m_2T_2}{m_1 + 2m_2} = \dl{300.46 \unit{K}} \hskip3cm (=27.31 \unit{\celsius})&&
		\end{flalign*}
	\Sepline
		\item \(  \)
		\begin{flalign*}
			Q_1 + 6Q_2 &= 0 &&\\
			\Rightarrow T &= \tfrac{m_1T_1 + 6m_2T_2}{m_1 + 6m_2} = \dl{293.46 \unit{K}} \hskip3cm (=20.31 \unit{\celsius})&&
		\end{flalign*}
	\end{enumerate}
\end{mybox}



\end{document}