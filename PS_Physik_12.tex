\documentclass{alex_hü}

\name{Alexander Helbok}
\course{PS Physik}
\hwnumber{12}


\begin{document}
\renewcommand{\labelenumi}{\alph{enumi}\hskip0pt)}

\begin{mybox}{207. Kupferblock in Styropor}
	\centering \( a = 0.1 \unit{m};\quad m = a^3\rho;\quad l = \tfrac{1}{10}a;\quad T_0 = 293.15;\quad T_1 = 333.15 \unit{K} \)\\
	\( \quad \rho = 8930 \unit{kg/m^3};\quad c = 385 \unit{J/kg.K};\quad \lambda = 0.027 \unit{W/m.K}\)
	\begin{empheq}[box=\fbox]{align*}
		\text{Def: }&\text{System wird erwärmt = positive Wärme = positive Arbeit}\\
		&\text{System kühlt ab = negative Wärme = negative Arbeit}
	\end{empheq}
	\tcblower
	\begin{enumerate}
		\item Die Wärmeleitfähigkeit der Luft sagt in diesem Beispiel nicht aus, wie gut diese Isoliert, da Luft bei Erwärmung zirkuliert und somit den "Wärmetransport"\hskip2pt fördert.
	\tcbline
		\item \( G = \lambda \tfrac{A}{l} \)
		\begin{flalign*}
			G &= - 60\lambda a = \dl{- 0.162 \unit{W/K}} &&
		\end{flalign*}
	\tcbline
		\item \( \dot{Q} = \dv{Q}{t} = G\Delta T \)
		\begin{flalign*}
			\dot{Q} &= G(T_0 - T_1) = \dl{6.48 \unit{W}} &&
		\end{flalign*}
	\tcbline
		\item \( T_2 = 332.15 \unit{K};\quad Q = mc\Delta T \)
		\begin{flalign*}
			Q_1 &= a^3\rho c(T_2 - T_1) &&\\
			t_1 &= \tfrac{Q_1}{\dot{Q}} = \dl{530.6 \unit{s}} &&
		\end{flalign*}
	\tcbline*
		\item \( \dv{T}{t} = r(T(t) - T_0);\quad r = -\tfrac{G}{cm};\quad T(0) = T_1 = 333.15 \unit{K} \)
		\begin{flalign*}
			\dv{T}{t} &=  r(T(t) - T_0) &&\\
			\int{\tfrac{1}{T(t) - T_0} \dd{T}} &= \int{r \dd{t}} &&\\
			\ln(T(t) - T_0) + c_1 &= rt + c_2 &c_3&:= c_2 - c_1 &&\\
			T(t) - T_0 &= \mathrm{e}^{rt + c_3}	&C&:= \mathrm{e}^{c_3} &&\\
			\Aboxed{T(t) &= C\mathrm{e}^{rt} + T_0} &&\\[3ex]
		\end{flalign*}
		\begin{flalign*}
			T(0) &= T_1 &&\\
			C\mathrm{e}^{rt} + T_0 &= T_1 &C &= T_1 - T_0&&\\
			T(t) &= (T_1 - T_0)\mathrm{e}^{rt} + T_0 &&\\[1ex]
			\Rightarrow\ \Aboxed{T(t) &= 40\mathrm{e}^{-4.7*10^{-5}t} + 20} &&
		\end{flalign*}
	\tcbline
		\item \(  \)
		\begin{flalign*}
			21 &= T(t) &&\\
			21 &= (T_1 - T_0)\mathrm{e}^{rt} + T_0 &&\\
			t &= \tfrac{\ln(\tfrac{21 - T_0}{T_1 - T_0}) }{r} = \dl{78287 \unit{s}}
		\end{flalign*}
	\end{enumerate}
\end{mybox}


\begin{mybox}{209. Limonade mit Eis}
	\centering \( m_1 = 0.24 \unit{kg};\quad T_1 = 306.15 \unit{K};\quad m_2 = 0.025 \unit{kg};\quad T_2 = 273.15 \unit{K} \)\\
	\centering \( L = 3.33 * 10^{5} \unit{J/kg}:\quad c = 4190 \unit{J/kg.K} \)
	\tcblower
	\begin{enumerate}
		\item \( Q = cm\Delta T;\quad Q = Lm;\quad \Delta T = T_{Ende} - T_{Start} \)
		\begin{flalign*}
			Q_s &= 2Lm_2 &(\text{Schmelzwärme}) &&\\
			T_s &= -\tfrac{Q_s}{cm_1} + T_1 &(\text{Temp nach Schmelze}) &&\\[2ex]
			Q_1 &= cm_1(T - T_s) &&\\
			Q_2 &= cm_2(T - T_2) &&\\
			Q_1 + 2Q_2 &= 0 &&\\
			\Rightarrow T &= \tfrac{m_1T_s + 2m_2T_2}{m_1 + 2m_2} = \dl{286.76 \unit{K}}  &(=13.61 \unit{\celsius})&&
		\end{flalign*}
	\tcbline*
		\item \(  \)
		\begin{flalign*}
			Q_s &= 6Lm_2 = 4.995 * 10^4 \unit{J} &(\text{Schmelzwärme}) &&\\
			Q &= cm_1(T_1 - T_2) = - 3.079 * 10^4 \unit{J} &(\text{bis zum Gefrieren}) &&\\
		\end{flalign*}
	\( \Rightarrow\ \)im Wasser ist nicht genügend Energie gespeichert, um die Eiswürfel zu schmelzen; es bleibt ein Wasser-Eis Gemisch bei \( \dl{273.15 \unit{K}}\ (= 0 \unit{\celsius}) \) über
	\end{enumerate}
\end{mybox}


\begin{mybox}{212. Luft}
	\centering \( V = 3 *4 * 5 \unit{m};\quad T = 300 \unit{K};\quad p = 1 \unit{atm};\quad k_B = 1.38 * 10^{-23} \unit{J/K} \)
	\tcblower
	\begin{enumerate}
		\item \( pV = nk_BT \)
		\begin{flalign*}
			n &= \tfrac{pV}{k_BT} = \dl{1.47 * 10^{27}} &&
		\end{flalign*}
	\tcbline
		\item \(  \)
		\begin{flalign*}
			\rho_n &= \tfrac{n}{V} = \tfrac{p}{k_BT} = \dl{2.45 * 10^{19} \unit{cm^{-3}}} &&
		\end{flalign*}
	\tcbline
		\item \( u = 1.66 * 10^{-27} \unit{kg};\quad A_{\ch{O2}} = 32;\quad A_{\ch{N2}} = 28\)	
		\begin{flalign*}
			m_{\ch{O2}} &= 0.2n*32u &&\\
			m_{\ch{N2}} &= 0.8n*28u &&\\
			m &= m_{\ch{O2}} + m_{\ch{N2}} = 28.8nu = \dl{70.19 \unit{kg}} &&
		\end{flalign*}
	\tcbline
		\item \( \Delta p = \rho_mgh;\quad \rho_m = \tfrac{m}{V};\quad h = 3 \unit{m}\)
		\begin{flalign*}
			\Delta p &= \tfrac{\rho_mgh}{V} = \dl{34.43 \unit{Pa}} &&\\[2ex]
			\tfrac{\Delta p}{h} &= \tfrac{\rho_mg}{V} = \dl{11.48 \unit{Pa/m}} &&
		\end{flalign*}
	\end{enumerate}
\end{mybox}

\end{document}