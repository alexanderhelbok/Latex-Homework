\documentclass{alex_hü}

\name{Alexander Helbok}
\course{PS Physik}
\hwnumber{2}

\begin{document}
	\renewcommand{\labelenumi}{\arabic{enumi}.}

	
	\section*{A. Einheiten und Einheiten umwandeln}
	\begin{enumerate}
		\item Die Wellenlänge von Wasserstoff (H$\alpha$) liegt bei 6562 Angström ($\si{\angstrom};\; 10 \si{\angstrom} = 1 \si{\nano\m}$). Wie lautet der Wert in m? 
		\begin{flalign*}
			6562\; \si{\angstrom} &= 656.2 * 10^{-9} \si{ \m} = \dl{6.562 * 10^{-7} \si{ \m}} &&
		\end{flalign*}
		\item Die Entfernung zur Sonne beträgt circa 150 000 000 km. Wie lange braucht das Licht für diese Strecke?\\[1ex]
		$ c = 299 792 458 \si{\m\per\s}; \quad d \approx 15*10^{7} \si{\km} = 1.5 * 10^{11} \si{\m}$
		\begin{flalign*}
			\frac{1.5 * 10^{11} \si{\m}}{299 792 458 \si{\m\per\s}} &= \dl{500 \si{\s}}&&
		\end{flalign*}
		\item  Eine alte Einheit zur Messung des Gewichts ist das Pfund. Ein Pfund entspricht 459 g. Ein Sack Getreide hat ein Gewicht von 12 Pfund. Wieviel ist das in kg? 
		\begin{flalign*}
			1 \text{ lb} &= 0.459 \si{\kilogram}&& \\
			12 \text{ lb} &= 12 * 0.459 \si{\kilogram}&& \\
			12 \text{ lb} &= \dl{5.5 \si{\kilogram}}&&
		\end{flalign*}
		
		
	\end{enumerate}
	
	\section*{B. Geradlinige Bewegung}
	\begin{enumerate}
		\item $ x_1 = -4 \si{\m}; \quad x_2 = +7 \si{\m} $\\[1.5ex]
		$ x_2 - x_1 = 7 \si{\m} - (-4) \si{\m} = \dl{11 \si{\m}}\quad \Rightarrow$ positive Bewegung\\ 
		\item $ x_1 = -3 \si{\m}; \quad x_2 = -8 \si{\m} $\\[1.5ex]
		$ x_2 - x_1 = -8 \si{\m} - (-3) \si{\m} = \dl{-5 \si{\m}}\quad \Rightarrow$ negative Bewegung\\
		\item $ x_1 = +5 \si{\m}; \quad x_2 = +9 \si{\m} $\\[1.5ex]
		$ x_2 -x_1 = 9 \si{\m} - 5 \si{\m} = \dl{4 \si{\m}}\quad \Rightarrow$ positive Bewegung\\
		\item $ x_1 = +2 \si{\m}; \quad x_2 = -1 \si{\m}$\\[1.5ex]
		$ x_2 - x_1 = -1 \si{\m} - 2 \si{\m} = \dl{-3 \si{\m}}\quad \Rightarrow$ negative Bewegung
	\end{enumerate}
 	\section*{C. Geradlinige Bewegung }
 	\begin{align*}
 		s_1 &= 10.4 \si{\km} = 10400 \si{\m}
 		&s_2 &= 2.5 \si{km} = 2500 \si{\m}\\
 		\Delta t_1 &= \frac{s_1}{v_1} = 468 \si{\s}
 		&\Delta t_2 &= 1800 \si{\s} \\
 		v_1 &= 80 \si{\km\per\hour} = 22.\dot{2} \si{\m\per\s}
 		&v_2 &= \frac{s_2}{\Delta t_2} = \frac{25}{18} \si{\m\per\s} \\
 	\end{align*}
 	\setlength{\columnsep}{-3cm}
	\begin{multicols}{2}
 			\begin{tikzpicture}
 				\begin{axis}[
 					width=207pt,
 					height=207pt,
 					axis lines=center,
 					axis line style={Stealth-Stealth},
 					xmin=-180,xmax=2500,ymin=-1080,ymax=15500,
 					xlabel style={below},
 					xtick={500, 1000, 1500, 2000},
 					y tick label style={/pgf/number format/.cd,%
 						scaled y ticks = false,
 						fixed},
 					xlabel=$t$ in s,
 					ylabel=$s$ in m,
 					grid=major,
 					grid style={thin,densely dotted,black!20}]
 					\addplot [-, thick,  black]
 					coordinates { (0,0) (468, 10400)};
 					\addplot [-, thick,  black]
 					coordinates { (468, 10400) (2268, 12900)} node[right,pos=1] {$f$};
					\addplot [-, thick, dotted,  black]	
 					coordinates { (2268,0) (2268, 12900)} node[left,pos=0.5] {$ \Delta s $};	
					\addplot [-, thick, dotted,  black]	
 					coordinates { (0,12900) (2268, 12900)} node[above,pos=0.5] {$ \Delta t $};
 					\addplot [-, thick,  dotted,  black]
 					coordinates { (0, 0) (2268, 12900)} node[left,pos=0.5,rotate=45,anchor=south] {$\bar{v} = \frac{\Delta s}{\Delta t}$};	
 				\end{axis}
 			\end{tikzpicture}\\
 		\columnbreak
 		\begin{equation*}
 			f(t) = \begin{cases}
 				22.\dot{2}t & \quad \text{für }\ 0 \le t \le 468, \\
 				\frac{25}{18}t & \quad $für $\ 468 < t \le 2268.
 			\end{cases}
 		\end{equation*}\\
	 	\begin{enumerate}
	 		\item Wie groß ist Ihre Bewegung insgesamt, gemessen vom Anfang der Fahrt bis zur Ankunft an der Tankstelle? 
	 		\begin{flalign*}
	 		\Delta s &= 10.4 \si{\km} + 2.5 \si{\km} = 12.9 \si{km}&& \\
			\Delta s &= \dl{12900 \si{\m}}&&%\\[2ex]
%			\Delta s &= f(2268) - f(0) = 12.9 \si{\km} - 0 \si{\km}&& \\
%			\Delta s &= \dl{12900 \si{\m}}&&
	 		\end{flalign*}
 		\end{enumerate}
	\end{multicols}
	\begin{enumerate}
		\setcounter{enumi}{1}
		\item  Wie groß ist das Zeitintervall $\Delta$t zwischen dem Anfang der Fahrt und der Ankunft an der Tankstelle? 
		\begin{flalign*}
			\Delta t &= 468 \si{\s} + 1800 \si{\s} = \dl{2268 \si{\s}}&& \\
%			t_{Ende} &= (f(t) = 12900); \quad t_{Start} = (f(t) = 0)&& \\
%			t_{Ende} &= 2268 \si{\s}; \quad t_{Start} = 0 \si{\s}&& \\
%			\Delta t &= t_{Ende} - t_{Start} = 2268 \si{\s} - 0 \si{\s} = \dl{2268 \si{\s}}
		\end{flalign*}
		\item Wie groß ist Ihre Durchschnittsgeschwindigkeit? 
		\begin{flalign*}
			\bar{v} &= \frac{\Delta s}{\Delta t} = \frac{12900}{2268} \si{\m\per\s} = \dl{5.69 \si{\m\per\s}}&&\\
%			\bar{v} &= \frac{f(2268)-f(0)}{2268 - 0} = \frac{12900}{2268} \si{\m\per\s} = \dl{5.69 \si{\m\per\s}}&&\\
		\end{flalign*}
	\end{enumerate}
 	
\end{document}