\documentclass[12pt,letterpaper]{article}
\usepackage{fullpage}
\usepackage[top=2cm, bottom=4.5cm, left=2.5cm, right=2.5cm]{geometry}
\usepackage[utf8]{inputenc}
\usepackage[T1]{fontenc}
\usepackage{fontspec}
\usepackage[ngerman]{babel}
\usepackage{amsmath,amsthm,amsfonts,amssymb,amscd,siunitx}
\usepackage{lastpage}
\usepackage{enumerate}
\usepackage{fancyhdr}
\usepackage{mathrsfs}
\usepackage{xcolor}
\usepackage{graphicx}
\usepackage{listings}
\usepackage{hyperref}
\usepackage{enumitem}
\usepackage{pgfplots}
\usepackage{tikz}
\usetikzlibrary{arrows.meta,positioning, calc}


\hypersetup{%
	colorlinks=true,
	linkcolor=blue,
	linkbordercolor={0 0 1}
}

\setlength{\parindent}{0.0in}
\setlength{\parskip}{0.05in}

% Edit these as appropriate
\newcommand\course{PS Vorbereitungskurs Mathematik}
\newcommand\hwnumber{1}                  % <-- homework number
\newcommand\NetIDa{Alexander Helbok}           % <-- NetID of person #1

\pagestyle{fancyplain}
\headheight 35pt
\lhead{\NetIDa}
\lhead{\NetIDa}                 % <-- Comment this line out for problem sets (make sure you are person #1)
\chead{\textbf{\Large Blatt \hwnumber}}
\rhead{\course \\ \today}
\lfoot{}
\cfoot{}
\rfoot{\small\thepage}
\headsep 1.5em

\begin{document}
	\renewcommand{\labelenumi}{\alph{enumi})}
	\def\dl#1{\underline{\underline{#1}}}
	
	\section*{2)}
	$A = (0, 0);\ B = (4, 0);\ C = (4, 1)$
	\begin{enumerate}
		\item  Zeichnen Sie die Koordinatenpfeile in ein Diagramm mit x- und y- Koordinatenachsen
		und geeigneten Koordinateneinteilungen \\
		\begin{center}
			\begin{tikzpicture}
				
				\begin{axis}[
					width=207pt,
					height=143pt,
					axis lines=center,
					axis line style={Stealth-Stealth},
					xmin=-1,xmax=5.5,ymin=-1,ymax=3.5,
					xtick distance=1,
					ytick distance=1,
					xlabel=$x$,
					ylabel=$y$,
					grid=major,
					grid style={thin,densely dotted,black!20}]
					\node[label={45:{A}},circle,fill,inner sep=2pt] at (axis cs:0,0) {};
					\node[label={90:{B}},circle,fill,inner sep=2pt] at (axis cs:4,0) {};
					\node[label={90:{C}},circle,fill,inner sep=2pt] at (axis cs:4,1) {};
					\addplot [->, thick,  black]
					coordinates { (0,0) (4,1)} node[above,pos=0.5,rotate=14.04,anchor=south] {\small$AC$};
					\addplot [->, thick,  black]
					coordinates { (0,0) (4,0)} node[above,pos=0.75] {\small$AB$};	
	    			\begin{scope}[shift={(0.95cm,0.76cm)}]
						\draw (0,0) -- (0:2.2cm) arc (0:29:1cm);
						\draw (1.75cm,0.185cm) node {\small$\varphi$};
					\end{scope}
				\end{axis}
			\end{tikzpicture}
		\end{center}

		\item Berechnen Sie die Länge der Strecke $AC$ \\
			\begin{flalign*}
				AC &= C-A = \begin{pmatrix}4\\1\end{pmatrix}&& \\
				\vert AC \vert &= \sqrt{17} \approx \dl{4.12}&&
			\end{flalign*}
		\item Berechnen Sie den Winkel an A zwischen $AB$ und $AC$ \\
			\begin{flalign*}
				\varphi &= \arccos\left(\frac{AB * AC}{\vert AB \vert * \vert AC \vert}\right)&& \\
				\varphi &= \arccos\left(\frac{16}{4 * \sqrt{17}}\right)&& \\
				\varphi &\approx \dl{\ang{14.04}}
			\end{flalign*}
		\item Der Punkt $C$ wird nun um $(-1, +1)$ verschoben. Skizzieren Sie auch die neue Lage C’ von C in das Diagramm \\
		\begin{center}
			\begin{tikzpicture}
				
				\begin{axis}[
					width=207pt,
					height=143pt,
					axis lines=center,
					axis line style={Stealth-Stealth},
					xmin=-1,xmax=5.5,ymin=-1,ymax=3.5,
					xtick distance=1,
					ytick distance=1,
					xlabel=$x$,
					ylabel=$y$,
					grid=major,
					grid style={thin,densely dotted,black!20}]
					\node[label={135:{A}},circle,fill,inner sep=2pt] at (axis cs:0,0) {};
					\node[label={90:{B}},circle,fill,inner sep=2pt] at (axis cs:4,0) {};
					\node[label={90:{C'}},circle,fill,inner sep=2pt] at (axis cs:3,2) {};
					\addplot [->, thick,  black]
					coordinates { (0,0) (3,2)} node[above,pos=0.5,rotate=33.69,anchor=south] {\small$AC'$};
					\addplot [->, thick,  black]
					coordinates { (0,0) (4,0)} node[above,pos=0.5] {\small$AB$};					
	    			\begin{scope}[shift={(0.95cm,0.76cm)}]
					    \draw (0,0) -- (0:1.25cm) arc (0:39:1cm);
						\draw (0.8cm,0.2cm) node {$\varphi$};
	    			\end{scope}
				\end{axis}
			\end{tikzpicture}
		\end{center}
		\item Berechnen Sie wiederum die Länge der Strecke $AC'$
			\begin{flalign*}
				AC' &= C'-A = \begin{pmatrix}3\\2\end{pmatrix}&& \\
				\vert AC' \vert &= \sqrt{13} \approx \dl{3.61}&&
			\end{flalign*}
		\item Berechnen Sie wiederum den Winkel an A\\
			\begin{flalign*}
				\varphi &= \arccos\left(\frac{AB * AC'}{\vert AB \vert * \vert AC' \vert}\right)&& \\
				\varphi &= \arccos\left(\frac{12}{4 * \sqrt{13}}\right)&& \\
				\varphi &\approx \dl{\ang{33.69}}
			\end{flalign*}
		\item Strecken (“skalieren”) Sie die Strecke AC’ um den skalaren Faktor $s = 4:$ das heißt
		die Länge $\vert AC'' \vert = s \vert AC' \vert$ ändert sich auf das Dreifache während der Winkel an A gleich
		bleibt. Auf welchem Koordinatentupel liegt der so erhaltenen Punkt C” ? \\
		\begin{align*}
			\vert AC'' \vert &= 4*\sqrt{13}, \qquad	C'' = (x,y), \qquad \varphi \approx \ang{33.69}&& \\[2.5ex]
			x &= \cos\left(\varphi\right)*\vert AC'' \vert  
			&y &= \sin\left(\varphi\right)*\vert AC'' \vert  \\
			x &= \cos\left(33.69\right) * 4*\sqrt{13} 
			&y &\approx \sin\left(33.69\right) * 4*\sqrt{13} \\
			x &= 12 
			&y &= 8 
		\end{align*}
		$\Rightarrow C'' = \dl{(12,8)}$
	\end{enumerate}



\end{document}