\documentclass{alex_hü}

\name{Alexander Helbok}
\course{PS Physik}
\hwnumber{3}


\begin{document}
\renewcommand{\labelenumi}{\alph{enumi})}


\begin{mybox}{Millikan-Versuch}
	\centering \( \eta = 18.6 \times 10^{-6}\unit{Pas};\quad \rho_{\text{air}} = 1.16 \unit{kg/m^3};\quad \rho_{\text{oil}} = 900 \unit{kg/m^3} \)
	\tcblower
	\begin{enumerate}
		\item \( v = 3.7 \times 10^{-6} \unit{\v} \)
		\begin{flalign*}
			F_{\text{g}} &= Vg\rho_{\text{oil}} = \tfrac{4}{3}r^3\pi g\rho_{\text{oil}} &&\\
			F_{\text{R}} &= a\eta v = 6\pi r\eta v &&\\
			F_{\text{A}} &= Vg\rho_{\text{air}} = \tfrac{4}{3}r^3\pi g\rho_{\text{air}} &&\\
			F_{\text{tot}} &= F_{\text{g}} - F_{\text{A}} - F_{\text{R}} = 0 &&\\
			r &= \sqrt{\tfrac{9\eta v}{2g\left(\rho_{\text{oil}} - \rho_{\text{air}}\right)}} = \dl{1.88 \times 10^{-7} \unit{m}} &&
 		\end{flalign*}
	\tcbline
		\item \( E = 500 \unit{V/m}:\quad v = 0 \unit{m/s} \)
		\begin{flalign*}
			F_{\text{el}} &= qE &&\\
			F_{\text{tot}} &= F_{\text{g}} - F_{\text{A}} - F_{\text{R}} - F_{\text{el}} = 0 &&\\
			q &= \tfrac{2\pi r\left(2gr^2\left(\rho_{\text{air}} - \rho_{\text{oil}}\right) + 9\eta v\right)}{3E} = \dl{4.86 \times 10^{-19} \unit{C}} &&
		\end{flalign*}
	\tcbline
		\item \( v = 1.2 \times 10^{-6} \unit{\v} \)
		Case 1: Oil receives an electron; Case 2: Oil loses a Neutron \\[1em]
		Case 1: 
		\begin{flalign*}
			F_{\text{tot}} &= F_{\text{g}} - F_{\text{A}} - F_{\text{R}} - F_{\text{el}} = 0 &&\\
			q &= \dl{6.44 \times 10^{-19} \unit{C} = 4 e} &&
		\end{flalign*}
		The roentgen rays knocked an electron out of an air molecule and into the oil droplet. It now has a net charge of \( 4^{-} \)
	\end{enumerate}
\end{mybox}
\newpage
\begin{mybox}{Massenspektrograph - Parabelmethode}
	\centering \(  \)
	\tcblower
	\begin{enumerate}
		\item \( T := \tfrac{l}{v_z} \)
		\begin{flalign*}
			F_{\text{E}} &= qE = m\dv{v_y}{t} &&\\
			v_y(t) &= \tfrac{qE}{m}t &&\\
			v_y(T) &= \dl{\tfrac{qEl}{mv_z}} &&\\[2ex]
			F_{\text{B}} &= qv_zB = m\dv{v_x}{t} &&\\
			v_x(t) &= \tfrac{qB}{m} v_zt &&\\
			v_x(T) &= \dl{\tfrac{qBl}{m}} &&\\[2ex]
			A_{\text{E}} &= \uint[0, T]{v_y(t)}{t} = \tfrac{qEl^2}{2mv_z^2} &&\\
			A_{\text{B}} &= \uint[0, T]{v_x(t)}{t} = \dl{\tfrac{qBl^2}{2mv_z}} &&
		\end{flalign*}
	\tcbline
		\item \( \tau = \tfrac{D}{v_z} \)
		\begin{flalign*}
			y &= A_{\text{E}} + v_y(T)\tau = \tfrac{qEl}{mv_z^2} \left(\tfrac{l}{2} + D \right) &&\\
			x &= A_{\text{B}} + v_x(T)\tau = \tfrac{qBl}{mv_z} \left(\tfrac{l}{2} + D \right) &&\\
			v_z &= \tfrac{qBl \left( \tfrac{l}{2} + D\right)}{mx} &&\\
			y(x) &= \dl{\tfrac{mE}{qB^2l \left(\tfrac{l}{2} + D \right)}}x^2 &&
		\end{flalign*}
	\tcbline
		\item \( x = 0.05 \unit{m};\quad D = 0.4 \unit{m};\quad l = 0.03 \unit{m};\quad a = 1.5 \unit{mm};\quad \Delta m = 2\unit{u} \\
		 U = 15 \unit{V};\quad B = 0.1 \unit{T} \)
		\begin{flalign*}
			\Delta y &= \tfrac{\Delta mE}{qB^2l \left(\tfrac{l}{2} + D \right)}x^2 = \dl{0.004 \unit{m}} &&
		\end{flalign*}
	\end{enumerate}
\end{mybox}

\begin{mybox}{Gruppengeschwindigkeit eines freien Teilchens}
	\centering \( \lambda_{\text{dB}} = \tfrac{h}{mv_T} \)
	\tcblower
	\begin{enumerate}
		\item The group velocity of a wave is the velocity of the enveloping wave, whereas the phase velocity is the speed the the inscribed wave travels through space.
	\tcbline
		\item Dispersion relation gives the relationship between the speed and the frequency of a wave. dispersion usually happens when waves go from one medium into another and there a change in speed happens. Electromagnetic waves in vacuum don't show dispersion because it always travels in the same medium and the frequency of em-waves does not depend on its speed. Matter waves however do have a speed dependent frequency and therefore do experience dispersion. 
	\tcbline
		\item \( p = mv_g;\quad E_{\text{kin}} = \tfrac{mv_g^2}{2} = \tfrac{p^2}{2m} \)
		\begin{flalign*}
			v_g &= v_T = \frac{p}{m} = \dl{\pdv{E_{\text{kin}}}{p}} &&
		\end{flalign*}
	\tcbline
		\item \( k = \tfrac{2\pi}{\lambda} = \tfrac{mv_T}{\hslash};\quad \omega = \tfrac{v_T\pi}{\lambda} = \tfrac{E_{\text{kin}}}{\hslash} = \tfrac{\hslash k^2}{2m} \)
		\begin{flalign*}
			v_g &= \pdv{\omega}{k} = \frac{\hslash k}{m} = \dl{v_T} &&
		\end{flalign*}
	\end{enumerate}
\end{mybox}


\end{document}