\documentclass{alex_hü}

\name{Alexander Helbok}
\course{PS Physik}
\hwnumber{3}


\begin{document}
\renewcommand{\labelenumi}{\alph{enumi})}


\begin{mybox}{Millikan-Versuch}
	\centering \( \nu = 18.6 * 10^{-6}\unit{Pas};\quad \rho_{\text{air}} = 1.16 \unit{kg/m^3};\quad \rho_{\text{oil}} = 900 \unit{kg/m^3} \)
	\tcblower
	\begin{enumerate}
		\item \( v = 3.7 * 10^{-6} \unit{\v} \)
		\begin{flalign*}
			F_{\text{g}} &= Vg\rho_{\text{oil}} = \tfrac{4}{3}r^3\pi g\rho_{\text{oil}} &&\\
			F_{\text{R}} &= \tfrac{v^2C_{\text{D}} A\rho_{\text{air}}}{2} = \tfrac{v^2 0.43 r^2 \pi\rho_{\text{air}}}{2} &&\\
			F_{\text{A}} &= Vg\rho_{\text{air}} = \tfrac{4}{3}r^3\pi g\rho_{\text{air}} &&\\
			F_{\text{tot}} &= F_{\text{g}} - F_{\text{A}} - F_{\text{R}} =  &&
 		\end{flalign*}
	\tcbline
		\item \(  \)
%		\begin{flalign*}
%		
%		\end{flalign*}
	\tcbline
		\item \(  \)
%		\begin{flalign*}
%			
%		\end{flalign*}
	\end{enumerate}
\end{mybox}

\begin{mybox}{Massenspektrograph - Parabelmethode}
	\centering \(  \)
	\tcblower
	\begin{enumerate}
		\item \(  \)
%		\begin{flalign*}
	%			
%		\end{flalign*}
	\tcbline
		\item \(  \)
%		\begin{flalign*}
	%		
%		\end{flalign*}
	\tcbline
		\item \(  \)
%		\begin{flalign*}
		%			
%		\end{flalign*}
	\end{enumerate}
\end{mybox}

\begin{mybox}{Gruppengeschwindigkeit eines freien Teilchens}
	\centering \(  \)
	\tcblower
	\begin{enumerate}
		\item \(  \)
%		\begin{flalign*}
		%			
%		\end{flalign*}
	\tcbline
		\item \(  \)
%		\begin{flalign*}
	%		
%		\end{flalign*}
	\tcbline
		\item \(  \)
%		\begin{flalign*}
		%			
%		\end{flalign*}
	\tcbline
		\item 
%		\begin{flalign*}
%			
%		\end{flalign*}
	\end{enumerate}
\end{mybox}


\end{document}