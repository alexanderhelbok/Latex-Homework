\documentclass{alex_hü}

\name{Alexander Helbok}
\course{PS Physik}
\hwnumber{4}


\begin{document}
\renewcommand{\labelenumi}{\alph{enumi})}


\begin{mybox}{Materiewellen und Wellenfunktionen}
	\centering \( \Psi(x) = \begin{cases}
		Ax(a-x) \quad &\text{für } 0 \leq x \leq a \\
		0 \quad &\text{für }  x < 0 \text{ und } x > a \\
	\end{cases} \)
	\tcblower
	\begin{enumerate}
		\item \(  \)
		\begin{flalign*}
			1 &= \uint[-\infty,\infty]{\Psi(x)}{x} = \uint[0,a]{Ax(a-x)}{x} = \tfrac{a^3A}{6} &&\\
			A &= \dl{\tfrac{6}{a^3}} &&
		\end{flalign*}
	\tcbline
		\item \(  \)
%		\begin{flalign*}
%		
%		\end{flalign*}
	\tcbline
		\item \( a_1 = 1 \unit{m};\quad a_2 = 0.5 \times 10^{-10} \unit{m} \)
%		\begin{flalign*}
%			
%		\end{flalign*}
	\end{enumerate}
\end{mybox}

\begin{mybox}{Doppelspaltversuch mit Elektronen}
	\centering \(  \)
	\tcblower
	\begin{enumerate}
		\item \(  \)
%		\begin{flalign*}
	%			
%		\end{flalign*}
	\tcbline
		\item \( E = 10 \unit{eV};\quad d = 3 \unit{nm} \)
%		\begin{flalign*}
	%		
%		\end{flalign*}
	\tcbline
		\item \(  \)
%		\begin{flalign*}
		%			
%		\end{flalign*}
	\end{enumerate}
\end{mybox}

\begin{mybox}{Neutronen im Interferometer}
	\centering \(  \)
	\tcblower
	\begin{enumerate}
		\item \(  \)
%		\begin{flalign*}
		%			
%		\end{flalign*}
	\tcbline
		\item \(  \)
%		\begin{flalign*}
	%		
%		\end{flalign*}
	\end{enumerate}
\end{mybox}

\end{document}