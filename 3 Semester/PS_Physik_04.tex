\documentclass{alex_hü}

\name{Alexander Helbok}
\course{PS Physik}
\hwnumber{4}


\begin{document}
\renewcommand{\labelenumi}{\alph{enumi})}


\begin{mybox}{Materiewellen und Wellenfunktionen}
	\centering \( \Psi(x) = \begin{cases}
		Ax(a-x) \quad &\text{für } 0 \leq x \leq a \\
		0 \quad &\text{für }  x < 0 \text{ und } x > a \\
	\end{cases} \)
	\tcblower
	\begin{enumerate}
		\item \(  \)
		\begin{flalign*}
			1 &= \uint[-\infty,\infty]{\abs{\Psi(x)}^2}{x} = \uint[0,a]{A^2x^2(a-x)^2}{x} = \tfrac{a^5A^2}{30} &&\\
			A &= \dl{\sqrt{\tfrac{30}{a^5}}} &&
		\end{flalign*}
	\tcbline
		\item \( k = \tfrac{2\pi}{\lambda} = \tfrac{mv_{\text{T}}}{\hbar} = \tfrac{p}{\hbar};\quad \omega = \tfrac{2\pi v_{\text{T}}}{\lambda} = \tfrac{mv_{\text{T}}^2}{2\hbar} = \tfrac{E}{\hbar} = \tfrac{p^2}{2m\hbar} = \tfrac{k^2\hbar}{2m} \)
		\begin{flalign*}
			v_{\text{g}} &= \dv{\omega}{k} = \dl{\tfrac{k\hbar}{m} = \tfrac{p}{m}}&&
		\end{flalign*}
	\tcbline
		\item \( a_1 = 1 \unit{m};\quad a_2 = 0.5 \times 10^{-10} \unit{m} \)
		\begin{flalign*}
			\lambda_{\text{db}} &= \tfrac{h}{mv_{\text{T}}} &&\\
			E_{\text{ges}} &= k\tfrac{e^2}{}
			v_{\text{T}} &= 
		\end{flalign*}
	\end{enumerate}
\end{mybox}

\begin{mybox}{Doppelspaltversuch mit Elektronen}
	\centering \(  \)
	\tcblower
	\begin{enumerate}
		\item \(  \)
%		\begin{flalign*}
	%			
%		\end{flalign*}
	\tcbline
		\item \( E = 10 \unit{eV};\quad d = 3 \unit{nm} \)
%		\begin{flalign*}
	%		
%		\end{flalign*}
	\tcbline
		\item \(  \)
%		\begin{flalign*}
		%			
%		\end{flalign*}
	\end{enumerate}
\end{mybox}

\begin{mybox}{Neutronen im Interferometer}
	\centering \(  \)
	\tcblower
	\begin{enumerate}
		\item \(  \)
		\begin{flalign*}
			\tfrac{mv_f^2}{2} &= \tfrac{mv_i^2}{2} - mgh &&\\
			v_f &= \sqrt{v_i - 2gh} &&
		\end{flalign*}
	\tcbline
		\item \(  \)
%		\begin{flalign*}
	%		
%		\end{flalign*}
	\end{enumerate}
\end{mybox}

\end{document}