\documentclass{alex_hü}

\name{Alexander Helbok}
\course{PS Physik}
\hwnumber{5}


\begin{document}
\renewcommand{\labelenumi}{\alph{enumi})}


\begin{mybox}{Wiensches Verschiebungsgesetz}
	\centering \( \rho(\nu)\dd{\nu} = \tfrac{8\pi h\nu^3}{c^3}\tfrac{\dd{\nu}}{\expo[h\nu/k_{\text{B}}T] - 1} \)
	\tcblower
	\begin{enumerate}
		\item \( \nu = \tfrac{c}{\lambda};\quad \dd{\nu} = -\tfrac{c}{\lambda^2}\dd{\lambda} \)
		\begin{flalign*}
			\rho(\nu)\dd{\nu} &= \tfrac{8\pi h\nu^3}{c^3}\tfrac{\dd{\nu}}{\expo[h\nu/k_{\text{B}}T] - 1} &&\\
			\rho(\lambda)\dd{\lambda} &= \dl{\tfrac{8\pi ch}{\lambda^5}\tfrac{\dd{\lambda}}{\expo[hc/\lambda k_{\text{B}}T] - 1}} &&
		\end{flalign*}
	\tcbline
		\item \(  \)
%		\begin{flalign*}
%		
%		\end{flalign*}
	\end{enumerate}
\end{mybox}

\begin{mybox}{Photoeffekt}
	\centering \( W = 2.9 \unit{eV} \)
	\tcblower
	\begin{enumerate}
		\item \(  E > \dl{W = 2.9 \unit{eV}}  \)
	\tcbline
		\item \( E = hf;\quad \lambda = \tfrac{c}{f} \)\\[2ex]
		\( \lambda = \tfrac{ch}{E} = \dl{4.28 \times 10^{-7} \unit{m}} \)
	\tcbline
		\item \( \lambda = 400 \unit{nm};\quad I = 1 \unit{mA} \)
%		\begin{flalign*}
		%			
%		\end{flalign*}
	\tcbline
		\item \(  \)
%		\begin{flalign*}
	%			
%		\end{flalign*}
	\tcbline
		\item \(  \)
%		\begin{flalign*}
	%			
%		\end{flalign*}
	\tcbline
		\item \(  \)
%		\begin{flalign*}
	%			
%		\end{flalign*}
	\end{enumerate}
\end{mybox}

\begin{mybox}{Zerfließen eines Gauß-Pakets}
	\vspace{-0.4cm}
	\centering \[ \psi(x, t) = \tfrac{\sqrt{a}}{(2\pi)^{3/4}} \uint[-\infty,\infty]{\exp\bigg(-\tfrac{a^2}{4}(k - k_0)^2 \bigg)\exp\bigg(\iu(kx - \omega(k)t) \bigg)}{k} \]
	\tcblower
	\begin{enumerate}
		\item \(  \)
%		\begin{flalign*}
		%			
%		\end{flalign*}
	\tcbline
		\item \(  \)
%		\begin{flalign*}
	%		
%		\end{flalign*}
	\tcbline
		\item \(  \)
%		\begin{flalign*}
		%			
%		\end{flalign*}
	\end{enumerate}
\end{mybox}


\end{document}