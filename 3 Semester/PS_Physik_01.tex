\documentclass{alex_hü}

\name{Alexander Helbok}
\course{PS Physik}
\hwnumber{1}


\begin{document}
\renewcommand{\labelenumi}{\alph{enumi})}


\begin{mybox}{Messung der Avogadro-Konstante durch radioaktiven Zerfall}
	\centering \(  \)
	\tcblower
	\begin{enumerate}
		\item \(  \)
%		\begin{flalign*}
%			
%		\end{flalign*}
	\tcbline
		\item \(  \)
%		\begin{flalign*}
%		
%		\end{flalign*}
	\tcbline
		\item \(  \)
%		\begin{flalign*}
%			
%		\end{flalign*}
	\end{enumerate}
\end{mybox}

\begin{mybox}{Ebene Wellen und Kugelwellen}
	\centering \(  \)
	\tcblower
	\begin{enumerate}
		\item \(  \)
		\begin{flalign*}
			E_1(r, t) &= A \cos(\omega t + kx) &&	
		\end{flalign*}
	\tcbline
		\item \(  \)
%		\begin{flalign*}
	%		
%		\end{flalign*}
	\tcbline
		\item \(  \)
		\begin{flalign*}
			E_1(r, t) &= A \cos(\omega t + kx) = A \expo[ikx - i\omega t] &&
		\end{flalign*}
	\end{enumerate}
\end{mybox}

\begin{mybox}{Bragg-Reflexion}
	\centering \( \rho = 8.91 \unit{g/cm3};\quad m = 63.5 \unit{u} \)
	\tcblower
	\begin{enumerate}
		\item \(  \)
%		\begin{flalign*}
		%			
%		\end{flalign*}
	\tcbline
		\item \( \theta_1 = \ang{20};\quad d_1 = a \)
%		\begin{flalign*}
	%		
%		\end{flalign*}
	\tcbline
		\item \(  \)
%		\begin{flalign*}
		%			
%		\end{flalign*}
	\end{enumerate}
\end{mybox}


\end{document}