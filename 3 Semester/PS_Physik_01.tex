\documentclass{alex_hü}

\name{Alexander Helbok}
\course{PS Physik}
\hwnumber{1}


\begin{document}
\renewcommand{\labelenumi}{\alph{enumi})}


\begin{mybox}{Messung der Avogadro-Konstante durch radioaktiven Zerfall}
	\centering \(  \)
	\tcblower
	\begin{enumerate}
		\item \(  \)
		\begin{flalign*}
			N &= \dl{3\Gamma t} &&\\
			N_{\text{A}} &= \dl{\tfrac{NRT}{pV}} &&
		\end{flalign*}
	\tcbline
		\item \( p = 1 \unit{atm};\quad T = 273.15 \unit{K};\quad V = 14 \unit{mm^3};\quad \Gamma = 7.2 * 10^{12} \unit{1/s} \)
		\begin{flalign*}
			N_{\text{A}} &= \tfrac{NRT}{pV} = \dl{6.23 * 10^{23} \unit{1/mol}} &&\\[2em]
			\Gamma(t) &= \Gamma_0 \left(\tfrac{1}{2}\right)^{\tfrac{t}{T}} &&
		\end{flalign*}
		From the equation above one can see that the bigger the half life \( T \) of an element, the less will the activity \( \Gamma \) decrease with time. So an element with longer half life is better suited in this case. 
	\end{enumerate}
\end{mybox}

\begin{mybox}{Ebene Wellen und Kugelwellen}
	\centering \(  \)
	\tcblower
	\begin{enumerate}
		\item \(  \)
		\begin{flalign*}
			A_1(x, t) &= \tikzmark{1}\ A_0\ \tikzmark{2} \cos(\tikzmark{3}\, kx \, \tikzmark{4} - \tikzmark{5}\, \omega t \,\tikzmark{6}) &&\\	
		\end{flalign*}
		The Amplitude of the wave, denoted \( A_0 \), in case of water waves this tells us how high the wave can get (in respect to still water) and in respect to sound it tells you how much the air molecules that transmit the wave "wiggle". The cosine term is responsible for the oscillation and inside the cosine the wavenumber \( k \) tells you how many wavecycles happen in a distance of \( 1 \unit{m} \), so how often the waterwave reaches its maximum in \( 1 \unit{m} \). The angular frequency tells us something about how many wavecycles happen in a time period of \( 1 \unit{s} \).
		\AddUnderBrace[0.1]{1}{2}{Amplitute}
		\AddOverBrace[0.1]{3}{4}{Wavenumber}
		\AddUnderBrace[0.1]{5}{6}{Frequency}
	\tcbline
		\item \(  \)
		\begin{flalign*}
			A_2(r, \theta, \varphi) &= A_0 \cos(kr - \omega t) &&
		\end{flalign*}
		\( A_0 \propto \tfrac{1}{r^2} \) because the Energy of the wave gets spread out across an ever increasing surface (increasing with \( r^2 \))
	\tcbline
		\item \(  \)
		\begin{flalign*}
			A_1(x, t) &= A_0 \cos(kx - \omega t) = \dl{A_0 \expo[ikx - i\omega t]} &&\\
			A_2(r, \theta, \varphi) &= A_0 \cos(kr - \omega t) = \dl{A_0 \expo[ikr - i\omega t]} &&\\
		\end{flalign*}
	\end{enumerate}
\end{mybox}

\begin{mybox}{Bragg-Reflexion}
	\centering \( \rho = 8.91 \unit{g/cm3};\quad m = 63.5 \unit{u} \)
	\tcblower
	\begin{enumerate}
		\item \( \rho = \tfrac{m}{a^3} \)
		\begin{flalign*}
			a &= \sqrt[3]{\tfrac{m}{\rho}} = \dl{2.28 * 10^{-10} \unit{m}} &&
		\end{flalign*}
	\tcbline
		\item \( \theta_1 = \ang{20};\quad d_1 = a \)
		\begin{flalign*}
			\Delta s &= 2d\sin(\theta) = k\lambda = \lambda \hspace{6cm} (k=1 \text{ for 1.Order})&&\\
			\lambda &= 2d\sin(\theta_1) = \dl{1.56 * 10^{-10} \unit{m}} &&
		\end{flalign*}
	\tcbline
		\item \(  \)
		\begin{flalign*}
			d_2 &= \tfrac{a}{\sqrt{2^2 + 1^2 + 0^2}} = \tfrac{a}{\sqrt{5}} = \dl{1.02 * 10^{-10} \unit{m}} &&\\
			\theta_2 &= \arcsin(\tfrac{\lambda}{2d_2}) = \dl{\ang{49.89}} &&
		\end{flalign*}
	\end{enumerate}
\end{mybox}


\end{document}