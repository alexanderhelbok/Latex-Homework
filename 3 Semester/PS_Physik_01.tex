\documentclass{alex_hü}

\name{Alexander Helbok}
\course{PS Physik}
\hwnumber{1}


\begin{document}
\renewcommand{\labelenumi}{\alph{enumi})}


\begin{mybox}{Messung der Avogadro-Konstante durch radioaktiven Zerfall}
	\centering \(  \)
	\tcblower
	\begin{enumerate}
		\item \(  \)
		\begin{flalign*}
			N &= \dl{3\Gamma t} &&\\
			N_{\text{A}} &= \dl{\tfrac{NRT}{pV}} &&
		\end{flalign*}
	\tcbline
		\item \( p = 1 \unit{atm};\quad T = 273.15 \unit{K};\quad V = 14 \unit{mm^3};\quad \Gamma = 7.2 * 10^{12} \unit{s^{-1}} \)
		\begin{flalign*}
			N_{\text{A}} &= \tfrac{NRT}{pV} = \dl{6.23 * 10^{23} \unit{/mol}} &&
		\end{flalign*}
	\end{enumerate}
\end{mybox}

\begin{mybox}{Ebene Wellen und Kugelwellen}
	\centering \(  \)
	\tcblower
	\begin{enumerate}
		\item \(  \)
		\begin{flalign*}
			A_1(x, t) &= \tikzmark{1}\ A\ \tikzmark{2} \cos(kx - \omega t) &&	
		\end{flalign*}
		The Amplitude of the wave, denoted A, in case of water waves this tells us how high the wave can get (in respect to still water) and in respect to sound 
		\AddUnderBrace[0.1]{1}{2}{Amplitute}
	\tcbline
		\item \(  \)
%		\begin{flalign*}
	%		
%		\end{flalign*}
	\tcbline
		\item \(  \)
		\begin{flalign*}
			A_1(x, t) &= A \cos(\omega t + kx) = A \expo[ikx - i\omega t] &&
		\end{flalign*}
	\end{enumerate}
\end{mybox}

\begin{mybox}{Bragg-Reflexion}
	\centering \( \rho = 8.91 \unit{g/cm3};\quad m = 63.5 \unit{u} \)
	\tcblower
	\begin{enumerate}
		\item \( \rho = \tfrac{m}{a^3} \)
		\begin{flalign*}
			a &= \sqrt[3]{\tfrac{m}{\rho}} = \dl{2.28 * 10^{-10} \unit{m}} &&
		\end{flalign*}
	\tcbline
		\item \( \theta_1 = \ang{20};\quad d_1 = a \)
		\begin{flalign*}
			\Delta s &= 2d\sin(\theta) = k\lambda = \lambda \hspace{6cm} (k=1 \text{ for 1.Order})&&\\
			\lambda &= 2d\sin(\theta_1) = \dl{1.56 * 10^{-10} \unit{m}} &&
		\end{flalign*}
	\tcbline
		\item \(  \)
		\begin{flalign*}
			d_2 &= \tfrac{a}{\sqrt{2^2 + 1^2 + 0^2}} = \tfrac{a}{\sqrt{5}} = \dl{1.02 * 10^{-10} \unit{m}} &&\\
			\theta_2 &= \arcsin(\tfrac{\lambda}{2d_2}) = \dl{\ang{49.89}} &&
		\end{flalign*}
	\end{enumerate}
\end{mybox}


\end{document}