\documentclass{alex_hü}

\name{Alexander Helbok}
\course{PS Physik}
\hwnumber{2}


\begin{document}
\renewcommand{\labelenumi}{\alph{enumi})}


\begin{mybox}{Michelson-Interferometer}
	\centering \(  \)
	\tcblower
	\begin{enumerate}
		\item \(  \)
%		\begin{flalign*}
%			
%		\end{flalign*}
	\tcbline
		\item \(  \)
%		\begin{flalign*}
%		
%		\end{flalign*}
	\end{enumerate}
\end{mybox}

\begin{mybox}{Bestimmung des Wirkungsquerschnitts}
	\centering \(  \)
	\tcblower
	\begin{enumerate}
		\item \(  \)
%		\begin{flalign*}
	%			
%		\end{flalign*}
	\tcbline
		\item \(  \)
%		\begin{flalign*}
	%		
%		\end{flalign*}
	\tcbline
		\item \(  \)
%		\begin{flalign*}
		%			
%		\end{flalign*}
	\end{enumerate}
\end{mybox}

\begin{mybox}{Geladene Teilchen in \( \vec{E} \)- und \( \vec{B} \)- Feldern}
	\centering \( d = 0.105 \unit{m};\quad B = 1 \unit{mT};\quad U_{\text{b}} = 220 \unit{V} \)
	\tcblower
	\begin{enumerate}
		\item \( F_{\text{L}} = qv_{\perp}B = \tfrac{mv_{\perp}}{r};\quad E = Uq = \tfrac{mv_{\perp}^2}{2}\)
		\begin{flalign*}
			\tfrac{q}{m} &= \tfrac{2U}{d^2B^2} = \dl{3.99 * 10^{10} \unit{C/kg}} &&
		\end{flalign*}
	\tcbline
		\item \(  \)
%		\begin{flalign*}
	%		
%		\end{flalign*}
	\end{enumerate}
\end{mybox}


\end{document}