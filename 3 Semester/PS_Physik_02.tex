\documentclass{alex_hü}

\name{Alexander Helbok}
\course{PS Physik}
\hwnumber{2}


\begin{document}
\renewcommand{\labelenumi}{\alph{enumi})}


\begin{mybox}{Michelson-Interferometer}
	\centering \(  \)
	\tcblower
	\begin{enumerate}
		\item \( E_1 = E_0 \expo[\iu(kx - \omega t)];\quad E_2 = E_0 \expo[\iu(kx - \omega t + k\Delta s)];\quad I = E^2 \)
		\begin{flalign*}
			I &= E^2 = \left(E_1 + E_2 \right)^2 = E_0^2 \expo[2\iu(kx - \omega t)] \left(1 + \expo[\iu\Delta sk] \right)^2 &&\\
			I &= \dl{E_0^2 \left( 1 + 2\cos(\Delta s k) + \cos(\Delta sk)^2 - \sin(\Delta sk)^2 \right)} &&
		\end{flalign*}
	\tcbline
		\item 
		The Energy goes back into the source because the two beams reflected from the mirror get split again into a wave interfering destructively (going towards the detector) and one interfering constructively traveling back towards the source.
	\end{enumerate}
\end{mybox}

\begin{mybox}{Bestimmung des Wirkungsquerschnitts}
	\centering \(  \)
	\tcblower
	\begin{enumerate}
		\item A cathode releases electrons, which pass through a hole in a membrane into a chamber full of nitrogen. At the end the electrons hit a detector and from the current \( I_{0} \) the electrons produce one can determine the cross-section of electrons and nitrogen.  
	\tcbline
		\item Instead of only varying the thickness of the medium, one can also change the pressure. I imagine that varying the thickness requires the detector to be mobile which increases the length of the airtight enclosure for the nitrogen and generally adds complexity because you now need a contraption that accurately moves the detector.\\
		\begin{flalign*}
			I(p) &= I_0\expo[-][\beta p] &&\\
			N(p) &= N_0\expo[-][\beta p] \qquad \text{,with } N_0 \propto I_0 &&\\
		\end{flalign*}
		\begin{flalign*}
			pV &= Nk_{\text{B}}T \qquad \Rightarrow \qquad p = \tfrac{Nk_{\text{b}}T}{V} &&\\
			\Delta N &= -WN = \tfrac{\tfrac{\Delta pV}{k_{\text{B}}T}\sigma N}{A} = \tfrac{\Delta p x \sigma N}{k_{\text{B}}T} &&\\
			\tfrac{\dd{N}}{N} &= \tfrac{\dd{p} x \sigma}{k_{\text{B}}T} &&\\
			N(p) &= N_0\expo[-][x\sigma/k_{\text{B}}T][p] \qquad \Rightarrow \qquad \beta := \dl{\tfrac{x\sigma}{k_{\text{B}}T}} &&
		\end{flalign*}
	\tcbline
		\item \( x = 2.5 \unit{m};\quad T = 300 \unit{K} \\ 
		p_1 = 2 * 10^{-2} \unit{Pa};\quad p_2 = 10^{-2} \unit{Pa};\quad  p_3 = 10^5 \unit{Pa};\quad p_4 = 7 * 10^4 \unit{Pa} \)
		\begin{flalign*}
			\expo[-][\beta p_1] &= 2 \expo[-][\beta p_2] &&\\
			\sigma_1 &= \tfrac{\ln(2)k_{\text{b}}T}{(p_1 - p_2) x} = \dl{1.15 * 10^{-19} \unit{m^2}} &&\\
			\expo[-][\beta p_3] &= 2 \expo[-][\beta p_4] &&\\
			\sigma_2 &= \tfrac{\ln(2)k_{\text{b}}T}{(p_3 - p_4) x} = \dl{3.83 * 10^{-26} \unit{m^2}} &&
		\end{flalign*}
		The Cross-section has drastically decreased because the more speed and momentum the electrons carry, the more they just smash through the gas and don't get scattered on the molecules 
	\end{enumerate}
\end{mybox}

\begin{mybox}{Geladene Teilchen in \( \vec{E} \)- und \( \vec{B} \)- Feldern}
	\centering \( d = 0.105 \unit{m};\quad B = 1 \unit{mT};\quad U_{\text{b}} = 220 \unit{V} \)
	\tcblower
	\begin{enumerate}
		\item \( F_{\text{L}} = qvB = \tfrac{mv}{r} \Rightarrow v = \tfrac{dqB}{m};\quad E = Uq = \tfrac{mv^2}{2} \Rightarrow v^2 = \tfrac{2Uq}{m}\)
		\begin{flalign*}
			\left( \tfrac{dqB}{m} \right)^2 &= \tfrac{2Uq}{m} &&\\
			\tfrac{q}{m} &= \tfrac{2U}{d^2B^2} = \dl{3.99 * 10^{10} \unit{C/kg}} &&
		\end{flalign*}
	\tcbline
		\item \( v = \sqrt{\tfrac{2Uq}{m}} \)
		\begin{flalign*}
			F_{\text{L}} &= q(E + vB) = 0 &&\\
			E &= -vB = \sqrt{\tfrac{2Uq}{m}}B &&\\
			\tfrac{q}{m} &= \dl{\tfrac{E^2}{UB^2}} &&
		\end{flalign*}
	\end{enumerate}
\end{mybox}


\end{document}