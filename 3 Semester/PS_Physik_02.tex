\documentclass{alex_hü}

\name{Alexander Helbok}
\course{PS Physik}
\hwnumber{2}


\begin{document}
\renewcommand{\labelenumi}{\alph{enumi})}


\begin{mybox}{Michelson-Interferometer}
	\centering \(  \)
	\tcblower
	\begin{enumerate}
		\item \( E_1 = E_0 \expo[\iu(kx - \omega t)];\quad E_2 = E_0 \expo[\iu(kx - \omega t + k\Delta s)];\quad I = E^2 \)
		\begin{flalign*}
			I &= E^2 = \left(E_1 + E_2 \right)^2 = E_0^2 \expo[2\iu(kx - \omega t)] \left(1 + \expo[\iu\Delta sk] \right)^2 &&\\
			I &= \dl{E_0^2} &&
		\end{flalign*}
	\tcbline
		\item 
		The Energy goes back into the source because the two beams reflected from the mirror get split again into a wave interfering destructively (going towards the detector) and one interfering constructively traveling back towards the source.
	\end{enumerate}
\end{mybox}

\begin{mybox}{Bestimmung des Wirkungsquerschnitts}
	\centering \(  \)
	\tcblower
	\begin{enumerate}
		\item A cathode releases electrons, which pass through a hole in a membrane into a chamber full of nitrogen. At the end the electrons hit a detector and from the current \( I_{0} \) the electrons produce one can determine the cross-section of electrons and nitrogen.  
	\tcbline
		\item Instead of only varying the distance between Nitrogen molecules in one plane only, changing the pressure will yield a similar result while being much easier to achieve. When varying the distance in one plane you have to keep the molecules from escaping in other directions, which is pretty hard using electrically neutral molecules. \\
		\begin{flalign*}
			I(p) = I_0\expo[-][\beta p]
		\end{flalign*}
	\tcbline
		\item \( x = 2.5 \unit{m};\quad T = 300 \unit{K} \\ 
		p_1 = 2 * 10^{-2} \unit{Pa};\quad p_2 = 10^{-2} \unit{Pa};\quad  p_3 = 10^5 \unit{Pa};\quad p_4 = 7 * 10^4 \unit{Pa} \)
%		\begin{flalign*}
	%			
%		\end{flalign*}
	\end{enumerate}
\end{mybox}

\begin{mybox}{Geladene Teilchen in \( \vec{E} \)- und \( \vec{B} \)- Feldern}
	\centering \( d = 0.105 \unit{m};\quad B = 1 \unit{mT};\quad U_{\text{b}} = 220 \unit{V} \)
	\tcblower
	\begin{enumerate}
		\item \( F_{\text{L}} = qv_{\perp}B = \tfrac{mv_{\perp}}{r} \Rightarrow v_{\perp} = \tfrac{dqB}{m};\quad E = Uq = \tfrac{mv_{\perp}^2}{2} \Rightarrow v_{\perp}^2 = \tfrac{2Uq}{m}\)
		\begin{flalign*}
			\left( \tfrac{dqB}{m} \right)^2 &= \tfrac{2Uq}{m} &&\\
			\tfrac{q}{m} &= \tfrac{2U}{d^2B^2} = \dl{3.99 * 10^{10} \unit{C/kg}} &&
		\end{flalign*}
	\tcbline
		\item \( v = \sqrt{\tfrac{2Uq}{m}} \)
		\begin{flalign*}
			F_{\text{L}} &= q(E + vB) = 0 &&\\
			E &= -vB = \sqrt{\tfrac{2Uq}{m}}B &&\\
			\tfrac{q}{m} &= \dl{\tfrac{E^2}{UB^2}} &&
		\end{flalign*}
	\end{enumerate}
\end{mybox}


\end{document}