\documentclass{alex_hü}

\name{Alexander Helbok}
\course{PS Physik}
\hwnumber{12}


\begin{document}
\renewcommand{\labelenumi}{\alph{enumi})}


\begin{mybox}{Atomvibrationen in einem Metall}
	\centering \(  \)
	\tcblower
	\begin{enumerate}
		\item \( \rho = \tfrac{3}{4\pi R^3};\quad V = \tfrac{4}{3}\pi r^3;\quad q = e\rho V;\quad k = \tfrac{1}{4\pi\epsilon_0} \)
		\begin{flalign*}
			E &= \tfrac{kq}{r^2} = \tfrac{1}{4\pi\epsilon_0} \tfrac{er}{R^3} &&\\
			F &= qE = kr &&\\
			\omega &= \SQRT{\tfrac{k}{mM}} = \dl{\sqrt{\tfrac{e^2}{4\pi\epsilon_0MR^3}}} &&
		\end{flalign*}
	\tcbline
		\item \(  \)
		\begin{flalign*}
			4R^2 &= 3a^2 \qquad \Rightarrow \qquad R = \dl{\tfrac{\sqrt{3}a}{2}} &&\\
			\omega &= \sqrt{\tfrac{e^2}{4\pi\epsilon_0MR^3}} = \dl{5.11 \times 10^{10} \unit{Hz}} &&
		\end{flalign*}
	\tcbline
		\item \( N = 9;\quad V = a^3 \)
		\begin{flalign*}
			v &= \tfrac{\Omega_{\text{D}}}{\sqrt[3]{\tfrac{6\pi^2N}{V}}} = \dl{1715} \unit{\v} &&
		\end{flalign*}
	\end{enumerate}
\end{mybox}

\begin{mybox}{Zwei-atomige Kette}
	\centering \(  \)
	\tcblower
	\begin{enumerate}
		\item \( \omega^2 = G\left(\tfrac{1}{m} + \tfrac{1}{M}\right) \pm G\SQRT{\left(\tfrac{1}{m} + \tfrac{1}{M}\right)^2 - \tfrac{4\sin[2](ka)}{Mm}};\quad ka \ll 1;\quad \mu = \tfrac{1}{m} + \tfrac{1}{M} \)
		\begin{flalign*}
			\omega^2 &= G\left(\tfrac{1}{m} + \tfrac{1}{M}\right) \pm G\SQRT{\left(\tfrac{1}{m} + \tfrac{1}{M}\right)^2 - \tfrac{4\sin[2](ka)}{Mm}} \approx \frac{G}{\mu} \pm G\SQRT{\tfrac{1}{\mu^2} - \tfrac{4(ka)^2}{Mm}} = &&\\
			&= \frac{G}{\mu} \pm \frac{G}{\mu} \SQRT{1 - \tfrac{4(ka)^2\mu^2}{Mm}} \approx \frac{G}{\mu} \pm \frac{G}{\mu} \left(1 - \tfrac{2(ka)^2\mu^2}{Mm}\right) &&\\[2ex]
			\omega_- &= \sqrt{\tfrac{2G(ka)^2\mu}{Mm}} = ka \sqrt{\tfrac{2G\mu}{Mm}} &&\\[2ex]
			v &= \frac{\omega_-}{k} = \dl{a \sqrt{\tfrac{2G\mu}{Mm}}} &&
		\end{flalign*}
	\tcbline
		\item \(  \)
		\begin{minipage}{\textwidth}
			\vspace{6cm}
		\end{minipage}
	\tcbline
		\item \(  \)
%		\begin{flalign*}
		%			
%		\end{flalign*}
	\tcbline
		\item \(  \)
%		\begin{flalign*}
%				
%		\end{flalign*}
	\end{enumerate}
\end{mybox}

\begin{mybox}{Eigenschaften eines Natriumkristalls}
	\centering \( m = 22.9897 \unit{u};\quad \rho = 0.968 \unit{g/cm^3}  \)
	\tcblower
	\begin{enumerate}
		\item \( V\rho = M  \)
		\begin{flalign*}
			a &= \sqrt[3]{\tfrac{9m}{\rho}} = \dl{7.08 \times 10^{-10} \unit{m}} &&
		\end{flalign*}
	\tcbline
		\item \( I \propto \abs{F_{\text{hkl}}}^2 \)
		\begin{flalign*}
			\abs{F_{\text{111}}}^2 &= f^2 \left(\abs{\expo[0] + \expo[3\iu\pi]}^2\right) = 0 &&\\
			\abs{F_{\text{110}}}^2 &= f^2 \left(\abs{\expo[0] + \expo[2\iu\pi]}^2\right) = 2f^2 &&
		\end{flalign*}
		The Intensity for a \( (1 1 1) \) Lattice is lower than for a (110) lattice.
	\tcbline
		\item \( T_{\text{D}} = 150 \unit{K} \)
		\begin{flalign*}
			\Omega_{\text{D}} &= \tfrac{T_{\text{D}}k_{\text{B}}}{\hbar} = \dl{1.96 \times 10^{13} \unit{Hz}} &&
		\end{flalign*}
	\end{enumerate}
\end{mybox}


\end{document}