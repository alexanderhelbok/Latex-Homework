\documentclass{alex_hü}

\name{Alexander Helbok}
\course{PS Physik}
\hwnumber{12}


\begin{document}
\renewcommand{\labelenumi}{\alph{enumi})}


\begin{mybox}{Atomvibrationen in einem Metall}
	\centering \(  \)
	\tcblower
	\begin{enumerate}
		\item \(  \)
%		\begin{flalign*}
%			
%		\end{flalign*}
	\tcbline
		\item \(  \)
%		\begin{flalign*}
%		
%		\end{flalign*}
	\tcbline
		\item \(  \)
%		\begin{flalign*}
%			
%		\end{flalign*}
	\end{enumerate}
\end{mybox}

\begin{mybox}{Zwei-atomige Kette}
	\centering \(  \)
	\tcblower
	\begin{enumerate}
		\item \(  \)
%		\begin{flalign*}
	%			
%		\end{flalign*}
	\tcbline
		\item \(  \)
%		\begin{flalign*}
	%		
%		\end{flalign*}
	\tcbline
		\item \(  \)
%		\begin{flalign*}
		%			
%		\end{flalign*}
	\end{enumerate}
\end{mybox}

\begin{mybox}{Eigenschaften eines Natriumkristalls}
	\centering \( m = 22.9897 \unit{u};\quad \rho = 0.968 \unit{g/cm^3}  \)
	\tcblower
	\begin{enumerate}
		\item \(  \)
		\begin{flalign*}
			a^3 &= \tfrac{9m}{\rho} = \dl{7.08 \times 10^{-10} \unit{m}} &&
		\end{flalign*}
	\tcbline
		\item \( I \propto \abs{F_{\text{hkl}}}^2 \)
		\begin{flalign*}
			\abs{F_{\text{111}}}^2 &= f^2 \left(\abs{\expo[0] + \expo[3\iu\pi]}^2\right) = 0 &&\\
			\abs{F_{\text{110}}}^2 &= f^2 \left(\abs{\expo[0] + \expo[2\iu\pi]}^2\right) = 2f^2 &&
		\end{flalign*}
		The Intensity for a \( (1 1 1) \) Lattice is lower than for a (110) lattice.
	\tcbline
		\item \( T_{\text{D}} = 150 \unit{K} \)
		\begin{flalign*}
			\Omega_{\text{D}} &= \tfrac{T_{\text{D}}k_{\text{B}}}{\hbar} = \dl{1.96 \times 10^{13} \unit{Hz}} &&
		\end{flalign*}
	\end{enumerate}
\end{mybox}


\end{document}