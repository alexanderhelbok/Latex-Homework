\documentclass{alex_hü}

\name{Alexander Helbok}
\course{PS Physik}
\hwnumber{7}
\usepackage{biblatex}

\begin{document}
\renewcommand{\labelenumi}{\alph{enumi})}


\begin{mybox}{Wellenfunktion und Aufenthaltswahrscheinlichkeit eines Teilchens}
	\centering \(  \Psi(x) = \begin{cases}
		2 \quad &\text{für } 0 \leq x \leq 1 \\
		1 \quad &\text{für }  1 \le x \le 2 \\
		-2 \quad &\text{für }  2 \leq x \leq 3
	\end{cases} \)
	\tcblower
	\begin{enumerate}
		\item The probability of finding the particle is the lowest in sector \RN{2}
	\tcbline
		\item the unit of \( \psi(x) \) is \( \unit{\sqrt{m}} \) and the unit of \( x \) is meter \( \unit{m} \)
	\tcbline
		\item \(  \)
		The normalization condition states that \\
		\begin{flalign*}
			&\uint[-\infty, \infty]{\abs{A\psi(x)}^2}{x} = 1 &&
		\end{flalign*}
		In our case \\
		\begin{flalign*}
			\uint[-\infty, \infty]{\abs{A\psi(x)}^2}{x} &= A^2 \left[ \uint[0, 1]{1}{x} + \uint[1, 2]{4}{x} + \uint[2, 3]{4}{x} \right] = 9A^2 \overset{!}{=} 1 &&
		\end{flalign*}
		\( \Rightarrow A = \tfrac{1}{3} \)
	\tcbline
		\item \(  \)
		\begin{flalign*}
			P &= \uint[2, 3]{\frac{4}{9}}{x} = \dl{\frac{4}{9}} &&
		\end{flalign*}
	\tcbline
		\item \(  \)
		\begin{flalign*}
			P &= \uint[2, 3]{\abs{\tfrac{2}{3} \expo[\pi/6][\iu]}^2}{x} = \uint[2, 3]{\frac{4}{9}}{x} = \dl{\frac{4}{9}} &&
		\end{flalign*}
	\end{enumerate}
\end{mybox}

\begin{mybox}{Teilchen im asymmetrischen Potentialtopf}
	\centering \( V(x) = \begin{cases}
		\infty \quad &\text{für } x \leq 0 \\
		-V_0 \quad &\text{für }  0 \le x \le a \\
		0 \quad &\text{für }  a \leq x
	\end{cases} \)
	\tcblower
	\begin{enumerate}
		\item \(  \)
%		\begin{flalign*}
	%			
%		\end{flalign*}
	\tcbline
		\item \(  \)
%		\begin{flalign*}
	%		
%		\end{flalign*}
	\tcbline
		\item \(  \)
%		\begin{flalign*}
		%			
%		\end{flalign*}
	\tcbline
		\item \(  \)
%		\begin{flalign*}
		%			
%		\end{flalign*}
	\tcbline
		\item \(  \)
%		\begin{flalign*}
		%			
%		\end{flalign*}
	\tcbline
		\item \(  \)
%		\begin{flalign*}
		%			
%		\end{flalign*}
\tcbline
		\item \(  \)
%		\begin{flalign*}
	%			
	%		\end{flalign*}
	\end{enumerate}
\end{mybox}

\begin{mybox}{Teilchen an einem Potentialabfall}
	\centering \( V(x) = \begin{cases}
		V_0 \quad &\text{für } x \leq 0 \\
		0 \quad &\text{für }  x > 0 \\
	\end{cases} \)
	\tcblower
	\begin{enumerate}
		\item \(  \)
%		\begin{flalign*}
		%			
%		\end{flalign*}
	\tcbline
		\item \(  \)
%		\begin{flalign*}
	%		
%		\end{flalign*}
	\tcbline
		\item \(  \)
%		\begin{flalign*}
		%			
%		\end{flalign*}
	\tcbline
		\item \(  \)
%		\begin{flalign*}
	%			
%		\end{flalign*}
	\tcbline
		\item \(  \)
%		\begin{flalign*}
	%			
%		\end{flalign*}
	\tcbline
		\item \(  \)
%		\begin{flalign*}
	%			
%		\end{flalign*}
	\end{enumerate}
\end{mybox}


\end{document}