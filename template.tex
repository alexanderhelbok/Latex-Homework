\documentclass{alex_hü}

\name{Alexander Helbok}
\course{PS Physik}
\hwnumber{7}


\begin{document}
\renewcommand{\labelenumi}{\alph{enumi})}



\begin{mybox}{193. Gekoppelte physikalische Pendel}
	\centering \( m = 5 \unit{kg};\quad k = 2 \unit{N/m};\quad a = 0.1 \unit{m};\quad h = 0.4 \unit{m};\quad l = 0.1 \unit{m} \)
	\tcblower
	\begin{enumerate}
		\item \( \omega = \sqrt{\tfrac{mgl}{I}} \)
		\begin{flalign*}
			I &= \tfrac{1}{12}m(a^2 + h^2) + ml^2 = 0.12 \unit{kg.m^2} &&\\
			\omega_a &= \sqrt{\tfrac{mgl}{I}} = \dl{6.37 \unit{rad/s}} &&
		\end{flalign*}
	\Sepline
		\item \( F = -m\omega^2 x;\quad \Delta x = x_1 - x_2 \)
		\begin{flalign*}
			m\ddot{x_1} &= -m\omega_a\!^2x_1 - k(x_1 - x_2) &&\\
			m\ddot{x_2} &= -m\omega_a\!^2x_2 - k(x_2 - x_1) &&\\
			\Rightarrow\ m(\ddot{x_1}& - \ddot{x_2}) = -m\omega_a\!^2(x_1 - x_2) - 2k(x_1 - x_2) = -(m\omega_a\!^2 + 2k)(x_1 - x_2) &&\\
			\Delta \ddot{x} &= -\tikzmark{Startw} (\omega_a\!^2 + \tfrac{2k}{m}) \tikzmark{Endw} \Delta x &&\\[3.5ex]
			\omega_b &= \sqrt{\omega_a^2 + \tfrac{2k}{m}} = \dl{6.43 \unit{rad/s}} &&
		\end{flalign*}
		\AddUnderBrace[line width=0.75pt]{Startw}{Endw}{$= \omega_b$}
	\Sepline
		\item \( \delta\omega = \omega_b - \omega_a \)
		\begin{flalign*}
			0 &= \cos(\tfrac{1}{2}\delta\omega t) &&\\
			t &= \tfrac{2\arccos(0)}{\delta\omega} = \dl{50.29 \unit{s}} &&
		\end{flalign*}
	\end{enumerate}
\end{mybox}

\begin{testcolors}[HTML]
	\testcolor{yellow!75!black}
	\testcolor{yellow!50!black}
	\testcolor{yellow!5!white}
\end{testcolors}


\end{document}