% !TeX root = Bericht.tex
% !TeX spellcheck = de_AT
\section{Diskussion}
Das zu Beginn eingeführte Dilemma wurde von Martin Rees gelöst, indem die Helligkeitsfluktuationen nicht durch intrinsische physikalische Prozesse hervorgerufen werden, sondern durch rasch ausdehnende strahlende Oberflächen. Falls die Ausbreitungsgeschwindigkeit nahe der Lichtgeschwindigkeit ist, muss die spezielle Relativitätstheorie berücksichtigt werden, die eine Verzerrung des Bildes verursacht. Dieser Effekt reicht aus, um vorhandene Beobachtungen zu erklären.

Das Modell löst aber auch weitere Probleme. Einerseits stimmt die Abschätzung des Alters von Quasaren über dem Abstand der Materieausstöße nicht mehr, da die beobachtete Entfernung nicht der tatsächlichen entspricht. Andererseits liefert es eine Erklärung für messbare Überlichtgeschwindigkeit. Letzteres war in den 60er Jahren noch nicht aufgekommen, wurde aber in darauf folgenden Jahren beobachtet.
