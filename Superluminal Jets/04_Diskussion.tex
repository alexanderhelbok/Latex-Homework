% !TeX root = Bericht.tex
% !TeX spellcheck = de_DE
\section{Diskussion}
Es wurden insgesamt 90 Datenpunkte aufgenommen, welche jedoch in drei getrennten Gruppen analysiert werden mussten. Das ist zwar besser als nur 30 Messdaten von einem Versuchsteilnehmer zu haben, da dann drei Teilmessungen der Elementarladung kombiniert werden können und somit der statistische Fehler ungefähr halbiert wird. Allerdings wären 90 Messdaten von einem Versuchsteilnehmer besser, da dann mehr Daten in den einzelnen Ladungsgruppen liegen und damit die Kernels im KDE kleiner gewählt werden können, was den statistischen Fehler (aus der Breite der KDE) verringert. 

Wir erhalten für die Elementarladung ohne Cunningham Korrektur \( e_{\text{unkorr}} = 1.8(3) \cdot 10^{-19} \unit{C} \). Vergleicht man das mit dem CODATA 2018 Wert \( e_{\text{lit}} = 1.602 \cdot 10^{-19} \unit{C} \) \cite{codata}, erkennt man dass der Literaturwert innerhalb einer Standardabweichung von unserem Wert liegt. 

Die Cunningham-Korrektur kann nur sinnvoll auf die Daten angewendet werden, wenn man für den Parameter \( b \) einen Literaturwert hernimmt. Für die Bestimmung über eine Geradenanpassung sind die Fehler zu klein abgeschätzt und weiters zu wenige Datenpunkte vorhanden, um einen klaren linearen Trend ausmachen und quantifizieren zu können. Als korrigierten Wert für die Elementarladung erhalten wir \( e_{\text{korr}} = 1.6(3) \cdot 10^{-19} \unit{C} \). Die Korrektur hat den Fehler unmerklich beeinflusst, der nominelle Wert hingegen ist etwas kleiner geworden. Die korrigierte Elementarladung ist hier wieder mit dem Literaturwert vereinbar.

Die Unsicherheit der hier bestimmten Werte setzt sich zum Großteil aus der Streuung der Messwerte in den einzelnen Ladungsgruppen zusammen. Um den Fehler zu minimieren müsste man in einem Folgeexperiment die Zeitmessungen akkurater durchführen, zum Beispiel unter Verwendung einer Digitalkamera, die die Tröpfchenbewegung aufnimmt. Da könnte man dann im Nachhinein die Zeitmessung entweder automatisieren oder über eine verlangsamte Aufnahme händisch durchführen.
