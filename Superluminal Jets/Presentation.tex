% !TeX document-id = {7b7d88a4-0972-4202-9c89-33327d1050f5}
% !TeX spellcheck = en_EN
\documentclass[10pt, dvipsnames]{beamer}
\usepackage{appendixnumberbeamer}
\usepackage[T1]{fontenc}
\usepackage[utf8]{inputenc}
\usepackage{hyperref}
\usepackage{fontawesome5, lipsum}
\usepackage{graphicx}
\usepackage[ngerman]{babel}
\usepackage{setspace, multicol, makecell, booktabs, multirow}
\usepackage{mathtools, amsthm, amsfonts, siunitx, physics, chemformula, empheq}
\usepackage{mtpro2, pifont, custom}
\pdfmapfile{=mtpro2.map}
\usepackage[backend=biber, abbreviate=true, doi=false, style=numeric-comp, giveninits=true, sorting=none]{biblatex}
\usepackage{enumitem, csquotes, xurl}
\usepackage[most]{tcolorbox}
\usepackage{multimedia, tikz}
\usepackage[font=footnotesize, bf, format=plain, font=normalsize]{caption}

\usetikzlibrary{math, arrows.meta, calc, angles, quotes}

\tikzset{
	partial ellipse/.style args={#1:#2:#3}{
		insert path={+ (#1:#3) arc (#1:#2:#3)}
	}
}

\graphicspath{ {./bilder/} }
\addbibresource{Literatur.bib}

%\catcode`_=\active
%\newcommand_[1]{\ensuremath{\sb{\fontfamily{Times New Roman}\selectfont \mathrm{#1}}}}

\AtBeginDocument{\sisetup{per-mode = symbol, sticky-per, mode=text, text-font-command=\fontfamily{Times New Roman}\selectfont}}
\let\oldunit\unit
\renewcommand{\unit}[1]{\hspace{4pt}\oldunit{#1}}
\DeclareSIUnit\year{yr}

\newcommand{\nair}{n\sb{\fontfamily{Times New Roman}\selectfont \mathrm{air}}}

\usepackage{scalerel,stackengine}
\newcommand\equalhat{\mathrel{\stackon[1.5pt]{=}{\stretchto{%
				\scalerel*[\widthof{=}]{\wedge}{\rule{1ex}{3ex}}}{0.5ex}}}}

% ------------------------------------------------------------------------------
% Use the beautiful metropolis beamer template
% ------------------------------------------------------------------------------
%\usepackage{FiraSans} 
\mode<presentation>
{
	\usetheme[progressbar=frametitle,background=light]{metropolis} 
	\usecolortheme{default} % or try albatross, beaver, crane, ...
	\usefonttheme[onlymath]{serif}  % or try serif, structurebold, ...
	\setbeamertemplate{navigation symbols}{}
	\setbeamertemplate{caption}[numbered]
	\setbeamertemplate{section in toc}[sections numbered]
	\setbeamerfont{frametitle}{size=\LARGE}
	\metroset{block=fill}
} 

\makeatletter
\setlength{\metropolis@progressinheadfoot@linewidth}{3pt}
\setlength{\metropolis@titleseparator@linewidth}{3pt}
\setlength{\metropolis@progressonsectionpage@linewidth}{3pt}

\makeatother

% ------------------------------------------------------------------------------
% tcolorbox / tcblisting
% ------------------------------------------------------------------------------	
\definecolor{MyBlue}{HTML}{0072B2}
\definecolor{MyOrange}{HTML}{D55E00}
\definecolor{MyRed}{HTML}{F00F0F}
\definecolor{bg}{HTML}{FBFBFB}

\newcommand{\s}[1]{{\color{MyBlue}#1}}

\setstretch{1.5}

\newtcolorbox{quotebox}{enhanced, colframe=MyOrange, top=0.2cm, bottom=0.2cm, colback=orange!15, left=0.1cm}

\title{Überlichtgeschwindigkeit in kosmischen Jets}
\author{Alexander Helbok}
\date{\today}

\begin{document}
	\maketitle
%	\begin{frame}[noframenumbering, plain]{Overview}
%		\tableofcontents
%	\end{frame}

%	\begin{frame}{Timeline}
%		\begin{tikzpicture}[scale=2, every node/.style={scale=0.8, black}]
%			\draw[line width=1.5pt] (0, 0) -- (5, 0);
%			\foreach \i in {0,...,5}{
%			\draw[MyBlue, fill](\i,0) circle (.045) node [below, yshift=-0.2cm] {\Large\( 19\i0 \)};}
%			
%			\draw[MyBlue, line width=1pt] (0.5, 0) -- (0.5, 0.25) node [above left, xshift=0.5cm] {SRT (1905)};
%			
%			\draw[MyBlue, line width=1pt] (1.5, 0) -- (1.5, .75) node [above left, xshift=0.2cm] {ART (1915)};
%			
%			\draw[MyBlue, line width=1pt] (1.7, 0) -- (1.7, 1.25) node [above left, align=left] {Einstein/de Sitter\\ Universum (1917)};
%			
%			\draw[MyBlue, line width=1pt] (1.9, 0) -- (1.9, 2) node [above left, align=left] {Eddington: Beweis über\\ Lichtumlenkung (1919)};
%			
%			\draw[MyBlue, line width=1pt] (2.2, 0) -- (2.2, 2.5) node [above right, xshift=-0.4cm] {Friedmann's Gleichungen (1922)};
%			
%			\draw[MyBlue, line width=1pt] (2.7, 0) -- (2.7, 1.5) node [above right, align=left, xshift=-0.4cm] {Lema\^{i}tres expandierendes\\ Universum (1927)};
%			
%			\draw[MyOrange, line width=1pt] (2.9, 0) -- (2.9, 0.8) node [above right, align=left, xshift=-0.5cm] {Hubble Messungen\\ (1929, 1931, 1934)};
%			\draw[MyOrange, line width=1pt] (3.1, 0) -- (3.1, 0.8);
%			\draw[MyOrange, line width=1pt] (3.4, 0) -- (3.4, 0.8);
%			
%			\draw[MyRed, line width=1pt] (4.8, 0) -- (4.8, 0.5) node [above] {CMBR postuliert (1948)};
%		\end{tikzpicture}
%	\end{frame}

	\begin{frame}{3c273}
		\centering
		\begin{tikzpicture}
			\node[anchor=south west,inner sep=0] (image) at (0,0) {\includegraphics[scale=0.3]{3c273new}};
			
			\begin{scope}[x={(image.south east)},y={(image.north west)}]
				\node[] at (0.15, -0.02) {\tiny© ESA/HUBBLE \& NASA};
			\end{scope}
		\end{tikzpicture}
	\end{frame}

	\begin{frame}{Helligkeitsschwankungen in 3c273}
		\begin{columns}
		\begin{column}{0.5\textwidth}
			\vspace{-0.35cm}
			\begin{tikzpicture}
				\node[anchor=south west,inner sep=0] (image) at (0,0) {\includegraphics[scale=0.33]{oldflux-}};
						
				\begin{scope}[x={(image.south east)},y={(image.north west)}]
					\node[] at (0.35, 0.925) {\tiny(\citeauthor{oldflux} \citeyear{oldflux})};
				\end{scope}
			\end{tikzpicture}
		\end{column}
		\begin{column}{0.5\textwidth}
			\hspace{-0.3cm}\vspace{-0.25cm}
			\begin{tikzpicture}
				\node[anchor=south west,inner sep=0] (image) at (0,0) {\includegraphics[scale=0.19]{newflux}};
					
				\begin{scope}[x={(image.south east)},y={(image.north west)}]
					\node[align=left] at (0.3, 0.925) {\tiny(\citeauthor{newflux} \citeyear{newflux})};
				\end{scope}
			\end{tikzpicture}
		\end{column}
		\end{columns}
	\end{frame}
	
	\begin{frame}{Problem}
		content...
	\end{frame}
	
%	########### Theorie #############
	\begin{frame}{Bild}
		\begin{tikzpicture}[scale=1.4]
			\coordinate (O) at (6,1.5);
			\coordinate (S) at (0,1.5);
			\coordinate (U) at (0,3);
			
			\filldraw[ball color=white] (S) circle (1.5);
			
			\draw[dashed,color=darkgray] (0,0) arc (-90:90:0.5 and 1.5);% right half of the left ellipse
			\draw[thick] (0,0) -- (O);% bottom line
			\draw[thick] (U) -- (O);% top line
			\draw[thick] (0,0) arc (270:90:0.5 and 1.5);% left half of the left ellipse
			
			\draw[fill] (S) circle (0.05) node [below, yshift=-2] {\( S \)};
			\draw[fill] (O) circle (0.05) node [above, yshift=2] {\( O \)};
			
			\draw[semithick, dashed] (O) -- (S) node [above, pos=0.6] {\( R \)};
			\draw[semithick, dashed] (S) -- (U) node [left, pos=0.5] {\( r \)}; 
			
			\pic [draw, angle radius=3cm] {angle = U--O--S};
			\pic [draw, angle radius=0.25cm, thick] {right angle = O--S--U};
			
			\draw (4.5, 1.7) node {\( \theta \)};
		\end{tikzpicture}	
		\vspace{-1cm}\hspace{4cm}
		\begin{minipage}{.5\textwidth}
			\begin{quotebox}
				\( r \ll R,\ \tan(\theta) \approx \theta \)\\
				\( \sphericalangle = 2\theta = 2\tfrac{r}{R} \)
			\end{quotebox}
%			
		\end{minipage}
	\end{frame}
	
	\begin{frame}{Analyse}
		\only<1>{%
		\begin{tikzpicture}[every node/.style={scale=0.9}, scale=1.4]
		%		\node[inner sep=0pt] (russell) at (0,0.05){\includegraphics[scale=0.25]{bilder/aberration2}};
		
		%		\draw [step=.1, gray,thin, opacity=0.5] (-4,-2) grid (4,2);
		%		\draw [step=1, black, opacity=0.5] (-4,-2) grid (4,2);
			
			\coordinate (S) at (-2.5, 0);
			
			\draw[-latex] (S) -- (-3.5, 0);
			
			\draw[-latex] (S) -- (-2.5+0.7,  0.7);
			\draw[-latex] (S) -- (-2.5+0.7, -0.7);
			
			\draw[-latex] (S) -- (-2.5-0.7,  0.7);
			\draw[-latex] (S) -- (-2.5-0.7, -0.7);
			
			\draw[-latex] (S) -- (-2.5, 1);
			\draw[-latex] (S) -- (-2.5, -1);
			
			
			\draw[fill=bg] (S) circle (0.6);
			
			\draw[MyBlue, line cap=round, thick] (S) -- (-2.5, 0.6) node [right, pos=0.55] {\( v t \)};
			
			\draw (S) circle (0.6);
			
			\draw[ultra thick, MyOrange, opacity=0.75, line cap=round] (-2.5, -0.6) arc (-90:90:0.6);
			
			\draw[fill] (S) circle (0.05cm) node [below, yshift=-1] {\( S \)};
			
			\draw[-latex] (S) -- (3.5, 0) node [above left, pos=0.975] {to \( O \)};
		\end{tikzpicture}}
		\only<2>{%
		\begin{tikzpicture}[every node/.style={scale=0.9}, scale=1.4]
	%		\node[inner sep=0pt] (russell) at (0,0.05){\includegraphics[scale=0.25]{bilder/aberration2}};
	
	%		\draw [step=.1, gray,thin, opacity=0.5] (-4,-2) grid (4,2);
	%		\draw [step=1, black, opacity=0.5] (-4,-2) grid (4,2);
			
			\coordinate (S) at (-2.5, 0);
			
			\draw[-latex] (S) -- (-3.5, 0);
			
			\draw[-latex] (S) -- (-3.2, 0.7);
			\draw[-latex] (S) -- (-3.2, -0.7);
			
			\draw[-latex] (S) -- (-2.5, 1.2);
			\draw[-latex] (S) -- (-2.5, -1.2);
			
			\draw[-latex] (S) -- (-1, 1.5);
			\draw[-latex] (S) -- (-1, -1.5);
			
			\draw[-latex] (S) -- (0.9, 1.7);
			\draw[-latex] (S) -- (0.9, -1.7);
			
			\draw[-latex] (S) -- (2.4, 1);
			\draw[-latex] (S) -- (2.4, -1);
			
			
			\draw[fill=bg] (-0.25,0) ellipse (2.7 and 1.1);
			
			\draw[MyBlue, line cap=round, thick] (S) -- (-2.5, 0.6) node [right, pos=0.55] {\( v t \)};
			\draw[MyRed, thick] (-0.25, 0) -- (-0.25, 1.1) node [left, pos=0.5] {\( \gamma v t \)};
			
			\draw (-0.25,0) ellipse (2.7 and 1.1);
			
			\draw[ultra thick, MyOrange, opacity=0.75, line cap=round] (-0.25,0) [partial ellipse=-90:90:2.7 and 1.1];
			
			\draw[fill] (S) circle (0.05cm) node [below, yshift=-1] {\( S \)};
			
			\draw[-latex] (S) -- (3.5, 0) node [above left, pos=0.975] {to \( O \)};
		\end{tikzpicture}}
	\end{frame}
	
%%	########## ergebnisse ##########
%	\begin{frame}{Ergebnisse}
%		\begin{tikzpicture}
%			\begin{scope}[xshift=1.5cm]
%				\node[anchor=south west,inner sep=0] (image) at (0,0) {\includegraphics[width=\textwidth]{HubbleLaw}};
%				\begin{scope}[x={(image.south east)},y={(image.north west)}]
%					\begin{onlyenv}<2>
%						\fill[Gray, opacity=0.3, rounded corners] (0.205, 0.665) rectangle (0.5, 0.97);
%						\draw[MyOrange,ultra thick,rounded corners] (0.2,0.15) rectangle (0.31,0.3);
%						\draw[MyOrange,ultra thick,rounded corners] (0.335,0.43) rectangle (0.37,0.49);
%						
%						\draw[rotate around={-60:(0.575,0.55)},MyBlue, ultra thick] (0.575,0.55) ellipse (40pt and 95pt);
%	%					\draw[red, ultra thick, line cap=round] (0.345,0.445) -- (0.36,0.475);
%	%					\draw[red, ultra thick, line cap=round] (0.345,0.475) -- (0.36,0.445);
%						
%						\draw[MyRed,ultra thick,rounded corners] (0.86,0.5) rectangle (0.9,0.9);
%						
%%						Legend
%						\begin{scope}[shift={(-0.48, 0.6)}]							
%							\draw[MyOrange,ultra thick,rounded corners] (0.7,0.3) rectangle (0.73,0.35) node [black, right, yshift=-0.18cm] {Cepheiden};
%							\draw[MyBlue,ultra thick,rounded corners] (0.7,0.22) rectangle (0.73,0.27) node [black, right, yshift=-0.18cm] {Standardkerzen};
%							\draw[MyRed,ultra thick,rounded corners] (0.7,0.14) rectangle (0.73,0.19) node [black, right, yshift=-0.18cm] {Virgo};
%							\draw[ultra thick, line cap=round] (0.7, 0.1) -- (0.73, 0.1) node [right] {Fit};
%						\end{scope}
%					\end{onlyenv}
%					\begin{onlyenv}<3>
%						\fill[Gray, opacity=0.3, rounded corners] (0.205, 0.72) rectangle (0.5, 0.89);
%%						Legend
%						\begin{scope}[shift={(-0.48, 0.6)}]
%							\draw[line width=1.25pt] (0.715,0.24) circle (2 pt) node [right, xshift=0.18cm] {Gebinnte Daten};
%							\draw[ultra thick, line cap=round, dashed] (0.7, 0.16) -- (0.73, 0.16) node [right] {Fit};
%						\end{scope}
%					\end{onlyenv}
%					\begin{onlyenv}<4>
%						\fill[Gray, opacity=0.3, rounded corners] (0.205, 0.805) rectangle (0.53, 0.88);
%						\draw[MyRed,ultra thick, rounded corners] (0.65,0.62) rectangle (0.7,0.69);
%%						Legend
%						\begin{scope}[shift={(-0.48, 0.6)}]
%							\draw[line width=1.25pt, line cap=round] (0.7, 0.24) -- (0.73, 0.24) node [right, xshift=0.05cm] {Gemittelte Daten};
%							\draw[line width=1.25pt, line cap=round] (0.715, 0.22) -- (0.715, 0.26);
%						\end{scope}
%					\end{onlyenv}
%					
%					\node[] at (0.9, -0.05) {(\citeauthor{Hubble1929} \citeyear{Hubble1929})};
%%					\foreach \i in {1,...,10}{
%%					\draw[black] (0.\i, 0) -- (0.\i, 1);
%%					\draw[black] (0, 0.\i) -- (1, 0.\i);}
%				\end{scope}
%			\end{scope}
%		\end{tikzpicture}
%	\end{frame}
%	
%%	######### 1st Interpretation ########
%	\begin{frame}{Intrepretation von Hubble}
%		\vspace{0.5cm}
%		\begin{minipage}[b][.425\textheight][t]{\textwidth}
%		\begin{quotebox}
%			\begin{itemize}
%				\only<1>{\item[1)] \blockcquote{Hubble1929}{The results establish a roughly linear relation between velocities and distances among nebulae ...}\\
%				Mit Faktor \( K = 465(50) \text{ bzw. } 513(60) \unit{km/s/Mpc} \)}
%				\only<2>{\item[2)] \blockcquote{Hubble1929}{New data to be expected in the near future may modify the significance of the present investigation or, if confirmatory, will lead to a solution having many times the weight.}}
%				\only<3>{\item[3)] \blockcquote{Hubble1929}{The outstanding feature, however, is the possibility that the velocity-distance relation may represent the de Sitter effect, ...}}
%			\end{itemize}
%		\end{quotebox}
%	\end{minipage}
%	\begin{minipage}[b][.5\textheight][t]{\textwidth}
%			\begin{itemize}[label={\textendash}]
%				\only<1>{
%					\item linearer Zusammenhang deutlich sichtbar
%					\item keine statistischen Fehler
%					\item dafür Entfernungen systematisch zu klein (Faktor 7!)
%					\item keine Fitgüte im Paper}
%				\only<2>{
%					\item Absolut Richtig
%					\item viele Nachfolgeexperimente (bis heute noch)
%					\item \enquote{crisis in cosmology}}
%				\only<3>{
%					\item deSitter Universum: statisch, keine (normale) Materie
%					\item Redshift durch Zeitverlangsamung bei großen Entfernungen \cite{deSitter}
%					\item \enquote{richtige} Interpretation: dynamisches Universum (nach Friedmann/Lema\^{i}tre)}
%			\end{itemize}
%		\end{minipage}
%	\end{frame}
%	
%%	######### folgeexperimente #########
%	\begin{frame}{Messung über SN1a}
%		\centering
%		\begin{tikzpicture}
%			\node[anchor=south west,inner sep=0] (image) at (0,0) {\includegraphics[height=0.9\textheight]{Sn1ahubble}};	
%			
%			\begin{scope}[x={(image.south east)},y={(image.north west)}]
%
%				\only<1>{\draw[line width=1pt, opacity=0] (0.09, 0.355) -- (1, 0.89);}
%				\only<2>{\draw[line width=1pt] (0.09, 0.355) -- (1, 0.89);}
%				\node[] at (0.75, 0.02) {(\citeauthor{betoule2014improved} \citeyear{betoule2014improved})};
%%				\draw[fill] (0.5, 0.59) circle (.005) node [below] {\( \approx 1.4 \unit{\giga\year} \)};
%			\end{scope}
%		\end{tikzpicture}
%	\end{frame}
%	
%	\begin{frame}{Inkonsistenz in \( H_0 \)?}
%		\centering	
%		\begin{tikzpicture}
%			\node[anchor=south west,inner sep=0] (image) at (0,0) {\includegraphics[height=0.9\textheight]{crisis}};
%			
%			\begin{scope}[x={(image.south east)},y={(image.north west)}]
%				\node[] at (0.15, 0.02) {(\citeauthor{Di_Valentino_2021} \citeyear{Di_Valentino_2021})};
%			\end{scope}
%		\end{tikzpicture}
%	\end{frame}
%%	######### Interpretation
%
%% 	######### Recap #########
%	\begin{frame}{Recap}
%		\begin{itemize}[label={\textendash}, itemindent=0.1cm]
%			\item {\color{MyBlue}Entferungsbestimmung} über Leuchtkraftrelationenen
%			\item {\color{MyOrange}Geschwindigkeit} über Doppler shift
%			\item[\( \Rightarrow \)] linearer Zusammenhang (obwohl systematische Abweichung)
%			\item \( H_0 \) erlaubt Vergleich Modell, Experiment
%		\end{itemize}
%	\end{frame}

\begin{frame}[standout]
	\Huge Fragen?
\end{frame}

\appendix

%\begin{frame}{Frage 1}
%	Warum habe ich \( H_0 \) immer Hubble Parameter und nicht Hubble Konstante genannt (wie man es in vielen Publikationen sieht)?
%	\begin{itemize}
%		\item[A)] Die beiden Begriffe sind ja Synonyme
%		\only<1>{\item[B)] Naja, \( H_0 \) ist ja zeitlich nicht konstant geblieben}
%		\only<2>{{\color{Green}\item[B)] Naja, \( H_0 \) ist ja zeitlich nicht konstant geblieben}}
%		\item[C)] Man kann \( H_0 \) nicht aus Naturkonstanten ableiten, also ist es selber keine Konstante
%	\end{itemize}
%\end{frame}
%
%\begin{frame}{Frage 2}
%	%	Wäre Hubble zum gleichen Schluss gekommen, wären die Fehler statistisch und nicht systematisch?	
%	Wenn jetzt alle Galaxien von uns wegfliegen, heißt das, dass wir uns im Zentrum des Universums befinden wo der Urknall stattfand?
%	\begin{itemize}
%		\item[A)] Ja, weil ... 
%		\only<1>{\item[B)] Nein, weil ...}
%		\only<2-3>{{\color{Green}\item[B)] Nein, weil ...}}
%	\end{itemize}
%	\hspace{5cm}\begin{minipage}{0.6\textwidth}
%		\vspace{-1.5cm}
%		\begin{onlyenv}<3>
%			\[ \vec{V}' = \vec{V} - \vec{V}_s = H_0 (\vec{r} - \vec{r}_s) = H_0 \vec{r}' \]
%			\begin{tikzpicture}
%				\coordinate (O) at (0,0);
%				\coordinate (Vs) at (2,1);
%				\coordinate (V) at (1,1.25);
%				\coordinate (rs) at (4,2);
%				\coordinate (r) at (2,2.5);
%				
%				\draw[ultra thick, -latex] (-0.3,0) -- (5,0) node [below] {\large\( X \)};
%				\draw[ultra thick, -latex] (0,-0.3) -- (0,3) node [left] {\large\( Y \)};
%				\draw[fill] (0,0) circle (0.1) node [below left] {\large\( 0 \)};
%				
%				\draw[thick, -{Latex[length=3mm,width=2mm]}] (O) -- (Vs) node [below, yshift=-0.1cm, pos=0.95] {\( \vec{V}_s \)};
%				\draw[thick, -{Latex[length=3mm,width=2mm]}] (O) -- (V) node [left, xshift=-0.2cm, pos=0.9] {\( \vec{V} \)};
%				\draw[thick, -{Latex[length=3mm,width=2mm]}] (Vs) -- (rs) node [below, yshift=-0.1cm, pos=0.95] {\( \vec{r}_s \)};
%				\draw[thick, -{Latex[length=3mm,width=2mm]}] (V) -- (r) node [left, xshift=-0.2cm, pos=0.9] {\( \vec{r} \)};
%				\draw[thick, -{Latex[length=3mm,width=2mm]}] (Vs) -- (V) node [above, pos=0.3] {\( \vec{V}' \)};
%				\draw[thick, -{Latex[length=3mm,width=2mm]}] (rs) -- (r) node [above, pos=0.5] {\( \vec{r}' \)};
%				
%				\draw[fill, Goldenrod] (rs) circle (0.1) node [above right, black] {\( \mathrm{Sonne} \)};
%			\end{tikzpicture}
%		\end{onlyenv}
%	\end{minipage}
%\end{frame}
%
%\begin{frame}{Doppler-Effekt/Kosmologischer Redshift}
%	\begin{columns}
%		\begin{column}{0.37\textwidth}
%			\vspace{0.5cm}
%			\begin{tikzpicture}
%				\node[anchor=south west,inner sep=0] (image) at (0,0) {\includegraphics[width=\textwidth]{doppler}};
%				
%				\begin{scope}[x={(image.south east)},y={(image.north west)}]
%					\node[] at (0.175, -0.05) {\tiny\url{www.simple.wikipedia.org}};
%				\end{scope}
%			\end{tikzpicture}
%		\end{column}
%		\begin{column}{0.6\textwidth}
%			\begin{tikzpicture}
%				\node[anchor=south west,inner sep=0] (image) at (0,0) {\includegraphics[width=\textwidth]{doppler_effect}};
%				
%				\begin{scope}[x={(image.south east)},y={(image.north west)}]
%					\node[align=left] at (0.5, -0.05) {\tiny\url{www.online-learning-college.com}};
%				\end{scope}
%			\end{tikzpicture}
%			\begin{tikzpicture}
%				\node[anchor=south west,inner sep=0] (image) at (0,0) {\includegraphics[width=\textwidth]{redshift}};
%				
%				\begin{scope}[x={(image.south east)},y={(image.north west)}]
%					\node[align=left] at (0.85, 0.02) {\tiny\url{www.sites.pitt.edu/}};
%				\end{scope}
%			\end{tikzpicture}
%		\end{column}
%	\end{columns}
%\end{frame}
%
%\begin{frame}{Cepheiden Perioden-Leuchtkraft Beziehung}
%	\vspace{1cm}
%	\begin{columns}
%		\begin{column}{0.5\textwidth}
%			\begin{tikzpicture}
%				\node[anchor=south west,inner sep=0] (image) at (0,0) {\includegraphics[width=\textwidth]{cepheid-variables}};
%				
%				\begin{scope}[x={(image.south east)},y={(image.north west)}]
%					\node[] at (0.175, -0.05) {\tiny\url{www.mso.anu.edu.au}};
%				\end{scope}
%			\end{tikzpicture}
%		\end{column}
%		\begin{column}{0.5\textwidth}
%			\begin{tikzpicture}
%				\node[anchor=south west,inner sep=0] (image) at (0,0) {\includegraphics[width=\textwidth]{out}};
%				
%				\begin{scope}[x={(image.south east)},y={(image.north west)}]
%					\node[align=left] at (0.2, 0.02) {\tiny\url{www.atnf.csiro.au}};
%				\end{scope}
%			\end{tikzpicture}
%		\end{column}
%	\end{columns}
%	\begin{flalign*}
%		\qquad \text{Typ I Cepheiden: } M_v &= -2.54 \log\left( P_{\oldunit{days}} \right) - 1.61 &&\\
%		M &= m - 5\log\left( d_{\oldunit{pc}} \right) + 5 &&
%	\end{flalign*}
%\end{frame}
%
%\begin{frame}{SN1a als Standardkerzen}
%	\centering
%	\begin{tikzpicture}
%		\node[anchor=south west,inner sep=0] (image) at (0,0) {\includegraphics[width=0.85\textwidth]{sn1a}};
%		
%		\begin{scope}[x={(image.south east)},y={(image.north west)}]
%			\node[align=left] at (0.85, 0.02) {\tiny\url{www.astronomy.swin.edu.au}};
%		\end{scope}
%	\end{tikzpicture}
%	\begin{tikzpicture}
%		\node[anchor=south west,inner sep=0] (image) at (0,0) {\includegraphics[width=0.85\textwidth]{stretch2}};
%		
%		\begin{scope}[x={(image.south east)},y={(image.north west)}]
%			\node[align=left] at (0.2, 0.02) {\tiny\url{www.johnlucey.webspace.durham.ac.uk}};
%		\end{scope}
%	\end{tikzpicture}
%\end{frame}

\begin{frame}[allowframebreaks]{Sources}
	\printbibliography
\end{frame}

\end{document}