% !TeX root = Bericht.tex
% !TeX spellcheck = de_DE

\section{Grundlagen und Theorie}
Allgemein betrachtet können auf ein Ölteilchen in dem Kondensator vier verschiedene Kräfte wirken. Dazu zählen die Gewichtskraft $F_{\text{g}} = \rho_{\text{Öl}} \cdot V \cdot g$ und die entgegengesetzte Auftriebskraft $F_{\text{A}} = \rho_{\text{Luft}} \cdot V \cdot g$. Die daraus resultierende Kraft berechnet sich dann zu
\begin{equation}
	F_{\text{g, A}} = F_{\text{g}} - F_{\text{A}} = (\rho_{\text{Öl}} - \rho_{\text{Luft}}) \cdot V \cdot g,
\end{equation}
mit den jeweiligen Dichten von Luft \( \rho_{\text{Luft}} \) und Öl \( \rho_{\text{Öl}} \), dem Volumen des Tropfens \( V \) und der Beschleunigung im Gravitationsfeld der Erde \( g \). Darüber hinaus wirkt in einem eingeschalteten Plattenkondensator die elektrische Kraft $F_{\text{el}}$ auf ein geladenes Öltröpfchen. Diese lautet:
\begin{equation}
	F_{\text{el}} = Q \cdot E = Q \cdot \frac{U}{d}
\end{equation}
Dabei steht \( E \) für die Stärke des elektrischen Feldes, \( U \) ist die angelegte Spannung und \( d \) der Abstand zwischen den zwei horizontalen Kondensatorplatten. 
Abschließend lässt sich eine Kraft, die Reibungskraft welche immer der Bewegung des Öltröpfchens entgegen wirkt, erkennen. Unter der Annahme, das Tröpfchen befindet sich in einer laminaren Strömung, wird der Luftwiderstand durch die Stokes'sche-Reibung charakterisiert und sieht wie folgt aus:
\begin{equation}
	F_{\text{R}} =  6 \pi \cdot r \cdot \eta \cdot v 
\end{equation}
Der Koeffizient $ \eta $ steht für die Viskosität der Luft, $ r $ für den Radius des Öltröpfchens und $v$ dessen Geschwindigkeit. 

Während der Versuchsdurchführung werden zwei verschiedene Zustände betrachtet. Zuerst fällt das Tröpfchen, der Kondensator ist ausgeschaltet. Nach einer gewissen Strecke, welche das Tröpfchen im Sinkflug zurückgelegt hat, wird der Kondensator eingeschalten und die elektrische Kraft beginnt zu wirken, das Öltröpfchen fängt an zu steigen. 

Aus physikalischer Sicht wird das Tröpfchen während des Sinkfluges durch die Gewichtskraft beschleunigt, bis diese durch die Reibungskraft kompensiert wird und sich ein Kräftegleichgewicht einstellt. Dieses lautet: 
\begin{equation*}
	F_{\text{g, A}} - F_{\text{R}} = 0 \quad\Leftrightarrow\quad F_{\text{g, A}} = F_{\text{R}}
\end{equation*}
\begin{equation}
	\label{eqn:1_raw}
	(\rho_{\text{Öl}} - \rho_{\text{Luft}}) \cdot V \cdot g = 6 \pi \cdot r \cdot \eta \cdot v_1
\end{equation}
Nimmt man an, die Öltropfen seien sphärisch, ist das Volumen \( V = 4/3 \pi r^3 \) und die einzigen verbleibenden unbekannten in \autoref{eqn:1_raw} sind \( v_1 \) und \( r \). Mithilfe der Messung der Sinkgeschwindigkeit $v_1$ ergibt sich für \( r \)
\begin{equation}
	\label{eqn:1}
	r = \sqrt{\frac{18\eta v_1}{4(\rho_{\text{Öl}} - \rho_{\text{Luft}}) g}}
\end{equation}
Nach einer gewissen Zeit im Sinkflug wird der Kondensator eingeschalten, das Kräfteverhältnis wird um die elektrische Kraft $F_{\text{el}}$ verändert. Ist diese groß genug, dreht sich die Bewegungsrichtung der Öltropfen um, was die Richtung der Reibungskraft invertiert. Das neue Kräftegleichung lässt sich folgend anschreiben: 
\begin{equation*}
	F_{\text{g, A}} + F_{\text{R}} - F_{\text{el}} = 0 \quad\Leftrightarrow\quad F_{g,A} + F_{\text{R}} = F_{\text{el}}
\end{equation*}
\begin{equation}
	(\rho_{\text{Öl}} - \rho_{\text{Luft}}) \cdot V \cdot g + 6 \pi \cdot r \cdot \eta \cdot v_{2} = Q \cdot \frac{U}{d}
\end{equation}
Mithilfe der Messung der Steiggeschwindigkeit $v_2$ und der Berechnung des Radius \( r \) kann die Ladung \( Q \) des steigenden Öltröpfchens bestimmt werden und es ergibt sich
\begin{equation}
	\label{eqn:2}
	Q = \left(\frac{4\pi r^3(\rho_{\text{Öl}} - \rho_{\text{Luft}}) g}{3} + 6\pi\eta rv_2 \right) \frac{d}{U}
\end{equation} 
Als Spezialfall der oberen Gleichung kann wenn die Spannung gerade richtig gewählt ist, dass das Tröpfchen zum Schweben gebracht wird und somit \( v_2 = 0 \) gilt. Damit vereinfacht sich \autoref{eqn:2} zu 
\begin{equation}
	Q = \left(\frac{4\pi r^3(\rho_{\text{Öl}} - \rho_{\text{Luft}}) g}{3} \right) \frac{d}{U}
\end{equation}
Dabei zeigt die Erfahrung, die aus mehreren Versuchsdurchführungen und Berechnungen der Elementarladung \( e \) gezogen werden konnte, dass die experimentell bestimmten Werte etwas zu groß ausfallen. In der Theorie wird das Stokessche-Gesetz für die Beschreibung der Reibungskraft verwendet, jedoch gilt diese Gesetzmäßigkeit nicht mehr exakt für die Größenordnung der Tröpfchenradien zwischen $10^{-3} \unit{mm}$ und $10^{-4} \unit{mm}$. Dies ist durch die Übereinstimmung der Größe der Radien mit der mittleren freien Weglänge der Luftmoleküle zu begründen. Die Korrektur, welche auch schon von Millikan selbst verwendet wurde, kann mathematisch wie folgt definiert werden:
\begin{equation}
	\label{eqn:b}
	Q_k = \frac{Q}{(1+\frac{b}{rp})^{2/3}} \quad\Leftrightarrow\quad Q^{2/3} = Q_{k}^{2/3} \left(1 + \frac{b}{rp}\right)
\end{equation}
mit der korrigierten Ladung \( Q_k \), dem Luftdruck \( p \), dem Radius der Tropfen \( r \) und einem zu bestimmenden Parameter \( b \). Dies kann durch eine lineare Geradengleichung der Form:
\begin{equation}
	y = y_0(1+bx)
\end{equation}
dargestellt werden. Dabei bezeichnet $y_0 = Q_{k}^{2/3} $ den Schnittpunkt der Gerade mit der y-Achse des Koordinatensystems, die Geradensteigung entspricht dem konstanten Parameter \( b \). 
