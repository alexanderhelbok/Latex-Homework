% !TeX root = Bericht.tex
% !TeX spellcheck = de_DE
\section{Einleitung}
\label{sec:einleitung}
In 60 Jahren kann sich das physikalische Weltbild stark verändern. Vor allem wenn man an die erste Hälfte des 20. Jahrhunder denkt, wo ein ganzer Teilbereich mit der Quantenmechanik neu aufgestellt wurde. Dennoch waren Physiker sich der Implikationen von Einsteins  Relativitätstheorie auch nach 60 Jahren noch nicht ganz bewusst. Wie die Welt bei relativistischen Geschwindigkeiten aussieht, kann man nur aus mathematischen Formeln erahnen, so 

So geschah es, dass 1966 (61 Jahre nach Veröffentlichung der speziellen Relativitätstheorie), die starken Helligkeitsschwankungen von entfernten Radioquellen durch ein simples geometrisches Argument erklärt werden konnten. 
Das geometrische Argument trägt aber auch ein scheinbares Paradoxon mit sich, und zwar sich mit Übberlichtgeschwindigket ausdehnende Körper.