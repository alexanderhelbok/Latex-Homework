\documentclass{alex_hü}

\name{Alexander Helbok}
\course{PS Physik}
\hwnumber{11}

\begin{document}
\renewcommand{\labelenumi}{\alph{enumi})}


\section*{193. Gekoppelte physikalische Pendel}

\section*{196. Schallgeschwindigkeit und Elastizitätseigenschaften}
\centering \( \rho = 4500 \unit{kg/m^3};\quad v_{\parallel} = 5050 \unit{\v};\quad v_{\perp} = 3100 \unit{\v}\) \\
	\begin{enumerate}
		\item \( v_{\parallel} =\sqrt{\tfrac{E}{\rho}};\quad v_{\perp} =\sqrt{\tfrac{G}{\rho}} \)
		\begin{flalign*}
			E &= {v_{\parallel}}^{2}\, \rho = \dl{1.15 * 10^{11}\unit{Pa}} &&\\
			G &= v_{\perp}\!^2\, \rho = \dl{4.32 * 10^{10} \unit{Pa}} &&\\
		\end{flalign*}
		\item \( \mu = \tfrac{E}{2G} - 1 \)
		\begin{flalign*}
			\mu &= \tfrac{{v_{\parallel}}^{2}}{2v_{\perp}\!^2} - 1 = \dl{0.33} &&
		\end{flalign*}
		\item \(  \)
		\begin{flalign*}
			\mu &= \dl{\tfrac{{v_{\parallel}}^{2}}{2v_{\perp}\!^2} - 1} &&\\[1.5ex]
		\tfrac{v_{\parallel}}{v_{\perp}} &= \dl{\sqrt{2\mu + 2}} &&
		\end{flalign*}
	\end{enumerate}


\end{document}