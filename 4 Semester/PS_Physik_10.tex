% !TeX spellcheck = de_DE
\documentclass{alex_hü}

\name{Alexander Helbok}
\course{PS Physik}
\hwnumber{10}

\usepackage{biblatex}

\NewDocumentCommand{\n}{}{\ket{\tfrac{1}{2}, -\tfrac{1}{2}}}
\NewDocumentCommand{\p}{}{\ket{\tfrac{1}{2}, \tfrac{1}{2}}}
\NewDocumentCommand{\pn}{}{\ket{1, 0}}
\RenewDocumentCommand{\pm}{}{\ket{1, -1}}
\NewDocumentCommand{\pp}{}{\ket{1, 1}}


\begin{document}
\renewcommand{\labelenumi}{\alph{enumi})}


\begin{mybox}{Erhaltungssätze und das \( \Sigma \)-Teilchen}
	\centering \(  \)
	\tcblower
	\begin{enumerate}
		\item \(  \)
		\begin{tabular}{c@{\extracolsep{0.5cm}} c c c c}
			& Q & L & B & J \\
			\midrule
			\ch{\( n \to p + \pi^- \)} & \( 0/0 \) & \( 0/0 \) & \( 1/1 \) & \( \tfrac{1}{2}/\tfrac{1}{2} \) \\
			\ch{\( p \to \pi^0 + e^+ \)} & \( 1/1 \) & \( 0/1 \) & \( 1/0 \) & \( 1/\tfrac{1}{2} \) \\
			\ch{\( \pi^+ \to e^+ + \nu_e + \gamma \)} & \( 1/1 \) & \( 0/0 \) & \( 1/1 \) & \( 0/2 \) \\
			\ch{\( \nu_e + p \to n + e^+ \)} & \( 1/1 \) & \( 1/1 \) & \( 1/1 \) & \( 1/1 \) \\
			\ch{\( \nu_e + n \to p + e^- \)} & \( 0/0 \) & \( 1/1 \) & \( 1/1 \) & \( 1/1 \) \\
		\end{tabular}
	\tcbline
		\item \ch{\( K^- + p -> \pi^- + \Sigma \)} \\[4ex]
		\begin{tabular}{@{\extracolsep{0.5cm}} c | c c | c}
			& \( K^- + p \) & \( \pi^- \) & \( \Sigma \) \\
			\midrule
			B & 1 & 0 & 1 \\
			S & -1 & 0 & -1 \\
			Q & 0 & -1 & 1
		\end{tabular}
	\end{enumerate}
\end{mybox}

\begin{mybox}{Isospin-Eigenzustände}
	\centering \(  \)
	\tcblower
	\begin{enumerate}
		\item \( p = \p,\quad n = \n,\quad \pi^0 = \pn,\quad \pi^+ = \pi^- = \pp \)
	\tcbline
		\item \(  \)
		\begin{tabular}{m{0.4\textwidth} | m{0.4\textwidth}}
			p & n \\
			\midrule
			{\begin{flalign*}
				\pi^+:\quad \p\otimes\pp &= \ket{T, \tfrac{3}{2}} &&\\
				\pi^0:\quad \p\otimes\pn &= \ket{T, \tfrac{1}{2}} &&\\
				\pi^-:\quad \p\otimes\pm &= \ket{T, -\tfrac{1}{2}} &&
			\end{flalign*}} & 
			{\begin{flalign*}
				\n\otimes\pp &= \ket{T, \tfrac{1}{2}} &&\\
				\n\otimes\pn &= \ket{T, -\tfrac{1}{2}} &&\\
				\n\otimes\pm &= \ket{T, -\tfrac{3}{2}} &&
			\end{flalign*}}
		\end{tabular}
	\tcbline
		\item \(  \)
		\begin{flalign*}
			\ket{\pi^+p} &= \pp\otimes\p 
				= \ket{\tfrac{3}{2}, \tfrac{3}{2}} &&\\
			\ket{\pi^-p} &= \pm\otimes\p 
				= \sqrt{\tfrac{1}{3}}\ket{\tfrac{3}{2}, -\tfrac{1}{2}} - \sqrt{\tfrac{2}{3}}\ket{\tfrac{1}{2}, -\tfrac{1}{2}} &&\\
			\ket{\pi^0n} &= \pn\otimes\n 
				= \sqrt{\tfrac{2}{3}}\ket{\tfrac{3}{2}, -\tfrac{1}{2}} + \sqrt{\tfrac{1}{3}}\ket{\tfrac{1}{2}, -\tfrac{1}{2}} &&
		\end{flalign*}
	\tcbline
		\item \(  \)
		\begin{flalign*}
			\ket{\pi^+p} \otimes \ket{\pi^0} &= \ket{\tfrac{3}{2}, \tfrac{3}{2}} \otimes \pn 
				= \sqrt{\tfrac{2}{5}}\ket{\tfrac{5}{2}, \tfrac{3}{2}} + \sqrt{\tfrac{3}{5}}\ket{\tfrac{3}{2}, \tfrac{3}{2}} &&
		\end{flalign*}
	\end{enumerate}
\end{mybox}

\begin{mybox}{Wirkungsquerschnitt bei der Pion-Proton-Streuung}
	\centering \(  \)
	\tcblower
	\begin{enumerate}
		\item \( \Delta^{++} = \ket{\tfrac{3}{2}, \tfrac{3}{2}},\quad \Delta^+ = \ket{\tfrac{3}{2}, \tfrac{1}{2}},\quad \Delta^0 = \ket{\tfrac{3}{2}, -\tfrac{1}{2}},\quad \Delta^- = \ket{\tfrac{3}{2}, -\tfrac{1}{2}} \)
	\tcbline
		\item \(  \)
		\begin{flalign*}
			\text{Für}\ \pi^+p:&\quad \sigma_{\text{total}} = \sigma_{\text{elastic}} \approx 200 \unit{mb} &&\\
			\text{Für}\ \pi^-p:&\quad \sigma_{\text{total}} \approx 80 \unit{mb};\quad \sigma_{\text{elastic}} \approx 20 \unit{mb} &&
		\end{flalign*}
	\tcbline
		\item \(  \)
		\begin{flalign*}
			\text{\RN{1}}&: \ch{\( \pi^+ + p\) -> \( \Delta^{++} -> \pi^+ + p \)} &&\\
			\text{\RN{2}}&: \ch{\( \pi^+ + p\) -> \( \Delta^{++} -> \pi^+ + p \)} &&\\
			\text{\RN{3}}&: \ch{\( \pi^- + p\) -> \( \Delta^0 -> \pi^- + p \)} &&\\
			\text{\RN{4}}&: \ch{\( \pi^- + p\) -> \( \Delta^0 -> \pi^0 + n \)} &&
		\end{flalign*}
	\tcbline
		\item \(  \)
		\begin{flalign*}
			\text{\RN{1}}&: \ket{\tfrac{3}{2}, \tfrac{3}{2}} 
				\rightarrow \ket{\tfrac{3}{2}, \tfrac{3}{2}} 
				\rightarrow \ket{\tfrac{3}{2}, \tfrac{3}{2}} &&\\
			\text{\RN{2}}&: \ket{\tfrac{3}{2}, \tfrac{3}{2}} 
				\rightarrow \ket{\tfrac{3}{2}, \tfrac{3}{2}} 
				\rightarrow \ket{\tfrac{3}{2}, \tfrac{3}{2}} &&\\
			\text{\RN{3}}&: \sqrt{\tfrac{1}{3}}\ket{\tfrac{3}{2}, -\tfrac{1}{2}} - \sqrt{\tfrac{2}{3}}\ket{\tfrac{1}{2}, -\tfrac{1}{2}} 
				\rightarrow \ket{\tfrac{3}{2}, -\tfrac{1}{2}}
				\rightarrow \sqrt{\tfrac{1}{3}}\ket{\tfrac{3}{2}, -\tfrac{1}{2}} - \sqrt{\tfrac{2}{3}}\ket{\tfrac{1}{2}, -\tfrac{1}{2}} &&\\
			\text{\RN{4}}&: \sqrt{\tfrac{1}{3}}\ket{\tfrac{3}{2}, -\tfrac{1}{2}} - \sqrt{\tfrac{2}{3}}\ket{\tfrac{1}{2}, -\tfrac{1}{2}} 
				\rightarrow \ket{\tfrac{3}{2}, -\tfrac{1}{2}}
				\rightarrow \sqrt{\tfrac{2}{3}}\ket{\tfrac{3}{2}, -\tfrac{1}{2}} + \sqrt{\tfrac{1}{3}}\ket{\tfrac{1}{2}, -\tfrac{1}{2}} &&
		\end{flalign*}
	\tcbline
		\item \( \hat{T}\ket{T, T_z} = T\ket{T, T_z} \)
		\begin{flalign*}
			\sigma_{\text{\RN{1}}} &\propto \abs{\bra{\tfrac{3}{2}, \tfrac{3}{2}} T \ket{\tfrac{3}{2}, \tfrac{3}{2}}}^2
				= \abs{\tfrac{3}{2}\bra{\tfrac{3}{2}, \tfrac{3}{2}}\ket{\tfrac{3}{2}, \tfrac{3}{2}}}^2
				= \tfrac{9}{4} &&\\
			\sigma_{\text{\RN{2}}} &= \sigma_{\text{\RN{1}}}
				\propto \tfrac{9}{4} &&\\
			\sigma_{\text{\RN{3}}} &\propto \abs{\tfrac{1}{3}\bra{\tfrac{3}{2}, -\tfrac{1}{2}} T \ket{\tfrac{3}{2}, -\tfrac{1}{2}} + \tfrac{2}{3}\bra{\tfrac{1}{2}, -\tfrac{1}{2}} T \ket{\tfrac{1}{2}, -\tfrac{1}{2}}}^2  &&\\
				&= \abs{\tfrac{1}{2}\bra{\tfrac{3}{2}, -\tfrac{1}{2}} \ket{\tfrac{3}{2}, -\tfrac{1}{2}} + \tfrac{1}{3}\bra{\tfrac{1}{2}, -\tfrac{1}{2}}\ket{\tfrac{1}{2}, -\tfrac{1}{2}}}^2 
				= \abs{\tfrac{1}{2} + \tfrac{1}{3}}^2 
				= \tfrac{5}{6} &&\\
			\sigma_{\text{\RN{4}}} &\propto \abs{\sqrt{\tfrac{2}{3}}\sqrt{\tfrac{1}{3}}\bra{\tfrac{3}{2}, -\tfrac{1}{2}} T \ket{\tfrac{3}{2}, -\tfrac{1}{2}} - \sqrt{\tfrac{2}{3}}\sqrt{\tfrac{1}{3}}\bra{\tfrac{1}{2}, -\tfrac{1}{2}} T \ket{\tfrac{1}{2}, -\tfrac{1}{2}}}^2  &&\\
			&= \abs{\tfrac{3}{2}\sqrt{\tfrac{2}{9}}\bra{\tfrac{3}{2}, -\tfrac{1}{2}} \ket{\tfrac{3}{2}, -\tfrac{1}{2}} - \tfrac{1}{2}\sqrt{\tfrac{2}{9}}\bra{\tfrac{1}{2}, -\tfrac{1}{2}}\ket{\tfrac{1}{2}, -\tfrac{1}{2}}}^2 
				= \abs{\tfrac{3}{2}\sqrt{\tfrac{2}{9}} - \tfrac{1}{2}\sqrt{\tfrac{2}{9}}}^2 
				= \tfrac{2}{9} &&\\
		\end{flalign*}
	\end{enumerate}
\end{mybox}


\end{document}