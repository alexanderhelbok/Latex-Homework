% !TeX spellcheck = de_DE
\documentclass{alex_hü}
\usepackage{booktabs}

\name{Alexander Helbok}
\course{PS Physik}
\hwnumber{4}


\NewDocumentCommand{\p}{}{\uparrow}
\NewDocumentCommand{\n}{}{\downarrow}

\begin{document}
\renewcommand{\labelenumi}{\alph{enumi})}


\begin{mybox}{Magnetisches Moment von Kernen}
	\centering \(  \)
	\tcblower
	\begin{enumerate}
		\item \( \vec{j} = \vec{l} + \vec{s} \)
		\begin{flalign*}
			\vec{j}^2 &= (\vec{l} + \vec{s})^2 
				= \vec{l}^2 + \vec{s}^2 + 2\vec{l} \cdot \vec{s} = \vec{l}^2 + \vec{s}^2 + 2\vec{l} \cdot (\vec{j} - \vec{l})
				= \vec{s}^2 - \vec{l}^2 + 2\vec{j} \cdot \vec{l} &&\\[2ex]
			\Rightarrow \vec{j} \cdot \vec{l} &= \frac{1}{2}\left( \vec{j}^2 + \vec{l}^2 - \vec{s}^2 \right) &&
		\end{flalign*}
	\tcbline
		\item \(  \)
		\begin{flalign*}
			\expval{g_l \vec{j}^2}{j m_j} &= j(j+1) &&\\[2ex]
			\expval{g_l \vec{j} \cdot \vec{l}}{j m_j} &= \expval{g_l \frac{1}{2}\left( \vec{j}^2 + \vec{l}^2 - \vec{s}^2 \right)}{j m_j} =&&\\
				&= \frac{g_l}{2} \bigg[ j(j+1) + l(l+1) - s(s+1) \bigg] &&\\[2ex]
			\expval{g_s \vec{s} \cdot \vec{j}}{j m_j} &= \expval{g_l \frac{1}{2}\left( \vec{j}^2 + \vec{s}^2 - \vec{l}^2 \right)}{j m_j} =&&\\
				&= \frac{g_s}{2} \bigg[ j(j+1) + s(s+1) - l(l+1) \bigg] &&
		\end{flalign*}
		\begin{flalign*}
			&\expval{g_l \vec{j} \cdot \vec{l} + g_s \vec{s} \cdot \vec{j}}{j m_j} = 
				\expval{g_l \vec{j} \cdot \vec{l}}{j m_j} + \expval{g_s \vec{s} \cdot \vec{j}}{j m_j} = &&\\
			&= \frac{1}{2}\left(g_l \bigg[ j(j+1) + l(l+1) - s(s+1) \bigg] + g_s \bigg[ j(j+1) + s(s+1) - l(l+1) \bigg] \right) &&\\
		\end{flalign*}
		\begin{flalign*}
			g_{\text{Kern}} &= \frac{\expval{g_l \vec{j} \cdot \vec{l} + g_s \vec{s} \cdot \vec{j}}{j m_j}}{\expval{g_l \vec{j}^2}{j m_j}} = &&\\[2ex]
			&= \frac{g_l \big[ j(j+1) + l(l+1) - s(s+1) \big] + g_s \big[ j(j+1) + s(s+1) - l(l+1) \big]}{2j(j+1)} &&
		\end{flalign*}
	\tcbline
		\item \( \Delta_{lsj} := j(j+1) + s(s+1) - l(l+1) \)
		\begin{flalign*}
			g_{\text{Kern}} &= g_{\text{Kern}} + g_l - g_l 
			= g_l + \frac{g_l(-\Delta_{lsj} + 2j(j+1)) + g_s\Delta_{lsj} - g_l2j(j + 1)}{2j(j+1)} &&\\[2ex]
			&= g_l + \frac{g_l(-\Delta_{lsj}) + g_s\Delta_{lsj}}{2j(j+1)}
			= g_l + (g_s - g_l) \frac{\Delta_{lsj}}{2j(j+1)}
		\end{flalign*}
	\tcbline
		\item \( j = l \pm \tfrac{1}{2};\quad s = \tfrac{1}{2};\quad s(s+1) = \tfrac{3}{4};\quad j(j+1) = l^2 + 2l + \tfrac{3}{4} \)
		\begin{flalign*}
			\frac{\Delta_{lsj}}{2j(j+1)} &= \frac{j(j+1) + s(s+1) - l(l+1)}{2j(j+1)}
				= \frac{l^2 + 2l + \tfrac{3}{4} + \tfrac{3}{4} - l^2 - l}{2l^2 + 4l + \tfrac{3}{2}} &&\\
			&= \frac{l + \tfrac{3}{2}}{2l^2 + 4l + \tfrac{3}{2}} 
				= \frac{l + \tfrac{3}{2}}{\left( 2l + 1 \right) \left( l + \tfrac{3}{2} \right)} 
				= \frac{1}{2l + 1} &&\\[4ex]
			\Rightarrow g_{\text{Kern}} &= g_l \pm (g_s - g_l) \frac{\Delta_{lsj}}{2j(j+1)} 
				= g_l \pm \frac{(g_s - g_l)}{2l + 1} &&
		\end{flalign*}
	\tcbline
		\item \( \mu_N = \tfrac{e\hbar}{2m_{\text{p}}} \)
%		\begin{flalign*}
%			
%		\end{flalign*}
	\end{enumerate}
\end{mybox}

\begin{mybox}{Der Isospin des Deuterons}
	\begin{enumerate}
		\item \(  \)
		\begin{flalign*}
			T_z = 1&:\ \psi_{pp} = \ket{\p\p} &&\\
			T_z = 0&:\ \psi_{pn} = \ket{\p\n} &&\\
			T_z = -1&:\ \psi_{nn} = \ket{\n\n} &&
		\end{flalign*}
	\tcbline
		\item \(  \)
		\begin{flalign*}
			\psi(T_z = 1) &= \ket{\p\p} &&\\
			\psi_1(T_z = 0) &= \tfrac{1}{\sqrt{2}}\left( \ket{\p\n} + \ket{\n\p} \right) &&\\
			\psi_2(T_z = 0) &= \tfrac{1}{\sqrt{2}}\left( \ket{\p\n} - \ket{\n\p} \right) &&\\
			\psi(T_z = -1) &= \ket{\n\n} &&
		\end{flalign*}
	\tcbline
		\item \(  \)
		\renewcommand{\arraystretch}{1.5}
		\setlength{\tabcolsep}{0.3cm} 		
		\begin{tabular}{ l l L L }
			Symmetrie & \( \psi \) & T_z & T \\
		\toprule
			Antisymmetrisch & \( \psi = \tfrac{1}{\sqrt{2}}\left( \ket{\p\n} - \ket{\n\p} \right) \) & 0 & 0 \\[1em]
			Symmetrisch & \( \psi = \ket{\p\p} \) & 1 & 1 \\
			 & \( \psi = \tfrac{1}{\sqrt{2}}\left( \ket{\p\n} + \ket{\n\p} \right) \) & 0 & 1 \\
			 & \( \psi = \ket{\n\n} \) & -1 & 1 
		\end{tabular}
	\tcbline
		\item \( \ket{\Psi_{\text{Ort}}} \) is symmetrisch, da \( L = 0 \)
	\tcbline
		\item \( \ket{\Psi_{\text{Spin}}}\ket{\Psi_{\text{Isospin}}} \) muss antisymmetrisch sein, damit \( \ket{\Psi} \) antisymmetrisch ist.
	\tcbline
		\item Da \( S = 1 \) ist die Spin-Wellenfunktion symmetrisch, weshalb die Isospin-Wellenfunktion antisymmetrisch sein muss.
	\tcbline
		\item 
	\end{enumerate}
\end{mybox}

\begin{mybox}{Quadrupolmoment der Kerne}
	\vspace{-0.75cm}\begin{align*}
		x &= ar\sin(\theta)\cos(\varphi) \\
		y &= ar\sin(\theta)\sin(\varphi) \\
		z &= br\cos(\theta) 
	\end{align*}
	\tcblower
	\begin{enumerate}
		\item \( \rho_{\text{el}} = \tfrac{Ze}{V} = \tfrac{3Ze}{4\pi a^2b};\quad \norm{\vec{r}}^2 = r^2 \left(a^2\sin[2](\theta) + b^2\cos[2](\theta) \right) \)
		\begin{flalign*}
			Q &= \uint[V]{\rho_{\text{el}}(\vec{r}) \left[3z^2 - \norm{\vec{r}}^2\right]}{V} 
				= \uint[0,1]{\uint[0,2\pi]{\uint[0,\pi]{\rho_{\text{el}} a^2b r^2\sin(\theta) \left[ 3z^2 - \norm{\vec{r}}^2 \right]}{\theta}}{\varphi}}{r} &&\\
			&= 2\pi\rho_{\text{el}} ab^2 \uint[0,1]{\uint[0,\pi]{r^4\sin(\theta) \left[ -a^2\sin[2](\theta) + 2b^2\cos[2](\theta) \right]}{\theta}}{r} &&\\
			&= 2\pi\rho_{\text{el}} ab^2 \uint[0,1]{r^4}{r}\uint[0,\pi]{\sin(\theta) \left[ -a^2\sin[2](\theta) + 2b^2\cos[2](\theta) \right]}{\theta} &&\\
			&= \frac{2}{5}\pi\rho_{\text{el}} ab^2 \left[ -a^2\uint[0,\pi]{\sin[3](\theta)}{\theta} + 2b^2\uint[0,\pi]{\cos[2](\theta)\sin(\theta)}{\theta} \right] &&\\
			&= \frac{2}{5}\pi\rho_{\text{el}} ab^2 \left[ -a^2\frac{4}{3} + b^2\frac{4}{3} \right] &&\\
			&= \frac{2}{5}(b^2 - a^2) \frac{4\rho_{\text{el}}\pi a b^2}{3} &&\\
			&= \frac{2}{5}Ze(b^2 - a^2) &&
		\end{flalign*}
	\( a \) und \( b \) hier vertauscht (im Vergleich zur Angabe) wegen der Parametrisierung. Können aber durch neue Parametrisierung einfach getauscht werden. Der Übersichtlichkeit halber wird in der nächsten Aufgabe die Notation aus der Angabe übernommen.
	\tcbline
		\item \( Q(\text{Ta}) = 6 \cdot 10^{-24} e \unit{cm^2};\quad Q(\text{Sb}) = -1.2 \cdot 10^{-24} e \unit{cm^2};\quad R = 1.3 \cdot A^{\tfrac{1}{3}} \unit{fm} \)\\
		\begin{minipage}{.2\textwidth}
			\begin{flalign*}
				a &= R(1+\epsilon) &&\\
				b &= \frac{R}{\sqrt{1+\epsilon}} &&
			\end{flalign*}
		\end{minipage}%
		\begin{minipage}{0.15\textwidth}
			\vspace{0.4cm}\fontsize{45}{12}\selectfont$\Rightarrow$
		\end{minipage}%
		\begin{minipage}{.6\textwidth}
			\begin{flalign*} 
				a^2 - b^2 &= (1 + \epsilon)^2 R^2-\frac{R^2}{1 + \epsilon} \approx 3R^2\epsilon &&\\
				\frac{a}{b} &= (1 + \epsilon)^{3/2} &&
			\end{flalign*}
		\end{minipage}
	
		\begin{flalign*}
			Q(\text{Ta}) &= \tfrac{2}{5}Z_{\text{Ta}}e(a^2 - b^2) 
				\approx \tfrac{2}{5}73 e 3R^2\epsilon_{\text{Ta}}
				= \tfrac{438}{5}e R^2\epsilon_{\text{Ta}} &&\\[2ex]
			\Rightarrow \epsilon_{\text{Ta}} &= \tfrac{5}{438}\tfrac{Q(\text{Ta})}{eR^2} 
				= 0.146 &&\\[4ex]
			Q(\text{Sb}) &= \tfrac{2}{5}Z_{\text{Sb}}e(a^2 - b^2) 
				\approx \tfrac{2}{5}51 e 3R^2\epsilon_{\text{Sb}}
				= \tfrac{306}{5}e R^2\epsilon_{\text{Sb}} &&\\[2ex]
			\Rightarrow \epsilon_{\text{Sb}} &= \tfrac{5}{306}\tfrac{Q(\text{Sb})}{eR^2} 
				= -0.053 &&\\[4ex]
			\left.\frac{a}{b}\right\vert_{\epsilon_{\text{Ta}}} &= \dl{1.227} &&\\[2ex]
			\left.\frac{a}{b}\right\vert_{\epsilon_{\text{Sb}}} &= \dl{0.921} &&
		\end{flalign*}
	\end{enumerate}
\end{mybox}


\end{document}