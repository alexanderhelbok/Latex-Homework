% !TeX spellcheck = de_DE
\documentclass{alex_hü}
%\usepackage{braket}

\name{Alexander Helbok}
\course{PS Physik}
\hwnumber{4}


\begin{document}
\renewcommand{\labelenumi}{\alph{enumi})}


\begin{mybox}{Magnetisches Moment von Kernen}
	\centering \(  \)
	\tcblower
	\begin{enumerate}
		\item \( \vec{j} = \vec{l} + \vec{s} \)
		\begin{flalign*}
			\vec{j}^2 &= (\vec{l} + \vec{s})^2 
				= \vec{l}^2 + \vec{s}^2 + 2\vec{l} \cdot \vec{s} = \vec{l}^2 + \vec{s}^2 + 2\vec{l} \cdot (\vec{j} - \vec{l})
				= \vec{s}^2 - \vec{l}^2 + 2\vec{j} \cdot \vec{l} &&\\[2ex]
			\Rightarrow \vec{j} \cdot \vec{l} &= \frac{1}{2}\left( \vec{j}^2 + \vec{l}^2 - \vec{s}^2 \right) &&
		\end{flalign*}
	\tcbline
		\item \(  \)
		\begin{flalign*}
			\expval{g_l \vec{j}^2}{j m_j} &= j(j+1) &&\\[2ex]
			\expval{g_l \vec{j} \cdot \vec{l}}{j m_j} &= \expval{g_l \frac{1}{2}\left( \vec{j}^2 + \vec{l}^2 - \vec{s}^2 \right)}{j m_j} =&&\\
				&= \frac{g_l}{2} \bigg[ j(j+1) + l(l+1) - s(s+1) \bigg] &&\\[2ex]
			\expval{g_s \vec{s} \cdot \vec{j}}{j m_j} &= \expval{g_l \frac{1}{2}\left( \vec{j}^2 + \vec{s}^2 - \vec{l}^2 \right)}{j m_j} =&&\\
				&= \frac{g_s}{2} \bigg[ j(j+1) + s(s+1) - l(l+1) \bigg] &&
		\end{flalign*}
		\begin{flalign*}
			&\expval{g_l \vec{j} \cdot \vec{l} + g_s \vec{s} \cdot \vec{j}}{j m_j} = 
				\expval{g_l \vec{j} \cdot \vec{l}}{j m_j} + \expval{g_s \vec{s} \cdot \vec{j}}{j m_j} = &&\\
			&= \frac{1}{2}\left(g_l \bigg[ j(j+1) + l(l+1) - s(s+1) \bigg] + g_s \bigg[ j(j+1) + s(s+1) - l(l+1) \bigg] \right) &&\\
		\end{flalign*}
		\begin{flalign*}
			g_{\text{Kern}} &= \frac{\expval{g_l \vec{j} \cdot \vec{l} + g_s \vec{s} \cdot \vec{j}}{j m_j}}{\expval{g_l \vec{j}^2}{j m_j}} = &&\\[2ex]
			&= \frac{g_l \big[ j(j+1) + l(l+1) - s(s+1) \big] + g_s \big[ j(j+1) + s(s+1) - l(l+1) \big]}{2j(j+1)} &&
		\end{flalign*}
	\tcbline
		\item \( \Delta_{lsj} := j(j+1) + s(s+1) - l(l+1) \)
		\begin{flalign*}
			g_{\text{Kern}} &= g_{\text{Kern}} + g_l - g_l 
			= g_l + \frac{g_l(-\Delta_{lsj} + 2j(j+1)) + g_s\Delta_{lsj} - g_l2j(j + 1)}{2j(j+1)} &&\\[2ex]
			&= g_l + \frac{g_l(-\Delta_{lsj}) + g_s\Delta_{lsj}}{2j(j+1)}
			= g_l + (g_s - g_l) \frac{\Delta_{lsj}}{2j(j+1)}
		\end{flalign*}
	\tcbline
		\item \( j = l \pm \tfrac{1}{2};\quad s = \tfrac{1}{2};\quad s(s+1) = \tfrac{3}{4};\quad j(j+1) =  \)
		\begin{flalign*}
			g_{\text{Kern}} &= g_l \pm (g_s - g_l)\tfrac{}{} 
		\end{flalign*}
	\tcbline
		\item \(  \)
%		\begin{flalign*}
%			
%		\end{flalign*}
	\end{enumerate}
\end{mybox}

\begin{mybox}{Der Isospin des Deuterons}
	\centering \(  \)
	\tcblower
	\begin{enumerate}
		\item \(  \)
%		\begin{flalign*}
	%			
%		\end{flalign*}
	\tcbline
		\item \(  \)
%		\begin{flalign*}
	%		
%		\end{flalign*}
	\tcbline
		\item \(  \)
%		\begin{flalign*}
		%			
%		\end{flalign*}
	\tcbline
		\item \(  \)
%		\begin{flalign*}
%			
%		\end{flalign*}
	\tcbline
		\item \(  \)
%		\begin{flalign*}
%			
%		\end{flalign*}
	\tcbline
		\item \(  \)
%		\begin{flalign*}
%			
%		\end{flalign*}
	\tcbline
		\item \(  \)
%		\begin{flalign*}
%			
%		\end{flalign*}
	\end{enumerate}
\end{mybox}

\begin{mybox}{Quadrupolmoment der Kerne}
	\vspace{-0.75cm}\begin{align*}
		x &= ar\sin(\theta)\cos(\varphi) \\
		y &= ar\sin(\theta)\sin(\varphi) \\
		z &= br\cos(\theta) 
	\end{align*}
	\tcblower
	\begin{enumerate}
		\item \( \rho_{\text{el}} = \tfrac{Ze}{V} = \tfrac{3Ze}{4\pi a^2b};\quad \norm{\vec{r}}^2 = r^2 \left(a^2\sin(\theta)^2 + b^2\cos(\theta)^2 \right) \)
		\begin{flalign*}
			Q &= \uint[V]{\rho_{\text{el}}(\vec{r}) \left[3z^2 - \norm{\vec{r}}^2\right]}{V} 
				= \uint[0,1]{\uint[0,2\pi]{\uint[0,\pi]{\rho_{\text{el}} a^2b r^2\sin(\theta)(3z^2 - r^2)}{\theta}}{\varphi}}{r} &&\\
			&= 2\pi\rho_{\text{el}} ab^2 \uint[0,1]{\uint[0,\pi]{r^2\sin(\theta)(3r^2\cos(\theta)^2 - r^2)}{\theta}}{r} 
		\end{flalign*}
	\tcbline
		\item \( Q(\text{Ta}) = 6 \cdot 10^{-24} e \unit{cm^2};\quad Q(\text{Sb}) = -1.2 \cdot 10^{-24} e \unit{cm^2}  \)
		\begin{flalign*}
			a &= R(1+\epsilon) &&\\
			b &= \frac{R}{\sqrt{1+\epsilon}} &&
		\end{flalign*}
		\begin{minipage}{0.5\textwidth}
			$\Rightarrow a^2 - b^2 = (1 + \epsilon)^2 R^2-\frac{R^2}{1 + \epsilon} \approx 3R^2\epsilon $ 
		\end{minipage}
	\end{enumerate}
\end{mybox}


\end{document}