\documentclass{alex_hü}

\name{Alexander Helbok}
\course{PS Physik}
\hwnumber{2}


\begin{document}
\renewcommand{\labelenumi}{\alph{enumi})}


\begin{mybox}{Materiewellen und Formfaktoren}
	\centering \(  \)
	\tcblower
	\begin{enumerate}
		\item \(  \)
%		\begin{flalign*}
%			
%		\end{flalign*}
	\tcbline
		\item \(  \)
		\begin{flalign*}
			F(\mathbold{q}^2) &= \uint[\mathbb{R}^3]{\expo[\iu\vec{q}\vec{x} / \hbar] f(\vec{x}) }{\vec{x}} 
				= \uint[0,2\pi]{\uint[0,\pi]{\uint[0,\infty]{\expo[\iu qr\cos(\theta) / \hbar] f(r) r^2 \sin(\theta)}{r}}{\theta}}{\varphi} &&\\
				&= 2\pi \uint[0,\infty]{f(r)r^2 \uint[0,\pi]{\expo[\iu qr\cos(\theta) / \hbar] \sin(\theta) }{\theta}}{r} 
				= 2\pi \uint[0,\infty]{f(r)r^2 \uint[-1,1]{\expo[\iu qru / \hbar] }{u}}{r} &&\\
				&= 4\pi \uint[0,\infty]{f(r)r^2 \left( \expo[\iu qr / \hbar] - \expo[\iu qr / \hbar] \right) \tfrac{\hbar}{2\iu qr}}{r}
				= 4\pi \uint[0,\infty]{\tfrac{\sin(\tfrac{qr}{\hbar})}{\tfrac{qr}{\hbar}} f(r)r^2}{r} &&
		\end{flalign*}
	\tcbline
		\item \(  \)
%		\begin{flalign*}
%			
%		\end{flalign*}
	\tcbline
		\item \( f(r) = f_0\expo[-][ar] \)
%		\begin{flalign*}
%			
%		\end{flalign*}
	\tcbline
		\item \(  \)
%		\begin{flalign*}
%			
%		\end{flalign*}
	\tcbline
		\item \(  \)
%		\begin{flalign*}
%			
%		\end{flalign*}\\
	\tcbline
		\item \(  \)
%		\begin{flalign*}
%			
%		\end{flalign*}
	\end{enumerate}
\end{mybox}

\begin{mybox}{Stabilstes Nuklid einer Isobare}
	\centering \(  \)
	\tcblower
	\begin{enumerate}
		\item \(  \)
%		\begin{flalign*}
	%			
%		\end{flalign*}
	\tcbline
		\item \(  \)
%		\begin{flalign*}
	%		
%		\end{flalign*}
	\tcbline
		\item \(  \)
%		\begin{flalign*}
		%			
%		\end{flalign*}
	\end{enumerate}
\end{mybox}

\begin{mybox}{Luminosität des LHC}
	\centering \(  \)
	\tcblower
	\begin{enumerate}
		\item \(  \)
%		\begin{flalign*}
		%			
%		\end{flalign*}
	\tcbline
		\item \(  \)
%		\begin{flalign*}
	%		
%		\end{flalign*}
	\tcbline
		\item \(  \)
%		\begin{flalign*}
		%			
%		\end{flalign*}
	\end{enumerate}
\end{mybox}


\end{document}