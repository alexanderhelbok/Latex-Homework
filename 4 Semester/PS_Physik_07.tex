% !TeX spellcheck = de_DE
\documentclass{alex_hü}

\name{Alexander Helbok}
\course{PS Physik}
\hwnumber{1}


\begin{document}
\renewcommand{\labelenumi}{\alph{enumi})}


\begin{mybox}{Beta-Zerfälle von Kalium-40}
	\centering \(  \)
	\tcblower
	\begin{enumerate}
		\item \ch{^{40}K ->[\( \beta- \)] ^{40}Ca + e- + \( \overline{\nu}_{\text{e}} \)}
		\begin{flalign*}
			Q &= (M_{\text{K}} - M_{\text{Ca}})c^2
				= \dl{1.31 \unit{MeV}} &&
		\end{flalign*}
	\tcbline
		\item \(  \)
		\begin{flalign*}
			E_{\text{kin}} &= \gamma m_0c^2 - m_0c^2
				\stackrel{!}{=} 1.31 \unit{MeV} &&\\[2ex]
			\gamma &= \tfrac{1.31 \unit{MeV}}{m_0c^2} + 1 
				= \dl{1.944} &&\\[3ex]
			\gamma &= \tfrac{1}{\sqrt{1 - \tfrac{v^2}{c^2}}} &&\\
			v &= c \sqrt{1 - \tfrac{1}{\gamma^2}}
				= \dl{2.57 \cdot 10^{8} \unit{\v}}
				= 0.858 \unit{c} &&
		\end{flalign*}
	\tcbline
		\item \ch{^{40}K ->[\( \beta+ \)] ^{40}Ar + e+ + \( \nu_{\text{e}} \)}
		\begin{flalign*}
			Q &= (M_{\text{K}} - M_{\text{Ar}} - 2M_{\text{e}})c^2
			= \dl{0.48 \unit{MeV}} &&
		\end{flalign*}
		Wenn man aber berücksichtigt, dass das Positron mit dem ungebundenen Hüllenelektron annihiliert werden wieder \( 1.5 \unit{MeV} \) an Energie frei.
	\tcbline
		\item \ch{^{40}K ->[EC] ^{40}Ar^* + \( \nu_{\text{e}} \) -> ^{40}Ar + \( \gamma \) + \( \nu_{\text{e}} \)}
		\begin{flalign*}
			E_{\nu} &= c^2(M_{\text{K}} - M_{\text{Ar}}) - E_{\gamma}
				= \dl{44.4 \unit{keV}} &&
		\end{flalign*}
	\tcbline
		\item \( m = 2.3 \unit{g/kg};\quad M = 70 \unit{kg};\quad p = 0.00012;\quad N_0 = \tfrac{mM}{M_{\text{K}}}p \)
		\begin{flalign*}
			A &= N_0\lambda 
				= \dl{5127.35 \unit{Bq}} &&
		\end{flalign*}
	\tcbline
		\item \(  \)
		\centering
		\begin{tabular}{m{0.4\textwidth} | m{0.4\textwidth}}
			Proton in \( 1d_{3/2} \) & Neutron in \( 1f_{7/2} \) \\
			\midrule
			\( \ell = 2 \) & \( \ell = 3 \) \\
			\( P_{\text{p}} = (-1)^\ell = 1 \) & \( P_{\text{n}} = (-1)^\ell = -1 \) \\
			\( s = \pm \tfrac{1}{2} \) & \( s = \pm \tfrac{1}{2} \) \\
			\( J_{\text{p}} = \ell + s = 2 \pm \tfrac{1}{2} \) & \( J_{\text{n}} = \ell + s = 3 \pm \tfrac{1}{2} \) \\
		\end{tabular}
		\vspace{0.5cm}
		\raggedright
		\vspace{0.5cm}
		\( P_{\text{tot}} = P_{\text{p}}P_{\text{n}} = -1 \)\\[2ex]
		\( J_{\text{tot}} = J_{\text{p}} + J_{\text{n}} = 5 \pm 1 \)\\[2ex]
		\( \Rightarrow \) Mögliche Zustände sind \( 4^-, 5^-, 6^- \)
	\tcbline
		\item Reiner E2 Zerfall, da \( \abs{\Delta J} = 2 \) und \( \Delta P = + \)
	\end{enumerate}
\end{mybox}

\begin{mybox}{Wechselwirkung von Neutronen in Materie}
	\centering \( E_1 = E_0 \left[1 - \tfrac{2mM}{(m + M)^2}\left(1 - \cos(\theta) \right) \right] \)
	\tcblower
	\begin{enumerate}
		\item Für \( \theta = \pi \), ist \( \tfrac{E_1}{E_0} = 1 - \tfrac{4mM}{(m + M)^2} \approx 98 \% \) 
	\tcbline
		\item \( m = 1;\quad M = A;\quad \expval{\cos(\theta)} = 0,\ \theta \in [-\pi, \pi) \)
		\begin{flalign*}
			\expval{\tfrac{\Delta E_n}{E_n}} &= \expval{\tfrac{E_n - E_n'}{E_n}}
				= 1 - \expval{\tfrac{E_n'}{E_n}}
				= 1 - \expval{1 - \frac{2mM}{(m + M)^2}\bigg(1 - \cos(\theta) \bigg)} &&\\[3ex]
			&= 1 - \expval{1 - \frac{2A}{(A + 1)^2}\bigg(1 - \cos(\theta) \bigg)}
				= 1 - \expval{1 - \frac{2A}{(A + 1)^2}} 
				= \dl{\frac{2A}{(A + 1)^2}} &&\\[2ex]
			\expval{\tfrac{E_n'}{E_n}}_{\ch{^{238}U}} &= \expval{1 - \frac{2A}{(A + 1)^2}\bigg(1 - \cos(\theta) \bigg)} 
				= 1 - \frac{476}{239^2} 
				\approx \dl{99 \%} &&
		\end{flalign*}
	\end{enumerate}
\end{mybox}

\begin{mybox}{Kernreaktionen in der Erdatmosphäre und der Sonne}
	\begin{enumerate}
		\item Nachdem bei der \ch{^{14}C} Reaktion ein Proton frei wird (das wie das Neutron auf der rechten Seite keine Bindungsenergie "beansprucht") und das Neutron mehr Masse besitzt als das Proton sollte hier der Q Wert positiv sein, sprich es wird bei der Reaktion Energie frei.
		
		Bei der \ch{^{12}C} + \ch{^{3}H} Reaktion bleibt kein freies Nuklid über. Daher beansprucht hier ein Nuklid mehr als vor der Reaktion Bindungsenergie,w as auf einen negativen Q Wert führt. Damit diese Reaktion abläuft muss man also Energie zuführen.
	\tcbline
		\item “The energy losses to radiation always overcompensate the gains due to the reactions.” Sprich die angeregten Kerne in der Atmosphäre strahlen ihre Energie schnell wieder ab, sodass es zu keiner stabilen (sich erhaltenden) Kernfusion kommen kann.
	\tcbline
		\item \ch{^{16}O} ist ein doppeltmagischer Kern, also ein sehr stabiler mit höherer Bindungsenergie im Vergleich zu seine Nachbarn in der Nuklidkarte. Deshalb wird zuerst Neon fusioniert, bevor es zum Sauerstoffbrennen kommt.
	\end{enumerate}
\end{mybox}


\end{document}