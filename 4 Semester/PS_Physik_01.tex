\documentclass{alex_hü}

\name{Alexander Helbok}
\course{PS Physik}
\hwnumber{1}


\begin{document}
\renewcommand{\labelenumi}{(\alph{enumi})}


\begin{mybox}{Einheiten der Kern- \& Teilchenphysik \& Raumwinkel}
	\centering \(  \)
	\tcblower
	\begin{enumerate}
		\item \( \Delta\phi = 1 \unit{V};\quad q_{\text{e}} = e \)
		\begin{flalign*}
			\Delta E &= q_{\text{e}}\Delta\phi = 1.602 \times 10^{-19} \unit{C J/C} = \dl{1.602 \times 10^{-19} \unit{J} = 1 \unit{eV}} &&
		\end{flalign*}
	\tcbline
		\item \(  \)
		\begin{enumerate}
			\renewcommand{\labelenumii}{(\roman{enumii})}
			\item \( m_{\text{e}} = 0.510999 \unit{MeV/c^2} = 0.510999 \frac{e}{c^2} \unit{J/C} = \dl{9.11 \times 10^{-31} \unit{kg}} \)
			\item \( m_{\text{p}} = 938.272 \unit{MeV/c^2} = 938.272 \frac{e}{c^2} \unit{J/C} = \dl{1.67 \times 10^{-27} \unit{kg}} \)
			\item \( m_{\text{d}} = 1875.613 \unit{MeV/c^2} = 1875.613 \frac{e}{c^2} \unit{J/C} = \dl{3.34 \times 10^{-27} \unit{kg}} \)
		\end{enumerate}
	\tcbline
		\item Der Raumwinkel beschreibt das Verhältnis zwischen der Teilfläche einer Kugel zum Kugelradius \( r \) zum Quadrat.
		\begin{flalign*}
			\dd{\Omega} &= \tfrac{\dd{A}}{r^2} = \dl{\sin(\theta)\dd{\theta}\dd{\varphi}} &&
		\end{flalign*}
	\tcbline
		\item \(  \)
		\begin{flalign*}
			A &= 2\pi r^2 &&\\
			\Omega &= \tfrac{A}{r^2} 
				= \dl{2\pi \unit{\steradian}} &&\\
			\Omega &= 2\pi \left( \tfrac{180}{\pi} \right)^2 
				= \dl{\tfrac{64800}{\pi} \unit{\deg\squared} 
				= 20626.5  \unit{\deg\squared}} &&
		\end{flalign*}
	\tcbline
		\item \( \theta = \ang{1} \)
		\begin{flalign*}
			A &= 2\pi r^2 \left(1 - \cos(\theta) + \tfrac{1}{2} \sin^{2}(\theta) \right) &&\\
			\Omega &= \tfrac{A}{r^2} 
				= 2\pi \left(1 - \cos(\theta) + \tfrac{1}{2} \sin^{2}(\theta) \right) 
				= \dl{5.11284 \unit{sr}} &&
		\end{flalign*}
	\end{enumerate}
\end{mybox}

\begin{mybox}{Relativistische Formel der Energie}
	\centering \(  \)
	\tcblower
	\begin{enumerate}
		\item \( p = \gamma m_0v;\quad \gamma = \tfrac{1}{\sqrt{1-\beta^2}};\quad \beta = \tfrac{v}{c};\quad F = \dv{p}{t} \)
		\begin{flalign*}
			E_{\text{kin}} &= \uint[0,r]{F}{r} 
				= \uint[0,r]{\dv{p}{t}}{r} 
				= \uint[0,r]{\dv{}{t}\left( \frac{m_0v}{\sqrt{1-\beta^2}} \right)}{r} &&
		\end{flalign*}
	\tcbline
		\item \( \dd{p} = m_0\dv{(\gamma v)}{t}\dd{t} = m_0\gamma (1+\gamma^2\beta^2) \dd{v} \)
		\begin{flalign*}
			E_{\text{kin}} &= \uint[0,r]{\dv{p}{t}}{r} = \uint[0,p]{\dv{r}{t}}{\tilde{p}} 
				= m_0 \uint[0,v]{\gamma \tilde{v}\left( 1 + \gamma^2\beta^2 \right)}{\tilde{v}} 
				= \dl{m_0c^2(\gamma - 1)} &&
		\end{flalign*}
	\tcbline
		\item \( \gamma \approx 1 + \tfrac{1}{2}\beta^2 \)
		\begin{flalign*}
			E_{\text{kin}} &= m_0c^2(\gamma - 1) \approx m_0c^2 \left( 1 + \tfrac{1}{2}\beta^2 - 1 \right) = \dl{\tfrac{m_0v^2}{2}} &&
		\end{flalign*}
	\end{enumerate}
\end{mybox}

\begin{mybox}{Streuung an harter Kugel}
	\centering \(  \)
	\tcblower
	\begin{enumerate}
		\item \( 2\alpha + \theta = \pi;\quad R = R_1 + R_2 \)
		\begin{flalign*}
			\tfrac{b}{R} &= \sin(\alpha) 
				= \sin(\tfrac{\pi}{2} - \tfrac{\theta}{2})
				= \cos(\tfrac{\theta}{2}) &&\\
			b &= R\cos(\tfrac{\theta}{2}) &&\\
			\dd{b} &= -\tfrac{R}{2}\sin(\tfrac{\theta}{2}) \dd{\theta} &&
		\end{flalign*}
	\tcbline
		\item \(  \)
		\begin{flalign*}
			\dv{\sigma}{\Omega} &= \abs{\frac{b\dd{b}}{\sin(\theta)\dd{\theta}}}
				= \abs{\frac{\left( R\cos(\tfrac{\theta}{2}) \right) \left( -\tfrac{R}{2}\sin(\tfrac{\theta}{2}) \dd{\theta} \right)}{\sin(\theta)\dd{\theta}}}
				= \abs{\frac{R^2\cos(\tfrac{\theta}{2})\sin(\tfrac{\theta}{2})}{2\sin(\theta)}}
				= \dl{\frac{R^2}{4}} &&
		\end{flalign*}
	\tcbline
		\item \(  \)
%		\begin{flalign*}
		%			
%		\end{flalign*}
	\tcbline
		\item \(  \)
%		\begin{flalign*}
%			
%		\end{flalign*}
	\end{enumerate}
\end{mybox}


\end{document}