% !TeX spellcheck = de_DE
\documentclass{alex_hü}

\name{Alexander Helbok}
\course{PS Physik}
\hwnumber{3}


\begin{document}
\renewcommand{\labelenumi}{\alph{enumi})}


\begin{mybox}{Coulombterm des Tröpfchenmodells}
	\begin{enumerate}
		\item \( E = \tfrac{3}{5}\tfrac{Q^2}{4\pi\epsilon_0}\tfrac{A^{-1/3}}{R_0} = a_CZ^2 A^{-1/3};\quad a_{\text{C}} \approx 0.714 \unit{MeV};\quad R_0 = 1.3(1) \unit{\femto\meter} \)
		\begin{flalign*}
			\tilde{a}_{\text{C}} &= \tfrac{3e^2}{20\pi\epsilon_0R_0}
				= 0.106(8) \unit{\pico\joule} 
				= \dl{0.66(5) \unit{MeV}} &&\\
		\end{flalign*}
		Der hier bestimmte Wert für \( \tilde{a}_{\text{C}} \) stimmt im Rahmen der Unsicherheit mit dem experimentell ermittelten von \( a_{\text{C}} \) überein.
	\end{enumerate}
\end{mybox}

\begin{mybox}{Fermigasmodell + Weiße Zwerge}
	\centering \(  \)
	\tcblower
	\begin{enumerate}
		\item Der Fermi Impuls eines Kerns ist der Impuls, des energiereichsten Nukleons, welches im Fermigasmodell das höchste erlaubt Energieniveau besetzt.
	\tcbline
		\item \( V_{x} = \tfrac{4}{3}\pi R^3;\quad V_{p} = \tfrac{4}{3}\pi p_{\text{F}}^3 \)
		\begin{flalign*}
			n &= 2\tfrac{V_{x}V_{p}}{h^3} 
				= \tfrac{4R^3p_{\text{F}}^3}{9\pi\hbar^3} 
		\end{flalign*}
	\tcbline
		\item \( N = \tfrac{A}{2};\quad R = R_0A^{1/3};\quad R_0 = 1.3 \unit{\femto\meter} \)
		\begin{flalign*}
			\tfrac{A}{2} &= \tfrac{4R_0^3A^3p_{\text{F}}^3}{9\pi\hbar^3} &&\\
			\Rightarrow p_{\text{F}} &= \tfrac{\hbar}{2R_0}\sqrt[3]{9\pi}
				\approx 1.24 \times 10^{-19} \unit{Ns}
				= \dl{231.21 \unit{\mega\electronvolt\per c}} &&
		\end{flalign*}
	\tcbline
	\item \(  \)
		\begin{flalign*}
			E_{\text{F}} &= \tfrac{p_{\text{F}}^2}{2m_{\text{N}}} 
				= \tfrac{\hbar^2}{8m_{\text{N}}R_0^2}\left(9\pi\right)^{2/3} 
				= \dl{28.49 \unit{MeV}} &&
		\end{flalign*}
	\tcbline
	\item \( E_{\text{kin}} = \tfrac{p^2}{2m_{\text{N}}};\quad \rho(\vec {p}) = \rho_0 = \tfrac{3}{4\pi p_{\text{F}}^3} \)
		\begin{flalign*}
			\langle E_{\text{kin}} \rangle &= \uint[\mathbb{R}^3]{\rho_0\tfrac{\vec{p}^2}{2m_{\text{N}}}}{\vec{p}} 
				= \tfrac{\rho_0}{2m_{\text{N}}} \uint[0,2\pi]{\uint[0,\pi]{\uint[0,p_{\text{F}}]{p^4 \sin(\theta)}{p}}{\theta}}{\varphi}
				= \tfrac{2\pi\rho_0p_{\text{F}}^5}{5m_{\text{N}}} 
				= \tfrac{3p_{\text{F}}^2}{10m_{\text{N}}} &&\\	
			&= \dl{\tfrac{3}{5}E_{\text{F}}} &&
		\end{flalign*}
	\tcbline
	\item \(  \)
%		\begin{flalign*}
%		
%		\end{flalign*}
	\tcbline
	\item \(  \)
%		\begin{flalign*}
%		
%		\end{flalign*}
	\tcbline
	\item \(  \)
%		\begin{flalign*}
%		
%		\end{flalign*}
	\end{enumerate}
\end{mybox}

\begin{mybox}{Schalenmodell}
	\centering \(  \)
	\tcblower
	\begin{enumerate}
		\item \(  \)
%		\begin{flalign*}
		%			
%		\end{flalign*}
	\tcbline
		\item \(  \)
%		\begin{flalign*}
	%		
%		\end{flalign*}
	\tcbline
		\item \(  \)
%		\begin{flalign*}
		%			
%		\end{flalign*}
	\end{enumerate}
\end{mybox}


\end{document}