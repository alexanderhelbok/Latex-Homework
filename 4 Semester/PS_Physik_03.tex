% !TeX spellcheck = de_DE
\documentclass{alex_hü}

\name{Alexander Helbok}
\course{PS Physik}
\hwnumber{3}


\begin{document}
\renewcommand{\labelenumi}{\alph{enumi})}


\begin{mybox}{Coulombterm des Tröpfchenmodells}
	\begin{enumerate}
		\item \( E = \tfrac{3}{5}\tfrac{Q^2}{4\pi\epsilon_0}\tfrac{A^{-1/3}}{R_0} = a_CZ^2 A^{-1/3};\quad a_{\text{C}} \approx 0.714 \unit{MeV};\quad R_0 = 1.3(1) \unit{\femto\meter} \)
		\begin{flalign*}
			\tilde{a}_{\text{C}} &= \tfrac{3e^2}{20\pi\epsilon_0R_0}
				= 0.106(8) \unit{\pico\joule} 
				= \dl{0.66(5) \unit{MeV}} &&\\
		\end{flalign*}
		Der hier bestimmte Wert für \( \tilde{a}_{\text{C}} \) stimmt im Rahmen der Unsicherheit mit dem experimentell ermittelten von \( a_{\text{C}} \) überein.
	\end{enumerate}
\end{mybox}

\begin{mybox}{Fermigasmodell + Weiße Zwerge}
	\centering \(  \)
	\tcblower
	\begin{enumerate}
		\item Der Fermi Impuls eines Kerns ist der Impuls, des energiereichsten Nukleons, welches im Fermigasmodell das höchste erlaubte Energieniveau besetzt.
	\tcbline
		\item \( V_{x} = \tfrac{4}{3}\pi R^3;\quad V_{p} = \tfrac{4}{3}\pi p_{\text{F}}^3 \)
		\begin{flalign*}
			n &= 2\tfrac{V_{x}V_{p}}{h^3} 
				= \tfrac{4R^3p_{\text{F}}^3}{9\pi\hbar^3} &&
		\end{flalign*}
	\tcbline
		\item \( N = \tfrac{A}{2};\quad R = R_0A^{1/3};\quad R_0 = 1.3 \unit{\femto\meter} \)
		\begin{flalign*}
			\tfrac{A}{2} &= \tfrac{4R_0^3A^3p_{\text{F}}^3}{9\pi\hbar^3} &&\\
			\Rightarrow p_{\text{F}} &= \tfrac{\hbar}{2R_0}\sqrt[3]{9\pi}
				\approx 1.24 \times 10^{-19} \unit{Ns}
				= \dl{231.21 \unit{\mega\electronvolt\per c}} &&
		\end{flalign*}
	\tcbline
	\item \(  \)
		\begin{flalign*}
			E_{\text{F}} &= \tfrac{p_{\text{F}}^2}{2m_{\text{N}}} 
				= \tfrac{\hbar^2}{8m_{\text{N}}R_0^2}\left(9\pi\right)^{2/3} 
				= \dl{28.49 \unit{MeV}} &&
		\end{flalign*}
	\tcbline
	\item \( E_{\text{kin}} = \tfrac{p^2}{2m_{\text{N}}};\quad \rho(\vec {p}) = \rho_0 = \tfrac{3}{4\pi p_{\text{F}}^3} \)
		\begin{flalign*}
			\langle E_{\text{kin}} \rangle &= \uint[\mathbb{R}^3]{\rho_0\tfrac{\vec{p}^2}{2m_{\text{N}}}}{\vec{p}} 
				= \tfrac{\rho_0}{2m_{\text{N}}} \uint[0,2\pi]{\uint[0,\pi]{\uint[0,p_{\text{F}}]{p^4 \sin(\theta)}{p}}{\theta}}{\varphi}
				= \tfrac{2\pi\rho_0p_{\text{F}}^5}{5m_{\text{N}}} 
				= \tfrac{3p_{\text{F}}^2}{10m_{\text{N}}} &&\\	
			&= \dl{\tfrac{3}{5}E_{\text{F}}} &&
		\end{flalign*}
	\tcbline
	\item \( n = \tfrac{Vp_{\text{F}}^3}{3\pi^2\hbar^3} \)
		\begin{flalign*}
			E_{\text{F}} &= \tfrac{p_{\text{F}}^2}{2m_{\text{n}}} 
				= \tfrac{1}{2m_{\text{n}}} \left(\hbar\, \sqrt[3]{\tfrac{3n\pi^2}{V}} \right)^2
				= \dl{\tfrac{\hbar^2}{2m_{\text{n}}} \left( \tfrac{3\pi^2n}{V} \right)^{2/3}} &&
		\end{flalign*}
	\tcbline
	\item \( E_{\text{F}} = \tfrac{\hbar^2}{2m_{\text{e}}} \left(\tfrac{9\pi Z}{4R^3}\right)^{2/3};\quad E_{\text{G}} = -\tfrac{3}{5}G\tfrac{M}{R^2};\quad E_{\text{ges}} = E_{\text{kin}} + E_{\text{G}} \)
		\begin{flalign*}
			\pdv{E_{\text{ges}}}{R} &= \frac{3 \left(4 G M^2 m_{\text{e}} - 3 \hbar^2 R Z \left(\frac{12\pi^2 Z^^2}{R^6}\right)^{1/3}\right)}{20 m_{\text{e}} R^2}
				\stackrel{!}{=} 0 &&\\[2ex]
			\Rightarrow R &= \tfrac{3\hbar^2 \sqrt[3]{3\pi^2Z^5}}{2\sqrt[3]{2}GM^2m_{\text{e}}} \propto M^{-2} &&
		\end{flalign*}
	\tcbline
	\item \( M_{\text{S}} = 1.99 \times 10^{30} \unit{kg};\quad Z \approx \tfrac{A}{2} \Rightarrow Z = \tfrac{M_{\text{S}}}{m_{\text{e}} + m_{\text{p}} + m_{\text{n}}} \)
		\begin{flalign*}
			R &= \tfrac{3\hbar^2 \sqrt[3]{3\pi^2Z^5}}{2\sqrt[3]{2}GM_{\text{S}}^2m_{\text{e}}} 
					= \dl{7.15 \times 10^{6} \unit{m}} &&\\[2ex]
			\rho_{\text{S}} &= \tfrac{M_{\text{S}}}{V} 
				= \tfrac{3M_{\text{S}}}{4\pi R^3} 
				= \dl{1.3 \times 10^{9} \unit{kg/m^3}} &&\\[2ex]
			\rho_{\text{E}} &= 5515 \unit{kg/m^3} 
				\approx 10^{-5} \rho_{\text{S}} &&
		\end{flalign*}
	\end{enumerate}
\end{mybox}

\begin{mybox}{Schalenmodell}
	\centering \(  \)
	\tcblower
	\begin{enumerate}
		\item \(  \)
		\begin{flalign*}
			\Delta E &= \left[ E_{\text{B}}(\ch{^{16}O}) - E_{\text{B}}(\ch{^{17}O}) \right] - \left[ E_{\text{B}}(\ch{^{15}O})	- E_{\text{B}}(\ch{^{16}O}) \right]
				= \dl{11.52 \unit{MeV}} &&
		\end{flalign*}
	\tcbline
		\item Protonen liegen aufgrund des Coulomb Potentials weniger tief bzw. die Energieniveaus stimmen weder in Höhe noch in Abstand mit denen von Neutronen überein. Deshalb kommt es zu einer Diskrepanz in der Energie, wenn man ein Neutron durch ein Proton ersetzt (was zur Entstehung neuer Teilchen führen kann, \\zb. \ch{_8^{17}O ->[\( \beta^- \)] _9^{17}F + e^{-} + \( \overline{\nu}_{\text{e}} \)}).
	\end{enumerate}
\end{mybox}


\end{document}