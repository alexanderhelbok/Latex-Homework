% !TeX spellcheck = de_DE
\documentclass{alex_hü}

\name{Alexander Helbok}
\course{PS Physik}
\hwnumber{3}


\begin{document}
\renewcommand{\labelenumi}{\alph{enumi})}


\begin{mybox}{Coulombterm des Tröpfchenmodells}
	\centering \(  \)
	\tcblower
	\begin{enumerate}
		\item \( E = \tfrac{5}{3}\tfrac{Q^2}{4\pi\epsilon_0}, a_C = Z^2 A^{1/3} \)
		\begin{flalign*}
			_{\text{C}} &= \tfrac{3e^2}{20\pi R_0e_0}
		\end{flalign*}
	\end{enumerate}
\end{mybox}

\begin{mybox}{Fermigasmodell + Weiße Zwerge}
	\centering \(  \)
	\tcblower
	\begin{enumerate}
		\item Der Fermi Impuls eines Kerns ist der Impuls, des energiereichsten Nukleons, welches im Fermigasmodell das höchste erlaubt Energieniveau besetzt.
	\tcbline
		\item \(  \)
%		\begin{flalign*}
	%		
%		\end{flalign*}
	\tcbline
		\item \(  \)
%		\begin{flalign*}
		%			
%		\end{flalign*}
	\end{enumerate}
\end{mybox}

\begin{mybox}{Schalenmodell}
	\centering \(  \)
	\tcblower
	\begin{enumerate}
		\item \(  \)
%		\begin{flalign*}
		%			
%		\end{flalign*}
	\tcbline
		\item \(  \)
%		\begin{flalign*}
	%		
%		\end{flalign*}
	\tcbline
		\item \(  \)
%		\begin{flalign*}
		%			
%		\end{flalign*}
	\end{enumerate}
\end{mybox}


\end{document}