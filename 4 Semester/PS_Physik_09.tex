% !TeX spellcheck = de_DE
\documentclass{alex_hü}

%\usepackage[mtpscr,mtpccal]{mtpro2}

\name{Alexander Helbok}
\course{PS Physik}
\hwnumber{9}

\NewDocumentCommand{\g}{}{g_{\mu\nu}}
\NewDocumentCommand{\lam}{}{\Lambda_{\mu}^{\nu'}}
\RenewDocumentCommand{\P}{m}{\mathscr{P}^{#1}}

\begin{document}
\renewcommand{\labelenumi}{\alph{enumi})}


\begin{mybox}{Relativistik und invariante Masse}
	\centering \( \P{\mu} = (E/c, p^{\mu});\quad \P{\nu'} = \lam\P{\mu} \)
	\tcblower
	\begin{enumerate}
		\item \(  \)
		\begin{flalign*}
			\braket{\P{}}{\P{}} &= \P{\mu}\P{\nu}\g
				= \tfrac{E^2}{c^2} - p_x^2 - p_y^2 - p_z^2 
				= \tfrac{E^2}{c^2} - p^2 &&\\[2ex]
			\braket{\P{'}}{\P{'}} &= \lam\P{\mu}\lam\P{\nu}\g
				= (\gamma\tfrac{E}{c} + \gamma\beta p_x)^2 - (\gamma\beta\tfrac{E}{c} + \gamma p_x)^2 - p_y^2 - p_z^2 &&\\
			&= (\gamma^2 - \gamma^2\beta^2)\tfrac{E^2}{c^2} - (\gamma^2 - \gamma^2\beta^2)p_x^2 - p_y^2 - p_z^2 &&\\
			&= \tfrac{E^2}{c^2} - p_x^2 - p_y^2 - p_z^2 
				= \tfrac{E^2}{c^2} - p^2 &&
		\end{flalign*}
	\tcbline
		\item \( \P{\mu} = (E/c, p^{\mu}) = (m_0c, 0) \)
		\begin{flalign*}
			\braket{\P{}}{\P{}} &= m_0^2c^2 
				\stackrel{a)}{=} \tfrac{E^2}{c^2} - p^2 &&\\[2ex]
			\Rightarrow E^2 &= p^2c^2 - m_0^2c^4 &&
		\end{flalign*}
	\end{enumerate}
\end{mybox}

\begin{mybox}{Pionen-Erzeugung}
	\centering \( \P{\mu}_{p1} = (m_p c, 0);\quad \P{\mu}_{p2} = (E/c, p^{\mu});\quad \P{\mu}_{\pi} = (m_{\pi}c, 0)  \)
	\tcblower
%	
\end{mybox}

\begin{mybox}{Teilchenzerfall im Laborsystem}
	\centering \(  \)
	\tcblower
	\begin{enumerate}
		\item \(  \)
		\begin{tabular}{c | c}
			Schwerpunktsystem & Laborsystem \\
			\begin{tikzpicture}
				\draw[-Latex] (0, 0) -- (0, 1) node [midway, right] {\( \pi^- \)};
				\draw[-Latex] (0, 0) -- (0, -1) node [midway, right] {\( p \)};
				\filldraw[black] (0,0) circle (2pt) node [left] {\( \Delta^0 \)};
			\end{tikzpicture} & 
			\begin{tikzpicture}
				\draw[-Latex] (0, 0) -- (1, 0) node [midway, above] {\( \Delta^0 \)};
				\draw[-Latex] (1, 0) -- (2, 1) node [pos=0.7, below right] {\( p \)};
				\draw[-Latex] (1, 0) -- (2, -1) node [pos=0.7, above right] {\( \pi^- \)};
			\end{tikzpicture}
		\end{tabular}
	\tcbline
		\item \( E_{\Delta} = 1.35 \unit{GeV} \)
		\begin{flalign*}
			E_{\Delta} &= \gamma m_{\Delta}c^2
				= \frac{m_{\Delta}c^2}{\sqrt{1-\beta^2}} &&\\[2ex]
			\Rightarrow \beta &= \sqrt{1 - \frac{m_{\Delta}^2c^4}{E_{\Delta}^2}}
				= \dl{} &&
		\end{flalign*}
	\tcbline
		\item \(  \)
		\begin{flalign*}
			\P{\mu}_{p} &= () &&\\
			\P{\mu}_{\pi} &= () &&
		\end{flalign*}
	\tcbline
		\item \(  \)
		\begin{flalign*}
			\P{\nu'}_{p} &= \lam\P{\mu}_{p} = () &&\\
			\P{\nu'}_{\pi} &= \lam\P{\mu}_{\pi} = () &&\\
		\end{flalign*}
	\tcbline
		\item \(  \)
%		\begin{flalign*}
%			
%		\end{flalign*}
	\tcbline
		\item \(  \)
%		\begin{flalign*}
%		
%		\end{flalign*}
	\end{enumerate}
\end{mybox}


\end{document}