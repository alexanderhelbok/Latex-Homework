% !TeX spellcheck = de_DE
\documentclass{alex_hü}

\name{Alexander Helbok}
\course{PS Physik}
\hwnumber{9}

\NewDocumentCommand{\g}{}{g_{\mu\nu}}
\NewDocumentCommand{\lam}{}{\Lambda_{\mu}^{\nu'}}
\RenewDocumentCommand{\P}{m}{\mathscr{P}^{#1}}
\NewDocumentCommand{\mpro}{}{m_{\text{p}}}

\begin{document}
\renewcommand{\labelenumi}{\alph{enumi})}


\begin{mybox}{Relativistik und invariante Masse}
	\centering \( \P{\mu} = (E/c, p^{\mu});\quad \P{\nu'} = \lam\P{\mu} \)
	\tcblower
	\begin{enumerate}
		\item \(  \)
		\begin{flalign*}
			\braket{\P{}}{\P{}} &= \P{\mu}\P{\nu}\g
				= \tfrac{E^2}{c^2} - p_x^2 - p_y^2 - p_z^2 
				= \tfrac{E^2}{c^2} - p^2 &&\\[2ex]
			\braket{\P{'}}{\P{'}} &= \lam\P{\mu}\lam\P{\nu}\g
				= (\gamma\tfrac{E}{c} + \gamma\beta p_x)^2 - (\gamma\beta\tfrac{E}{c} + \gamma p_x)^2 - p_y^2 - p_z^2 &&\\
			&= (\gamma^2 - \gamma^2\beta^2)\tfrac{E^2}{c^2} - (\gamma^2 - \gamma^2\beta^2)p_x^2 - p_y^2 - p_z^2 &&\\
			&= \tfrac{E^2}{c^2} - p_x^2 - p_y^2 - p_z^2 
				= \tfrac{E^2}{c^2} - p^2 &&
		\end{flalign*}
	\tcbline
		\item \( \P{\mu} = (E/c, p^{\mu}) = (m_0c, 0) \)
		\begin{flalign*}
			\braket{\P{}}{\P{}} &= m_0^2c^2 
				\stackrel{a)}{=} \tfrac{E^2}{c^2} - p^2 &&\\[2ex]
			\Rightarrow E^2 &= p^2c^2 + m_0^2c^4 &&
		\end{flalign*}
	\end{enumerate}
\end{mybox}

\begin{mybox}{Pionen-Erzeugung}
	\centering \( \P{\mu}_{p1} = (\mpro, 0);\quad \P{\mu}_{p2} = (E, p^{\mu});\quad \P{\mu}_{\pi} = (m_{\pi}, 0)  \)
	\tcblower
	\begin{align*}
		(\P{\mu}_{p1} + \P{\mu}_{p2})(\P{\nu}_{p1} + \P{\nu}_{p2})\g &= (2\P{\mu}_{p1} + \P{\mu}_{\pi})(2\P{\nu}_{p1} + \P{\nu}_{\pi})\g \\
		E^2 + 2E\mpro + \mpro^2 - p^2 &= (m_\pi + 2\mpro)^2 \\
		p^2 + \mpro^2 + 2E\mpro + \mpro^2 - p^2 &= (m_\pi + 2\mpro)^2 \\
		\gamma\mpro + \mpro &= \tfrac{(m_\pi + 2\mpro)^2}{2\mpro} \\
		T = (\gamma - 1)\mpro c^2 &= (\tfrac{(m_\pi + 2\mpro)^2}{2\mpro} - 2\mpro)c^2 = \dl{279.66 \unit{MeV}} 
	\end{align*}
\end{mybox}

\begin{mybox}{Teilchenzerfall im Laborsystem}
	\centering \(  \)
	\tcblower
	\begin{enumerate}
		\item \(  \)
		\begin{tabular}{c | c}
			Schwerpunktsystem & Laborsystem \\
			\begin{tikzpicture}
				\draw[-Latex] (0, 0) -- (0, 1) node [midway, right] {\( \pi^- \)};
				\draw[-Latex] (0, 0) -- (0, -1) node [midway, right] {\( p \)};
				\filldraw[black] (0,0) circle (2pt) node [left] {\( \Delta^0 \)};
			\end{tikzpicture} & 
			\begin{tikzpicture}
				\draw[-Latex] (0, 0) -- (1, 0) node [midway, above] {\( \Delta^0 \)};
				\draw[-Latex] (1, 0) -- (3, 1) node [pos=0.3, above left] {\( p \)};
				\draw[-Latex] (1, 0) -- (2, -1) node [pos=0.7, below left] {\( \pi^- \)};
				\draw[dashed] (1, 0) -- (4, 0);
				
				\draw (2.6, 0) arc (0:27:1.6);
				\node[above] at (2,0) {\small$\theta$};
				
				\draw (2.2, 0) arc (0:-45:1.2);
				\node[below] at (1.8, -0.1) {\small$\varphi$};
			\end{tikzpicture}
		\end{tabular}
	\tcbline
		\item \( E_{\Delta} = 1.35 \unit{GeV} \)
		\begin{flalign*}
			E_{\Delta} &= \gamma m_{\Delta}c^2
				= \frac{m_{\Delta}c^2}{\sqrt{1-\beta^2}} &&\\[2ex]
			\Rightarrow \beta &= \sqrt{1 - \frac{m_{\Delta}^2c^4}{E_{\Delta}^2}}
				= \dl{0.41} &&
		\end{flalign*}
	\tcbline
		\item \( E_1 = \tfrac{(m_0^2 + m_1^2 - m_2^2)c^2}{2m_0} \)
		\begin{flalign*}
			\P{\mu}_{p} &= (0.97, 0, 0.23, 0) \unit{GeV/c} &&\\
			\P{\mu}_{\pi} &= (0.27, 0, -0.23, 0) \unit{GeV/c} &&
		\end{flalign*}
	\tcbline
		\item \(  \)
		\begin{flalign*}
			\P{\nu'}_{p} &= \lam\P{\mu}_{p} = (1.06, 0.12, 0.23, 0) \unit{GeV/c} &&\\
			\P{\nu'}_{\pi} &= \lam\P{\mu}_{\pi} = (0.29, 0.43, -0.23, 0) \unit{GeV/c} &&\\[4ex]
			\P{\nu'}_{p} &+ \P{\nu'}_{\pi} = (1.35, 0.55, 0, 0) \unit{GeV/c} = \P{\nu'}_{\Delta} &&
		\end{flalign*}
	\tcbline
		\item \(  \)
		\begin{flalign*}
			\theta &= \arctan(\frac{\P{2'}_p}{\P{1'}_p})
				= \dl{\ang{27.71}} &&\\
			\varphi &= \arctan(\frac{\P{2'}_\pi}{\P{1'}_\pi})
				= \dl{\ang{62.26}} &&\\	
			\theta &+ \varphi = \ang{89.97} &&
		\end{flalign*}
	\tcbline
		\item Im Schwerpunktsystem sind die Impulse der Teilchen betragsmäßig gleich, was bedeutet, dass das Proton aufgrund seiner geringeren Masse schneller ist als das Pion. Die Transformation ins Laborsystem beeinflusst die vertikale Komponente des Impulses nicht, wodurch das Proton auch im Laborsystem das schnellere Teilchen ist (die x-Komponenten der Geschwindigkeit sind nach gemeinsamer Lorentz-Trafo gleich).
	\end{enumerate}
\end{mybox}


\end{document}