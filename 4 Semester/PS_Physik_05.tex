% !TeX spellcheck = de_DE
\documentclass{alex_hü}

\name{Alexander Helbok}
\course{PS Physik}
\hwnumber{5}


\begin{document}
\renewcommand{\labelenumi}{\alph{enumi})}


\begin{mybox}{Radiokarbonmethode}
	
\end{mybox}

\begin{mybox}{Theorie des Alpha-Zerfalls und Alternative}
	\centering \(  \)
	\tcblower
	\begin{enumerate}
		\item 
	\tcbline
		\item 
		\begin{flalign*}
			\ch{^{228}Th &-> ^{224}Ra + \( \alpha \) -> ^{220}Rn + \( \alpha \) -> ^{116}Po + \( \alpha \)} &&\\
			\ch{^{228}Th &-> ^{116}Po + ^{12}C} &&
		\end{flalign*}
	\tcbline
		\item \(  \)
%		\begin{flalign*}
		%			
%		\end{flalign*}
	\end{enumerate}
\end{mybox}

\begin{mybox}{Energieversorgung des Perseverance Rovers}
	\centering \(  \)
	\tcblower
	\begin{enumerate}
		\item Der MMRTG startet mit einer Leistung von \( 110 \unit{W} \) und hat eine operational lifetime von 14 Jahren.
	\tcbline
		\item \(  \)
%		\begin{flalign*}
	%		
%		\end{flalign*}
	\tcbline
		\item \( P_1 = 0.9 \unit{kWh};\quad P_2 = 110 \unit{W};\quad A_1 = 1.3 \unit{m^2};\quad t = 1 \unit{Marstag} = 88642.66 \unit{s} \)
		\begin{flalign*}
			\frac{P_1}{t} &= 36.55 \unit{W} &&\\[2ex]
			A_2 &= \frac{P_2 t}{P_1} A_1
				= \dl{3.91 \unit{m^2}} &&
		\end{flalign*}
	\end{enumerate}
\end{mybox}


\end{document}