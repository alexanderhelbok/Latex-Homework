% !TeX spellcheck = de_DE
\documentclass{alex_hü}
\usepackage{biblatex}
\name{Alexander Helbok}
\course{PS Physik}
\hwnumber{5}
\newcommand{\Th}{\ch{^{228}Th}}
\newcommand{\Po}{\ch{^{116}Po}}
\newcommand{\C}{\ch{^{12}C}}
\newcommand{\Pu}{\ch{^{238}Pu}}
\newcommand{\U}{\ch{^{234}U}}

\begin{document}
\renewcommand{\labelenumi}{\alph{enumi})}


\begin{mybox}{Radiokarbonmethode}
	\( t_{1/2} \approx 5730 \unit{yr};\quad \lambda = \tfrac{\ln(2)}{t_{1/2}};\quad M = 15.7 \unit{\gram};\quad m_{\text{C}} \approx 12 \unit{u};\quad n = \tfrac{M}{m_{\text{C}}} \)
	\begin{flalign*}
		N_0 &= \frac{n}{8 \cdot 10^{11}}  &&\\
		\tfrac{1}{3} \unit{Bq} &= \lambda N_0 \expo[-][\lambda t] &&\\
		t &= -\tfrac{\ln(\tfrac{1 \unit{Bq}}{3\lambda N_0})}{\lambda} 	
			= \dl{9.85 \cdot 10^{11} \unit{s}} \approx 20070 \unit{yr} &&
	\end{flalign*} 
\end{mybox}

\begin{mybox}{Theorie des Alpha-Zerfalls und Alternative}
	\centering \(  \)
	\tcblower
	\begin{enumerate}
		\item
		\begin{flalign*}
			\text{\RN{1}}:\ \ch{^{228}Th &-> ^{224}Ra + \( \alpha \) -> ^{220}Rn + \( \alpha \) -> ^{116}Po + \( \alpha \)} &&\\
			\text{\RN{2}}:\ \ch{^{228}Th &-> ^{116}Po + ^{12}C} &&
		\end{flalign*}
	\tcbline
		\item 
		\begin{flalign*}
			\Delta M_{\text{\RN{1}}} &= M_{\Th} - 3M_{\alpha} - M_{\Po} 
				= \dl{17.71 \unit{MeV/c^2}} &&\\
			\Delta M_{\text{\RN{2}}} &= M_{\Th} - M_{\C} - M_{\Po} 
			= \dl{24.99 \unit{MeV/c^2}}&&
		\end{flalign*}
		Der \C-Zerfall sollte häufiger vorkommen, da in Summe mehr Energie freigesetzt wird und das Endprodukt somit einen energetisch niedrigeren Zustand einnimmt.
	\tcbline
		\item \( k = \tfrac{1}{4\pi\varepsilon_0};\quad R_0 = 1.3 \unit{fm} \)
		\begin{flalign*}
			V_{\text{\RN{1}}} &= k\tfrac{e^2Z_{\alpha} Z_{\text{Ra}}}{R_0 \sqrt[3]{A_{\alpha} + A_{\text{Ra}}}} 
				= \dl{25.45 \unit{MeV}} &&\\[2ex]
			V_{\text{\RN{2}}} &= k\tfrac{e^2Z_{\text{C}} Z_{\text{Po}}}{R_0 \sqrt[3]{A_{\text{C}} + A_{\text{Po}}}} 
				= \dl{115.60 \unit{MeV}} &&	
		\end{flalign*}
	\tcbline
		\item \( G = \tfrac{2}{\hbar} \uint{\sqrt{2m(V_i(r) - E_i)}}{r};\quad r_i = k\tfrac{e^2Z_iZ_j}{E_i} = \tfrac{K_i}{E_i};\quad V_i(r) = k\tfrac{e^2Z_iZ_j}{r} = \tfrac{K_i}{r} \)
		\begin{flalign*}
			G_{i} &= \frac{2}{\hbar} \uint[R, r_i]{\sqrt{2m(V_i(r) - E_i)}}{r} 
				= \frac{\sqrt{8m}}{\hbar} \uint[R, r_i]{\SQRT{\frac{K_i}{r} - E_i}}{r} &&\\
			&= \frac{\sqrt{8mE_i}}{\hbar} \uint[R, r_i]{\SQRT{\frac{r_i}{r} - 1}}{r}
				= \frac{\sqrt{8mE_i}}{\hbar} \left[ r\SQRT{\frac{r_i}{r} - 1} - r_i\arctan(\SQRT{\frac{r_i}{r} - 1}) \right]_{R}^{r_i} &&\\
		\end{flalign*}
		\( r_{\alpha} = 4 \cdot 10^{-14} \unit{m};\quad E_{\alpha} = 5.9 \unit{MeV};\quad r_{\text{C}} = 4.01 \cdot 10^-{14} \unit{m};\quad E_{\text{C}} = 24.99 \unit{MeV} \)
		\begin{flalign*}
			G_{\alpha} &= \dl{64.52} &&\\[1ex]
			G_{\text{C}} &= \dl{205.26} &&
		\end{flalign*}
	\tcbline
		\item \( \lambda = \lambda_0\expo[-][G] \)
		\begin{flalign*}
			\tfrac{m_iv_i^2}{2} &= E_i + V_0 \quad\Rightarrow\quad v_i = \sqrt{\tfrac{2(E_i + V_0)}{m_i}} &&\\
			\lambda_0(i) &= \frac{v_i}{2R} 
				= \sqrt{\tfrac{E_i + V_0}{2m_iR^2}} &&\\[2ex]
			\lambda_0(\alpha) &= \dl{2.36 \cdot 10^{21} \unit{1/s}} &&\\
			\lambda_0(\C) &= \dl{1.86 \cdot 10^{21} \unit{1/s}} &&
		\end{flalign*}
	\tcbline
		\item \(  \)
		\begin{flalign*}
			t_{\text{1/2}}(\alpha) &= \frac{\ln(2)}{\lambda} 
				 = \frac{\ln(2)}{\lambda_0}\expo[G] 
				 = \dl{1.47 \cdot 10^6 \unit{s}} &&\\[2ex]
			t_{\text{1/2}}(\C) &= \dl{5.16 \cdot 10^{67} \unit{s}}	&&
		\end{flalign*}
		Der \C-Zerfall ist energetisch zwar günstiger, wird aber so gut wie nicht beobachtet, da die Halbwertszeit sehr sehr groß ist.
	\end{enumerate}
\end{mybox}

\begin{mybox}{Energieversorgung des Perseverance Rovers}
	\centering \(  \)
	\tcblower
	\begin{enumerate}
		\item Der MMRTG startet mit einer Leistung von \( 110 \unit{W} \) und hat eine operational lifetime von 14 Jahren.
	\tcbline
		\item \(  \)
		\begin{flalign*}
			\ch{^{238}Pu &-> ^{234}U + \( \alpha \) + \( \Delta E \)} &&\\
			\Delta E &= (M(\Pu) - M(\U) - M(\alpha))c^2 
				= \dl{5.59 \unit{MeV}} &&
		\end{flalign*}
	\tcbline
		\item \(  \)
		\begin{flalign*}
			P &= 0.06A\Delta E 
				= 0.06N_0\lambda\Delta E 
				\stackrel{!}{=} 110 \unit{MeV} &&\\[2ex]
			N_0 &= \frac{110 \unit{MeV}}{0.06\lambda\Delta E}
				= \dl{8.17 \cdot 10^{24}} &&	
		\end{flalign*}
	\tcbline
		\item \(  \)
		\begin{flalign*}
			N_0M_{\ch{PuO2}} &= N_0(M_{\text{Pu}} + 2M_{\text{O}})
				=  \dl{3.66 \unit{kg}} &&
		\end{flalign*}
		Website: \( 4.8 \unit{kg} \)
	\tcbline
		\item \( t_1 = 14 \unit{\day};\quad t_2 = 14 \unit{yr} \)
		\begin{flalign*}
			P_1 &= N_0\expo[-][\lambda t_1]0.06\lambda\Delta E 
				= \dl{109.52 \unit{W}} &&\\
			P_2 &= N_0\expo[-][\lambda t_2]0.06\lambda\Delta E 
				= \dl{98.48 \unit{W}} &&
		\end{flalign*}
	\tcbline
		\item \( P_1 = 0.9 \unit{kWh};\quad P_2 = 110 \unit{W};\quad A_1 = 1.3 \unit{m^2};\quad t = 1 \unit{Marstag} = 88642.66 \unit{s} \)
		\begin{flalign*}
			\frac{P_1}{t} &= 36.55 \unit{W} &&\\[2ex]
			A_2 &= \frac{P_2 t}{P_1} A_1
				= \dl{3.91 \unit{m^2}} &&
		\end{flalign*}
	\end{enumerate}
\end{mybox}


\end{document}